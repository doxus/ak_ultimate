% !TeX spellcheck = pl_PL

%===============================================================================
% *** PYTANIA I ODPOWIEDZI *****************************************************
%===============================================================================
\part{Pytania zamknięte}
\begin{enumerate}
\begin{minipage}{\textwidth}
	\question{%
		question={Cechy architektury CISC}%
	}{%
	isTrue1={Nie},%
	answer1={Czy może być wykonana w VLIW},%
	explain1={Nie, architektura VLIW dotyczy mikroprocesorów i miała na celu jak największe zmniejszenie jednostki centralnej i jej rozkazów (RISC).},%
	isTrue2={Tak},%
	answer2={Czy występuje model wymiany danych typu pamięć - pamięć},%
	explain2={Tak, posiada również niewielką ilość rejestrów.},%
	isTrue3={Nie},%
	answer3={Jest mała liczba rozkazów},%
	explain3={Nie, w tej architekturze jest PEŁNA (complex) lista rozkazów. Niektóre z zaawansowanych pleceń nawet nie były wykorzystywane, i bum! tak powstał RISC.},%
}
\end{minipage}
\begin{minipage}{\textwidth}
	\question{%
		question={Cechy architektury RISC}%
	}{%
		isTrue1={Tak},%
		answer1={Czy występuje model wymiany danych typu rej-rej},%
		explain1={Tak, a komunikacja z pamięcią operacyjną odbywa się wyłącznie za pomocą rozkazów LOAD i STORE.},%
		isTrue2={Tak},%
		answer2={Jest mała liczba trybów adresowania},%
		explain2={Tak, raptem 4 w procesorze RISC I podczas gdy CISCi mogą mieć ich kilkanaście, w tym takie bardzo złożone.},%
		isTrue3={Nie},%
		answer3={Jest wykonywanych kilka rozkazów w jednym takcie},%
		explain3={Fałsz. Prawdziwe wykonywanie wielu rozkazów w jednym takcie wymaga superskalarnosci - wielu jednostek potokowych. Cechą architektury RISC jest potokowość, ale pojedyncza.},%
		isTrue4={Tak},%
		answer4={Jest wykonywanych kilka rozkazów w jednym takcie (w danej chwili czasu)},%
		explain4={Chodzi o przetwarzanie potokowe. Tu jest haczyk - pierwszy procesor RISC I (1980) stawiał sobie za cel wykonanie \emph{jednego rozkazu w jednym takcie} i dokładnie tak brzmiało jego założenie projektowe. Jednak jego fizyczna realizacja (1982) posiadała dwustopniowy potok. Również w wykładach jako cecha tej architektury jest napisane "Intensywne wykorzystanie przetwarzania potokowego", co odnosi się do faktu, że obecnie nie ma procesora typu RISC, który go nie ma. Wg mnie prawda.},%
		isTrue5={Nie},%
		answer5={Jest wykonywanych kilka instrukcji procesora w jednym rozkazie asemblerowym}, %
		explain5={Nic mi na ten temat nie wiadomo. Brzmi jednak zbyt hardo i odlegle od tematu zmniejszania ilości rozkazów.}, %
		isTrue6={Tak}, %
		answer6={Układ sterowania w postaci logiki szytej}, %
		explain6={Tak.}, %
	}
\end{minipage}
\begin{minipage}{\textwidth}
	\question{%
		question={Okna rejestrów}%
	}{%
		isTrue1={Nie},%
		answer1={Chronią przez hazardem danych},%
		explain1={Lolnope, od tego są mechanizmy skoków opóźnionych i przewidywania rozgałęzień. Okno rejestrów zapewnia ciągłe i optymalne wykonywanie procedur.},%
		isTrue2={Tak},%
		answer2={Minimalizują liczbę odwołań do pamięci operacyjnej przy operacjach wywołania procedur},%
		explain2={Tak, dokładnie do tego one służą. Rejestr niski procedury A staje się rejestrem wysokim procedury B itd. Innymi słowy, procedura A wywołuje procedurę B, i tak dalej. I po coś w tym wszystkim są rejestry globalne.},%
		isTrue3={Nie},%
		answer3={Są charakterystyczne dla architektury CISC},%
		explain3={Nie, zostały zaprojektowane specjalnie dla architektury RISC. Jako pierwszy posiadał je procesor RISC I.},%
		isTrue4={Nie},%
		answer4={Są zamykane po błędnym przewidywaniu wykonania skoków warunkowych.},%
		explain4={W mechanizmie prognozowania rozgałęzień jest możliwość błędnego przewidywania. Jednak błędna prognoza powoduje tylko zmianę strategii (przewidywanie wykonania lub niewykonania), a nie zamykanie okna.},%
		isTrue5={Tak},%
		answer5={Są przesuwane przy operacjach wywołania procedur}, %
		explain5={Tak, z każdą nową wywołaną procedurą okno rejestrów przesuwane jest w dół (ze 137 do 0)}, %
		isTrue6={Nie},%
		answer6={Są przesuwane przy wystąpieniu rozkazów rozgałęzień.}, %
		explain6={W architekturze SPARC okno jest przesuwane rozkazami SAVE i RESTORE, na życzenie programisty, a nie na wskutek rozkazów warunkowych.}, %
	}
\end{minipage}
\begin{minipage}{\textwidth}
	\question{%
		question={Przetwarzanie potokowe:}%
	}{%
		isTrue1={Nie},%
		answer1={Nie jest realizowane dla operacji zmiennoprzecinkowych},%
		explain1={Nie ma takiego ograniczenia. Przetwarzanie potokowe dotyczy optymalizacji czasu wykonywania rozkazów - podziału realizacji rozkazu na fazy. Owszem, dla argumentów zmiennoprzecinkowych mogą wystąpić problemy związane z czasem obliczeń (uniemożliwienie wykonania rozkazu w jednym takcie), co może zablokować napełnianie potoku, jednak nie uniemożliwia to zastosowania potoku.},%
		isTrue2={Nie},%
		answer2={Nie jest realizowane w procesorach CISC},%
		explain2={Przetwarzanie potokowe znalazło zastosowanie głównie w architekturze RISC, jednak CISC też z niej korzysta. Przykłady: VAX 11/780 (CISC), Ultra SPARC III (RISC)},%
		isTrue3={Tak},%
		answer3={Daje przyspieszenie nie większe od liczby segmentów (stopni) jednostki potokowej},%
		explain3={Tak, przyspieszenie jest stosunkiem czasu wykonywania \emph{n} rozkazów dla procesora niepotokowego oraz czasu dla procesora potokowego. W idealnym przypadku, gdy każdy stopień dzieli okres rozkazu po równo, a liczba rozkazów dąży do nieskończoności, stosunek ten jest równy P - ilości stopni.},%
		isTrue4={Nie},%
		answer4={W przypadku wystąpienia zależności między danymi wywołuje błąd i przerwanie wewnętrzne.},%
		explain4={Hm, dobre pytanie. Tak, zależności danych mogą wystąpić (zjawisko hazardu) i rozdupić program, ale po to właśnie istnieją mechanizmy by temu zapobiegać. Każda szanująca się architektura to potrafi: albo sprzętowo, albo na etapie kompilacji, która modyfikuje i optymalizuje program. A jeżeli po modyfikacji pewien rozkaz nie wykona się w jednym takcie, napełnianie potoku jest przerywane (ale błędu chyba nie wywala), patrz wyżej.},%
		isTrue5={Nie},%
		answer5={Jest realizowane tylko dla operacji zmiennoprzecinkowych}, %
		explain5={Pfff, no chyba nie XD Jest realizowane dla każdego rodzaju rozkazu.} %
	}
\end{minipage}
\begin{minipage}{\textwidth}
	\question{%
		question={Mechanizmy potokowe stosowane są w celu:}%
	}{%
		isTrue1={Nie},%
		answer1={Uszeregowania ciągu wykonywanych rozkazów},%
		explain1={Nie, zupełnie nie o to chodzi. Ciąg może zostać uszeregowany przez kompilator w celu optymalizacji. Jednak celem tego mechanizmu jest zrównoleglenie wykonywania rozkazów $ \rightarrow $ zmiana kolejności ich realizacji nie jest założeniem.},%
		isTrue2={Tak},%
		answer2={Uzyskania równoległej realizacji rozkazów},%
		explain2={No tyć. Potoki umożliwiają realizację wielu rozkazów jednocześnie dzieląc jednostkę centralną na wg stopni, jak np. pobranie rozkazu i wykonania rozkazu. Dzięki temu dwa rozkazy mogą wykonywać się jednocześnie, oba w innych fazach (jednostkach czasu).},%
		isTrue3={Tak},%
		answer3={Przyspieszenia realizacji rozkazów},%
		explain3={Tak, to główny cel. Umożliwienie wykonania rozkazów umożliwia przyspieszenie, które oblicza się jako stosunek czasu wykonywania rozkazów w procesorze niepotokowym do czasu realizacji w procesorze potokowym. W idealnym przypadku jest ono równe \emph{P} - ilości podziałów / stopni / faz / zwał jak zwał.},%
	}
\end{minipage}
\begin{minipage}{\textwidth}
	\question{%
		question={Hazard danych:}%
	}{%
		isTrue1={Tak},%
		answer1={Czasami może być usunięty przez zmianę kolejności wykonania rozkazów},%
		explain1={Tak, służy do tego mechanizm skoków opóźnionych, który odbywa się na poziomie kompilacji programu.},%
		isTrue2={Nie},%
		answer2={Nie występuje w architekturze superskalarnej},%
		explain2={Występuje wszędzie tam gdzie jest potokowe przetwarzania rozkazów.},%
		isTrue3={Nie},%
		answer3={Jest eliminowany przez zastosowanie specjalnego bitu w kodzie program},%
		explain3={Nic mi o tym nie wiadomo. Pewne dodatkowe bity są wykorzystywane w mechanizmie przewidywania rozgałęzień, który służy do eliminacji hazardu, jednak on to odbywa się PRZED realizacją programu i sprowadza się do zmiany kolejnosci wykonywania rozkazów przez kompilator. Nic nie dodaje do treści programu.},%
		isTrue4={Nie},%
		answer4={Może wymagać wyczyszczenia potoku i rozpoczęcia nowej (...)},%
		explain4={Nie wiem jak hazard danych może czegokolwiek wymagać skoro jest zjawiskiem ubocznym i je eliminujemy. Sprzętowa i programowa eliminacja hazardu jedynie może doprowadzić do \textbf{wstrzymania} napełniania potoku.},%
	}
\end{minipage}
\begin{minipage}{\textwidth}
	\question{%
		question={Jak można ominąć hazard danych:}%
	}{%
		isTrue1={Nie},%
		answer1={Poprzez rozgałęzienia},%
		explain1={Nie, rozgałęzienie to po prostu instrukcje typu IF, które tworzą takie rozgałęzienia. Mechanizm przewidywania rozgałęzień jest stosowany do usuwania hazardu sterowania związanego ze skokami i rozgałęzieniami.},%
		isTrue2={Nie},%
		answer2={Poprzez uproszczenie adresowania - adresowanie bezpośrednie.},%
		explain2={Bullshit. Nie wiem w czym miało by pomóc uproszczenia adresowania, poza pójściem w stronę RISCu, ale na hazard to nie pomoże. Tym można tylko skrócić czas odwołania się do danych.},%
		isTrue3={Tak},%
		answer3={Przez zamianę rozkazów},%
		explain3={Tak, i na tym polega mechanizm skoków opóźnionych, które mogą program zmodyfikować (dodać rozkaz NOP) albo zoptymalizować, właśnie zamieniają rozkazy kolejnością.},%
	}
\end{minipage}
\begin{minipage}{\textwidth}
	\question{%
		question={Mechanizm skoków opóźnionych:}%
	}{%
		isTrue1={Tak},%
		answer1={Polega na opóźnianiu wykonywania skoku do czasu wykonania rozkazu następnego za skokiem},%
		explain1={Tak, cały ten mechanizm sprowadza się do opóźnienia efektu skoku o jeden rozkaz. Zapewnia to, że rozkaz następny po skoku zawsze będzie wykonywany w całości.},%
		isTrue2={Nie},%
		answer2={Wymaga wstrzymania potoku na jeden takt.},%
		explain2={Nie, mechanizm potoków nie musi być wstrzymywany. Mechanizm ten zmienia postać programu w trakcie kompilacji, ale na samą realizację potoku nie ma wpływu (afaik, not sure).},%
		isTrue3={Nie},%
		answer3={Powoduje błąd na końcu pętli},%
		explain3={Pfff, jak programista ssie pałę to tak, jednak w założeniu tak się nie dzieje.},%
		isTrue4={Tak},%
		answer4={Wymaga umieszczenia rozkazu NOP za rozkazem skoku lub reorganizacje programu},%
		explain4={Tak, mechanizm sprowadza się do tego, i tylko do tego, patrz pierwsza odpowiedź.},%
	}
\end{minipage}
\begin{minipage}{\textwidth}
	\question{%
		question={Tablica historii rozgałęzień:}%
	}{%
		isTrue1={Tak},%
		answer1={Zawiera m.in. adresy rozkazów rozgałęzień},%
		explain1={Tak, tablica ta zawiera bit ważności, \emph{adres rozkazu rozgałęzienia}, bity historii oraz \emph{adres docelowy rozgałęzienia}.},%
		isTrue2={Tak},%
		answer2={Pozwala zminimalizować liczbę błędnych przewidywań rozgałęzień w zagnieżdżonej pętli},%
		explain2={Tak, z tego co wiem jest strategią dynamiczną i najbardziej optymalną ze wszystkich - skończony automat przewidywania rozgałęzień oparty na tej tablicy (z dwoma bitami historii) może być zrealizowany na dwóch bitach.},%
		isTrue3={Nie},%
		answer3={Nie może być stosowana w procesorach CISC},%
		explain3={Ten mechanizm służy zabezpieczeniu przed hazardem, który występuje w przetwarzaniu potokowym, a z tego korzystają zarówno CISC jak i RISC.},%
		isTrue4={Nie},%
		answer4={Jest obsługiwana przez jądro systemu operacyjnego},%
		explain4={Chyba nie, ten mechanizm znajduje się w sprzęcie procesora.},%
	}
\end{minipage}
\begin{minipage}{\textwidth}
	\question{%
		question={W tablicy historii rozgałęzień z 1 bitem historii można zastosować następujący algorytm przewidywania (najbardziej złożony):}%
	}{%
		isTrue1={Nie},%
		answer1={Skok opóźniony},%
		explain1={Nie, skoki opóźnione nie służą do przewidywania rozgałęzień, są zupełnie innym mechanizmem eliminacji hazardu.},%
		isTrue2={Nie},%
		answer2={Przewidywanie, że rozgałęzienie (skok warunkowy) zawsze nastąpi},%
		explain2={Nie, to strategia statyczna, która może być wykonywana gdy adres rozkazu rozgałęzienia NIE jest w tablicy. Nie wykorzystuje bitu historii.},%
		isTrue3={Nie},%
		answer3={Przewidywanie, że rozgałęzienie nigdy nie nastąpi},%
		explain3={Nie, to strategia statyczna, która może być wykonywana gdy adres rozkazu rozgałęzienia NIE jest w tablicy. Nie wykorzystuje bitu historii.},%
		isTrue4={Tak},%
		answer4={Przewidywanie, że kolejne wykonanie rozkazu rozgałęzienia będzie przebiegało tak samo jak poprzednie},%
		explain4={Tak, i to jest wszystko na co stać historię 1-bitową. Historia 2-bitowa umożliwia interpretację:\\- historii ostatniego wykonania skoku - tak lub nie;\\- przewidywania następnego wykonania skoku - tak lub nie\\A zamiana strategii następuje dopiero po drugim błędzie przewidywania.},%
		isTrue5={Nie},%
		answer5={Wstrzymanie napełniania potoku}, %
		explain5={Nie, wstrzymywanie potoku mogą spowodować algorytmy zajmujące się eliminacją hazardu danych - zależnosci między argumentami.}, %
	}
\end{minipage}
\begin{minipage}{\textwidth}
	\question{%
		question={Problemy z potokowym wykonywaniem rozkazów skoków (rozgałęzień) mogą być wyeliminowane lub ograniczone przy pomocy}:%
	}{%
		isTrue1={Nie},%
		answer1={Zapewnienia spójności pamięci podręcznej},%
		explain1={Nie, to problem komputerów wieloprocesorowych.},%
		isTrue2={Tak},%
		answer2={Tablicy historii rozgałęzień},%
		explain2={Tak, to najprawdopodobniej najlepszy służący ku temu mechanizm. Stara się przewidywać czy skok będzie wykonany bądź nie, wykorzystuje do tego kilka strategii.},%
		isTrue3={Nie},%
		answer3={Techniki wyprzedzającego pobrania argumentu},%
		explain3={Nie, ten mechanizm służy do eliminacji hazardu danych - zależności między argumentami.},%
		isTrue4={Tak},%
		answer4={Wystawienia do programu rozkazów typu „nic nie rób”},%
		explain4={Tak, tym rozkazem jest \emph{NOP} i jest wstawiany przez mechanizm skoków opóźnionych, który służy do zabezpieczania potoku.},%
		isTrue5={Nie},%
		answer5={Protokołu MESI}, %
		explain5={Nie, on jest od zapewnienia spójności pamięci wspólnej czy jakoś tak.}, %
		isTrue6={Tak}, %
		answer6={Wykorzystania techniki skoków opóźniających}, %
		explain6={Tak, umożliwiają ona modyfikację programu (wstawienie rozkazu NOP), albo jego optymalizację (zamiana kolejności wykonywania rozkazów.) Mechanizm ten opóźnia efekt skoku o jeden rozkaz, co zapewnia, że rozkaz po skoku będzie w całości wykonany.}, %
		isTrue7={Nie}, %
		answer7={Technologii MMX}, %
		explain7={Polega zupełnie na czym innym.}, %
	}
\end{minipage}
\begin{minipage}{\textwidth}
	\question{%
		question={Konsekwencją błędu przy przewidywaniu rozgałęzień może być:}%
	}{%
		isTrue1={Nie},%
		answer1={Wstrzymanie realizowanego wątku i przejście do realizacji innego wątku},%
		explain1={Przewidywanie rozgałęzień odbywa się lokalnie, osobno dla każdego wątku.},%
		isTrue2={Tak},%
		answer2={Konieczność wyczyszczenia kolejki rozkazów do potoku},%
		explain2={W przypadku jeżeli mechanizm przewidywania rozgałęzienia się pomyli i zacznie pobierać rozkazy z błędnego rozgałęzienia, potok rozkazów musi zostać wyczyszczony. Czyli np. w IFie miało być \emph{true}, a okazało się, że \emph{false}, to należy zbędne pobrane rozkazy wyczyścić.},%
		isTrue3={Nie},%
		answer3={Konieczność wyczyszczenia tablicy historii rozgałęzień.},%
		explain3={W przypadku błędu należy tablicę aktualizować, aby w przyszłości to rozgałęzienie było przewidywane z większą dokładnością.},%
		isTrue4={Nie},%
		answer4={Przerwanie realizowanego procesu / wątku i sygnalizacja wyjątku},%
		explain4={Nie, należy jedynie zmienić strategię / przejść do innej gałęzi, a nie usuwać proces. Błąd przewidywania rozgałęzień nie jest czymś na tyle złym, by przerywać program. To naturalna konsekwencja potokowości, z którą należy się uporać.},%
		isTrue5={Nie},%
		answer5={Konieczność przemianowania rejestrów w procesorach}, %
		explain5={Przemianowanie rejestrów występuje by procesorach skalarnych w celu równoległej realizacji zadań na potokach.}, %
	}
\end{minipage}
\begin{minipage}{\textwidth}
	\question{%
		question={W procesorach superskalarnych:}%
	}{%
		isTrue1={Tak},%
		answer1={Liczba rozkazów, które procesor może wykonać w 1 takcie zależy od liczby jednostek potokowych w procesorze},%
		explain1={Procesory superskalarne posiadają wiele jednostek potokowych, które są konieczne by móc wykonywać wiele rozkazów w jednym takcie. Od ich liczby zależy owa liczba rozkazów.},%
		isTrue2={Nie},%
		answer2={Liczba rozkazów, które procesor może wykonać w jednym takcie, zależy od liczby stopni potoku.},%
		explain2={Nie, liczba stopni potoku mówi, na ile części dzieli się dany rozkaz w tej jednostce potokowej. One umożliwiają wykonanie wielu rozkazów w jednej jednostce czasu, jednak nie przekłada się to bezpośrednio na liczbę rozkazów, ze względu na zawikłania czasowe, oraz nie jest to idea procesora superskalarnego.},%
		isTrue3={Nie},%
		answer3={Liczba rozkazów pobieranych z pamięci, w każdym takcie musi przekraczać liczbę jednostek potokowych},%
		explain3={Liczba pobranych rozkazów powinna być co najmniej równa ilości jednostek potokowych.},%
		isTrue4={Tak},%
		answer4={Liczba rozkazów, które procesor może wykonać w taktach zależy od liczby jednostek potokowych w procesorze},%
		explain4={Tak, patrz pierwsza odpowiedź.},%
	}
\end{minipage}
\begin{minipage}{\textwidth}
	\question{%
		question={Architektura superskalarna:}%
	}{%
		isTrue1={Nie},%
		answer1={Dotyczy systemów SMP},%
		explain1={Zdecydowanie nie tylko. Architektura superskalarna wymaga mechanizmu potokowego, czyli dotyczy głównie architektury RISC.},%
		isTrue2={Nie},%
		answer2={Wymaga zastosowania protokołu MESI},%
		explain2={Nie, architektura superskalarna wymaga jedynie zastosowania co najmniej dwóch jednostek potokowych.},%
		isTrue3={Tak},%
		answer3={Umożliwia równoległe wykonywanie kilku rozkazów w jednym procesorze},%
		explain3={Tak, i taki jest cel jej istnienia. Umożliwia to mechanizm potokowy.},%
		isTrue4={Nie},%
		answer4={Wywodzi się z architektury VLIW},%
		explain4={Wręcz odwrotnie, to VLIW wykorzystuje architekturę superskalarną na której opiera swój podział rozkazów na paczki.},%
		isTrue5={Tak},%
		answer5={Wykorzystuje wiele potokowych jednostek funkcjonalnych},%
		explain5={Tak, wymagane są co najmniej dwie jednostki potokowe, ponieważ architektura superskalarna w założeniu może wykonywać wiele rozkazów w jednym takcie.},%
		isTrue6={Nie},%
		answer6={Nie dopuszcza do wystąpienia hazardu sterowania},%
		explain6={Hazard sterowania jest problemem jednostek potokowych i muszę one być rozwiązane przez skoki opóźnione lub tablicę historii rozgałęzień. Sama architektura superskalarna nie gwarantuje rozwiązania tego problemu.},%
		isTrue7={Tak},%
		answer7={Umożliwia wykonanie wielu rozkazów w jednym takcie},%
		explain7={Jest to idea architektury superskalarnej.},%
		isTrue8={Nie},%
		answer8={Wykorzystuje model obliczeń pamięć - pamięć},%
		explain8={Ta architektura nie jest ograniczona do jednego modelu obliczeń.},%
		isTrue9={Nie},%
		answer9={Jest stosowana tylko w procesorach wielordzeniowych},%
		explain9={Architektura superskalarna to nie wielordzeniowość! Jednostka superskalarna wykonuje wiele instrukcji jednocześnie, ale w pojedynczym wątku. Z kolei procesor wielordzeniowy potrafi wykonywać wiele wątków, czyli wiele ciągów instrukcji jednocześnie. Jednostka superskalarna może być rdzeniem, ale ofc nie tylko w tym typie procków jest stosowana.},%
	}
\end{minipage}
\begin{minipage}{\textwidth}
	\question{%
		question={W procesorach superskalarnych:}%
	}{%
		isTrue1={Tak},%
		answer1={Jest możliwe równoległe wykonywanie kilku rozkazów w jednym procesorze (rdzeniu)},%
		explain1={Tak, własnie taka jest idea stworzenia procesorów superskalaranych, by móc w jednym takcie wykonać $>1$ liczby instrukcji. Zapewnia to niepojedyncza liczba jednostek potokowych.},%
		isTrue2={Tak},%
		answer2={Rozszerzenia architektury wykorzystujące model SIMD umożliwiają wykonanie rozkazów wektorowych},%
		explain2={},%
		isTrue3={Nie},%
		answer3={Nie występuje prawdziwa zależność danych},%
		explain3={Niestety występuje, i prawdę mówiąc, występuje tutaj każdy rodzaj zależności między rozkazami: prawdziwa zależność danych, zależność wyjściowa oraz antyzależność.},%
		isTrue4={Tak},%
		answer4={Mogą wystąpić nowe formy hazardu danych: zależności wyjściowe między rozkazami oraz antyzależności},%
		explain4={Tak, patrz wyżej.},%
	}
\end{minipage}
\begin{minipage}{\textwidth}
	\question{%
		question={Przetwarzanie wielowątkowe:}%
	}{%
		isTrue1={Tak},%
		answer1={Zapewnia lepsze wykorzystanie potoków},%
		explain1={Tak, ma na celu minimalizację strat cykli w trakcie realizacji wątku, jakie mogą powstać na wskutek:\\ - chybionych odwołań do pamięci podręcznej;\\- błędów w przewidywaniu rozgałęzień;\\- zależności między argumentami},%
		isTrue2={Tak},%
		answer2={Minimalizuje straty wynikające z chybionych odwołań do pamięci podręcznej},%
		explain2={Tak, patrz wyżej.},%
		isTrue3={Tak},%
		answer3={Wymaga zwielokrotnienia zasobów procesora (rejestry, liczniki rozkazów, itp.)},%
		explain3={Niestety tak, jest to warunek sprzętowej realizacji wielowątkowości.},%
		isTrue4={Nie},%
		answer4={Nie może być stosowane w przypadku hazardu danych},%
		explain4={Nie, hazard danych wynika z zależności między argumentami, które są naturalnym ryzykiem przy stosowaniu mechanizmu potokowego. Nie powinny być blokowane z tego powodu, tym bardziej, że wielowątkowość ma dodatkowo chronić liczbę cykli przed zgubnym wpływem hazardu.}, %
	}
\end{minipage}
\begin{minipage}{\textwidth}
	\question{%
		question={Pojęcie równoległości na poziomie rozkazów:}%
	}{%
		isTrue1={Nie},%
		answer1={Dotyczy architektury MIMD},%
		explain1={Nie, ten rodzaj równoległości dotyczy mechanizmów potokowych (CISC i RISC), architektury superskalarnej oraz VLIW.},%
		isTrue2={Tak},%
		answer2={Odnosi się m.in. do przetwarzania potokowego},%
		explain2={Tak, ideą mechanizmu potoków jest zrównoleglenie rozkazów i możliwość wykonywania wielu z nich w tej samej chwili czasu.},%
		isTrue3={Nie},%
		answer3={Dotyczy architektury MPP},%
		explain3={Nie, patrz wyżej.},%
		isTrue4={Tak},%
		answer4={Dotyczy m.in. architektury superskalarnej},%
		explain4={Tak, patrz wyżej.},%
	}
\end{minipage}
\begin{minipage}{\textwidth}
	\question{%
		question={Efektywne wykorzystanie równoległości na poziomie danych umożliwiają:}%
	}{%
		isTrue1={Tak},%
		answer1={Komputery wektorowe},%
		explain1={},%
		isTrue2={Tak},%
		answer2={Komputery macierzowe},%
		explain2={},%
		isTrue3={Tak},%
		answer3={Klastry},%
		explain3={},%
		isTrue4={Tak},%
		answer4={Procesory graficzne},%
		explain4={},%
		isTrue5={Tak},%
		answer5={Rozszerzenia SIMD procesorów superskalarnych}, %
		explain5={\\Ogółem zastosowanie tej równoległości jest możliwe gdy mamy do czynienia z wieloma danymi, które mogą być przetwarzane w tym samym czasie. A grafika, wektory, macierze itp. do takich należą.}, %
	}
\end{minipage}
\begin{minipage}{\textwidth}
	\question{%
		question={Wielowątkowość współbieżna w procesorze wielopotokowym zapewnia:}%
	}{%
		isTrue1={Tak},%
		answer1={Możliwość wprowadzenia rozkazów różnych wątków do wielu potoków},%
		explain1={Tak, jest to charakterystyczna cecha tego typu wielowątkowości. Z kolei wielowątkowości grubo- i drobnoziarniste umożliwiają wprowadzenie do wielu potoków \emph{wyłącznie} jednego wątku (w jednym takcie!)},%
		isTrue2={Tak},%
		answer2={Realizację każdego z wątków do momentu wstrzymania któregoś rozkazu z danego wątku},%
		explain2={Tak, wątek jest realizowany do momentu wstrzymania rozkazu. Tę samą cechę posiada wielowątkowość gruboziarnista. Z kolei wielowątkowość drobnoziarnista w kolejnych taktach realizuje naprzemiennie rozkazy kolejnych wątków.},%
		isTrue3={Nie},%
		answer3={Przełączanie wątków co takt},%
		explain3={Nie, to umożliwia tylko wielowątkowość drobnoziarnista.},%
		isTrue4={Nie},%
		answer4={Automatyczne przemianowanie rejestrów},%
		explain4={Głowy nie dam, ale chyba żadna wielowątkowość nie zapewnia automatycznego przemianowania.},%
	}
\end{minipage}
\begin{minipage}{\textwidth}
	\question{%
		question={Metoda przemianowania rejestrów jest stosowana w celu eliminacji:}%
	}{%
		isTrue1={Nie},%
		answer1={Błędnego przewidywania rozgałęzień},%
		explain1={Nie, do tego służy m.in. tablica historii rozgałęzień.},%
		isTrue2={Nie},%
		answer2={Chybionego odwołania do pamięci podręcznej},%
		explain2={Nie, to jest problem architektury VLIW i eliminuje się do przez przesunięcie rozkazów LOAD jak najwyżej, tak aby zminimalizować czas ewentualnego oczekiwania},%
		isTrue3={Nie},%
		answer3={Prawdziwej zależności danych},%
		explain3={Nie, od tego jest metoda wyprzedzającego pobierania argumentu.},%
		isTrue4={Tak},%
		answer4={Zależności wyjściowej między rozkazami.},%
		explain4={Tak, ta metoda eliminuje powyższy i poniższy problem. Polega na dynamicznym przypisywaniu rejestrów do rozkazów.},%
		isTrue5={Tak},%
		answer5={Antyzależności między rozkazami},%
		explain5={Patrz wyżej.},%
	}
\end{minipage}
\begin{minipage}{\textwidth}
	\question{%
		question={Wyprzedzające pobranie argumentu pozwala rozwiązać konflikt wynikający z:}%
	}{%
		isTrue1={Nie},%
		answer1={Zależności wyjściowej miedzy rozkazami},%
		explain1={Tę zależność musi kontrolować układ sterujący.},%
		isTrue2={Tak},%
		answer2={Prawdziwej zależności danych},%
		explain2={Tak, do tego służy, patrz: prawdziwa zależność danych \ref{subsubsec:RAW}},%
		isTrue3={Nie},%
		answer3={Błędnego przewidywania rozgałęzień},%
		explain3={Nie powoduje konfliktów, należy je tylko obsłużyć i ograniczyć liczbę występowań.},%
		isTrue4={Tak},%
		answer4={Antyzależności miedzy rozkazami},%
		explain4={Rozkazowi, który odczytuje jakąś zmienną, od razu jest podrzucana jej zmieniona wartość. zamiast przechodzić przez rejestr pośredni (zapis i ponowny odczyt).},%
	}
\end{minipage}
\begin{minipage}{\textwidth}
	\question{%
		question={Przepustowość (moc obliczeniowa) dużych komputerów jest podawana w:}%
	}{%
		isTrue1={Tak},%
		answer1={GFLOPS},%
		explain1={},%
		isTrue2={Nie},%
		answer2={Liczbie instrukcji wykonywanych na sekundę},%
		explain2={},%
		isTrue3={Tak},%
		answer3={Liczbie operacji zmiennoprzecinkowych na sekundę},%
		explain3={},%
		isTrue4={Nie},%
		answer4={Mb/sek},%
		explain4={To jest do zapamiętania na prostu - takie są standardy},%
	}
\end{minipage}
\begin{minipage}{\textwidth}
	\question{%
		question={Podstawą klasyfikacji Flynna jest:}%
	}{%
		isTrue1={Nie},%
		answer1={Liczba jednostek przetwarzających i sterujących w systemach komputerowych},%
		explain1={},%
		isTrue2={Nie},%
		answer2={Protokół dostępu do pamięci operacyjnej},%
		explain2={},%
		isTrue3={Nie},%
		answer3={Liczba modułów pamięci operacyjnej w systemach komputerowych},%
		explain3={},%
		isTrue4={Tak},%
		answer4={Liczba strumieni rozkazów i danych w systemach komputerowych},%
		explain4={To po prostu należy zapamiętać. \textbf{Kryterium klasyfikacji Flynna jest \emph{liczba strumieni rozkazów} oraz \emph{liczba strumieni danych} w systemie komputerowym. NIC WIĘCEJ, NIC MNIEJ.\\Albo inaczej: \emph{$Liczba\_strumieni\times(rozkazow+danych)$}}},%
	}
\end{minipage}
\begin{minipage}{\textwidth}
	\question{%
		question={Model SIMD:}%
	}{%
		isTrue1={Nie},%
		answer1={Był wykorzystywany tylko w procesorach macierzowych},%
		explain1={Nie, o niego oparte są również m.in. procesory wektorowe, GPU, technologie MMX oraz SSE. Nie, był również wykorzystywany w komputerach wektorowych, rozszerzeniach SIMD oraz GPU.},%
		isTrue2={Tak},%
		answer2={Jest wykorzystywany w multimedialnych rozszerzeniach współczesnych procesorów},%
		explain2={Tak, multimedialne rozszerzenie służą do działania na wielu rejestrach jednocześnie, co umożliwiają m.in. technologie MMX i SSE.},%
		isTrue3={Tak},%
		answer3={Jest wykorzystywany w heterogenicznej architekturze PowerXCell},%
		explain3={Być może XD},%
		isTrue4={Tak},%
		answer4={Zapewnia wykonanie tej samej operacji na wektorach argumentów},%
		explain4={Znaczenie: \emph{Single Instruction Multiple Device}. Wykonuje jedną instrukcję na wielu urządzeniach.},%
		isTrue5={Tak},%
		answer5={Jest podstawą rozkazów wektorowych},%
		explain5={Pod ten model są zaprojektowane rozkazy wektorowe - jedno instrukcja, wiele urządzeń.},%
		isTrue6={Nie},%
		answer6={Jest podstawą architektury procesorów superskalarnych},%
		explain6={Nie, jest wykorzystywany w rozszerzeniach wektorowych, ale nie jest podstawą.},%
	}
\end{minipage}
\begin{minipage}{\textwidth}
	\question{%
		question={Komputery wektorowe:}%
	}{%
		isTrue1={Nie},%
		answer1={Posiadają jednostki potokowe o budowie wektorowej},%
		explain1={Nie, posiadają potokowe jednostki arytmetyczne, które nie są wektorowe.},%
		isTrue2={Tak},%
		answer2={Posiadają w liście rozkazów m.in. rozkazy operujące na wektorach danych},%
		explain2={Jak najbardziej, nie mogłyby się bez tego obejść.},%
		isTrue3={Tak},%
		answer3={Wykorzystują od kilku do kilkunastu potokowych jednostek arytmetycznych},%
		explain3={Tak, tych jednostek może być wiele, można to zauważyć na przykładzie komputera Cray-1 (wykład 7-8, slajd 31)},%
		isTrue4={Nie},%
		answer4={Posiadają listę rozkazów operujących wyłącznie na wektorach},%
		explain4={Zdecydowanie nie. Owszem, te komputery posiadają rejestry wektorowe i wektorowe jednostki zmiennoprzecinkowe, ale nie jest to wszystko. Mają również normalne rejestry, adresację, jednostki skalarne i możliwość wykonywania na nich operacji.},%
	}
\end{minipage}
\begin{minipage}{\textwidth}
	\question{%
		question={Moc obliczeniowa komputerów wektorowych:}%
	}{%
		isTrue1={Nie},%
		answer1={Zależy od liczby stopni potoku.},%
		explain1={Moc obliczeniowa nie jest zależna od liczby stopni potoku. Ta jedynie wpływa na ilosć rozkazów jakie mogą być wykonane w chwili czasu w jednostce potokowej.},%
		isTrue2={Tak},%
		answer2={Jest odwrotnie proporcjonalna do długości taktu zegarowego},%
		explain2={Tak, obliczamy ją wzorem $Przep=lim_{n\to\infty}\frac{n}{t_{start}+(n-1)\times\tau}=\frac{1}{\tau}$},%
		isTrue3={Nie},%
		answer3={Jest wprost proporcjonalna do długości taktu zegarowego},%
		explain3={Nie, patrz wyżej.},%
		isTrue4={Nie},%
		answer4={Zależy odwrotnie proporcjonalnie od liczby jednostek potokowych połączonych łańcuchowo.},%
		explain4={Nie, idea operacji wektorowej na komputerze wektorowym zakłada jedną jednostkę potokową. Ich zwiększenie nie powinno wpłynąć bezpośrednio na moc.},%
		isTrue5={Tak},%
		answer5={Zmierza asymptotycznie do wartości maksymalnej wraz ze wzrostem długości wektora}, %
		explain5={Tak, istnieje pewna wartość maksymalna do której moc dąży logarytmicznie wraz ze wzrostem długości wektora.}, %
		isTrue6={Nie}, %
		answer6={Nie zależy od długości wektora}, %
		explain6={Bzdura, patrz wyżej.}, %
		isTrue7={Nie}, %
		answer7={Zależy liniowo od długości wektora}, %
		explain7={Bzdura, patrz wyżej.}, %
	}
\end{minipage}
\begin{minipage}{\textwidth}
	\question{%
		question={Procesory wektorowe:}%
	}{%
		isTrue1={Tak},%
		answer1={Mogą być stosowane w systemach wieloprocesorowych},%
		explain1={Domyślnie procesory wektorowe mogą pracować pojedynczo, ale mogą być częścią takiego systemu. Poza tym nie znalazłem nic, co by temu przeczyło. Jest też np. CUDA - architektura wielordzeniowych procesorów graficznych. Sama architektura SIMD działa na wielu procesorach.},%
		isTrue2={Nie},%
		answer2={Mają listę rozkazów operującą jedynie na wektorach},%
		explain2={Nie, posiadają też m.in. potokowe jednostki arytmetyczne oraz jednostki skalarne, do operowania na zwykłych liczbach.},%
		isTrue3={Tak},%
		answer3={Mają moc kilka razy większą od procesorów skalarnych},%
		explain3={Tak, przyspieszenie jest ilorazem czasu wykonywania na procesorze niewektorowym do czasu wykonywania na procesorze wektorowym. Np. dla rozkazu dodawania \emph{n} wektorów przyspieszenie wyliczane jest wg wzoru $ a=\frac{15\tau n}{t_{start}+(n-1)\tau} $, gdzie przy \emph{n} dążącym do nieskończoności \emph{a} jest równe 15.},%
	}
\end{minipage}
\begin{minipage}{\textwidth}
	\question{%
		question={Komputery macierzowe:}%
	}{%
		isTrue1={Tak},%
		answer1={Mają w liście rozkazów m.in. rozkazy operujące na wektorach danych},%
		explain1={Tak, te komputery są rozwinięciem komputerów wektorowych i muszą mieć rozkazy wektorowe. Komputery macierzowe posiadają po \emph{n} jednostek przetwarzających, które potrafią razem obliczyć \emph{n} składowych wektora.},%
		isTrue2={Nie},%
		answer2={Mają macierzowe potokowe układy arytmetyczne},%
		explain2={Nie, posiadają natomiast jednostki przetwarzające. Z kolei potokową jednostkę arytmetyczną posiadają komputery wektorowe.},%
		isTrue3={Nie},%
		answer3={Mają w typowych rozwiązaniach zestaw pełnych procesów połączonych siecią połączeń},%
		explain3={Nie, w typowym rozwiązaniu jest jeden pełny procesor z wieloma jednostkami potokowymi, które są połączone siecią łączącą (statyczną lub dynamiczną). Sieć połączeń pełnych procków posiadają superkomputery z top500 (Nie jestem pewien tej odpowiedzi).},%
		isTrue4={Tak},%
		answer4={Wykonują synchroniczną operację wektorową w sieci elementów przetwarzających},%
		explain4={Tak właśnie działają.},%
	}
\end{minipage}
\begin{minipage}{\textwidth}
	\question{%
		question={Rozkazy wektorowe mogą być realizowane przy wykorzystaniu:}%
	}{%
		isTrue1={Tak},%
		answer1={Macierzy elementów przetwarzających},%
		explain1={Tak, komputery macierzowe operują na rozkazach wektorowych.},%
		isTrue2={Nie},%
		answer2={Zestawu procesorów superskalarnych},%
		explain2={Procesory superskalarne w założeniu nie posiadają rozkazów wektorowych.},%
		isTrue3={Tak},%
		answer3={Technologii MMX},%
		explain3={Tak, jest to pochodna technologia modelu SIMD, wykonuje operacje na krótkich wektorach (64-bit)},%
		isTrue4={Nie},%
		answer4={Sieci połączeń typu krata},%
		explain4={Jest to sieć połączeń, która łączy jednostki przetwarzające w komputerze macierzowym. Raczej na wektorach na częsć komputera nie działa.},%
		isTrue5={Tak},%
		answer5={Potokowych jednostek arytmetycznych}, %
		explain5={Tak, takie znajdują się w komputerach wektorowych.}, %
	}
\end{minipage}
\begin{minipage}{\textwidth}
	\question{%
		question={Rozkazy wektorowe:}%
	}{%
		isTrue1={Nie},%
		answer1={Nie mogą być wykonywane bez użycia potokowych jednostek arytmetycznych},%
		explain1={Mogą. Komputery macierzowe ich nie posiadają i wykonują rozkazy wektorowe sprawnie.},%
		isTrue2={Tak},%
		answer2={W komputerach wektorowych ich czas wykonania jest wprost proporcjonalny do długości wektora},%
		explain2={Tak, na przykładzie rozkazu dodawania wektorów widać, że czas rośnie równomiernie wraz z ilością elementów wektora.\\$t_{w}=t_{start}+(n-1)\times\tau$},%
		isTrue3={Tak},%
		answer3={Są charakterystyczne dla architektury SIMD},%
		explain3={Tak, z niej się zrodziły, tak samo jak m.in. technologie MMX i SSE.},%
		isTrue4={Nie},%
		answer4={Są rozkazami dwuargumentowymi i w wyniku zawsze dają wektor},%
		explain4={Nie, mogą operować na 1 argumencie na przykład. Rozkaz może być też 3 argumentowy, jak rozkaz dodawania VADD. Pierwszym argumentem jest rejestr docelowy, zawartość pozostałych dwóch jest dodana.},%
	}
\end{minipage}
\begin{minipage}{\textwidth}
	\question{%
		question={Architektura CUDA:}%
	}{%
		isTrue1={Tak},%
		answer1={Umożliwia bardzo wydajne wykonywanie operacji graficznych},%
		explain1={Tak, ta architektura jest rozwinięciem mechanizmów wektorowych oraz macierzowych i jest przeznaczona specjalnie dla przetwarzania grafiki.},%
		isTrue2={Tak},%
		answer2={Stanowi uniwersalną architekturę obliczeniowa połączoną z równoległym modelem programistycznym},%
		explain2={Tak, pomimo specjalizacji graficznej, architektura ta jest uniwersalna i zdolna do wszystkiego. Procesory posiadają uniwersalne programy obliczeniowe, a CUDA posiada model programistyczny (oraz podział programu na 5 faz). Składa się on z:\\- Kompilatora NVCC;\\- Podział programu na kod wykonywany przez procesor (host code) oraz kartę graficzną (kernel);\\- Realizacja obliczeń równoległych wg modelu SIMT (Single Instruction Multiple Threading)},%
		isTrue3={Tak},%
		answer3={Realizuje model obliczeniowy SIMT},%
		explain3={Tak, patrz wyżej. Działanie: wiele niezależnych wątków wykonuje tę samą operację. Architektura posiada również mechanizm synchronizacji wątków (\emph{barrir synchronization}) dla komunikacji oraz współdzielona pamięć.},%
		isTrue4={Nie},%
		answer4={Jest podstawą budowy samodzielnych, bardzo wydajnych komputerów},%
		explain4={Komputery CUDA nie są ogólnego zastosowania, tylko do ogólnych problemów numerycznych. Na pewno nie są podstawą, bo np. komputer ... (dokończyć by trza)},%
	}
\end{minipage}
\begin{minipage}{\textwidth}
	\question{%
		question={Systemy SMP:}%
	}{%
		isTrue1={Nie},%
		answer1={Wykorzystują protokół MESI do sterowania dostępem do wspólnej magistrali},%
		explain1={Ten protokół wykorzystują systemy \textbf{UMA} (podkategoria systemów SMP) ze wspólną magistralą w celu zapewnienia spójności pamięci podręcznych (\emph{snooping}). Mogą też używać katalogów, ale podkategoria \textbf{NUMA} wykorzystuje wyłącznie katalogi.},%
		isTrue2={Nie},%
		answer2={Posiadają skalowalne procesory},%
		explain2={SMP należy do systemów wieloprocesorowych, ale te nie muszą być skalowalne.},%
		isTrue3={Nie},%
		answer3={Posiadają pamięć fizycznie rozproszoną, ale logicznie wspólną},%
		explain3={Nie, pamięć jest fizycznie wspólna. Fizycznie rozproszoną pamięć posiadają systemy MPP.},%
	}
\end{minipage}
\begin{minipage}{\textwidth}
	\question{%
		question={Protokół MESI:}%
	}{%
		isTrue1={Nie},%
		answer1={Jest wykorzystywany do sterowania dostępem do magistrali w systemie SMP},%
		explain1={Protokół MESI wykorzystywany jest do zapewniania spójności pamięci podręcznych \emph{cache} w architekturze SMP (\emph{snooping}), a dokładniej w UMA. NUMA korzysta tylko z katalogów.},%
		isTrue2={Tak},%
		answer2={Zapewnia spójność pamięci cache w systemie SMP},%
		explain2={Do tego właśnie służy.},%
		isTrue3={Nie},%
		answer3={Służy do wymiany komunikatów w systemie MPP},%
		explain3={Patrz wyżej.},%
		isTrue4={Nie},%
		answer4={Chroni przed hazardem w procesorach superskalarnych},%
		explain4={Patrz wyżej.},%
	}
\end{minipage}
\begin{minipage}{\textwidth}
	\question{%
		question={W architekturze NUMA}%
	}{%
		isTrue1={Tak},%
		answer1={Dane są wymieniane między węzłami w postaci linii pamięci podręcznej (PaP)},%
		explain1={Tak, każdy procesor / węzeł posiada swoją własną szybką pamięć podręczną. Pamięć ta jest publiczna - inne procesory mają do niej dostęp, ale wymiana informacji na linii \emph{moja pamięć - inny procesor} jest znacznie wolniejsza niż \emph{procesor - jego pamięć}.},%
		isTrue2={Nie},%
		answer2={Spójność PaP węzłów jest utrzymywana za pomocą protokołu MESI},%
		explain2={Protokół MESI jest wykorzystywany w architekturze \textbf{UMA} do \emph{snoopingu} - zapewnienia spójności pamięci podręcznych procków.},%
		isTrue3={Nie},%
		answer3={Czas dostępu do pamięci lokalnej w węźle jest podobny do czasu dostępu do pamięci nielokalnej},%
		explain3={Odwołanie do nielokalnej pamięci są znacznie wolniejsze niż do lokalnej, ok. 10-krotnie bardziej. Dotyczy to głównie architektury NC-NUMA, patrz: rodzaje systemów NUMA \ref{subsec:rodzajeNUMA}},%
		isTrue4={Tak},%
		answer4={Czas zapisu danych do pamięci nielokalnej może być znacznie dłuższy od czasu odczytu z tej pamięci},%
		explain4={Patrz wyżej.},%
		isTrue5={Tak},%
		answer5={Każdy procesor ma dostęp do pamięci operacyjnej każdego węzła}, %
		explain5={Patrz wyżej.}, %
		isTrue6={Nie}, %
		answer6={Procesy komunikują się poprzez przesył komunikatów}, %
		explain6={Przesył komunikatów występuje w systemach MPP, gdzie pamięć jest rozproszona fizycznie i logicznie. W NUMA jest fizycznie rozproszona między węzłami (do przesyłu informacji wykorzystywana jest sieć łączącą węzły), ale stanowi logicznie jedną całość.}, %
		isTrue7={Tak}, %
		answer7={Pamięć operacyjna jest rozproszona fizycznie pomiędzy węzłami, ale wspólna logicznie}, %
		explain7={Patrz wyżej.}, %
	}
\end{minipage}
\begin{minipage}{\textwidth}
	\question{%
		question={W architekturze CC-NUMA}%
	}{%
		isTrue1={Tak},%
		answer1={Każdy procesor ma dostęp do pamięci operacyjnej każdego węzła},%
		explain1={Tak, ponieważ architektura NUMA opiera się o niejednorodny dostęp do pamięci - każdy procesor ma pełny dostęp do pamięci lokalnej oraz nielokalnej, czyli pamięci podręcznych wszystkich innych procesorów.},%
		isTrue2={Nie},%
		answer2={Spójność pamięci pomiędzy węzłami jest utrzymywana za pomocą protokołu MESI},%
		explain2={Nie, nie jest potrzebna spójność pamięci, ponieważ każdy procesor odczytuje potrzebne mu zmienne itp. pośrednio przez katalog.},%
		isTrue3={Tak},%
		answer3={Dane są wymieniane między węzłami w postaci linii pamięci podręcznej},%
		explain3={Każda linia posiada pewną liczbę bajtów, które inne procesory mogą pobierać. Możliwe, że i katalogi korzystają z wymiany danych poprzez linie.},%
		isTrue4={Tak},%
		answer4={Pamięć operacyjna jest fizycznie rozproszona pomiędzy węzłami, ale wspólna logicznie},%
		explain4={Dokładnie tak.},%
	}
\end{minipage}
\begin{minipage}{\textwidth}
	\question{%
		question={W systemach wieloprocesorowych o architekturze CC-NUMA:}%
	}{%
		isTrue1={Tak},%
		answer1={Spójność pamięci wszystkich węzłów jest utrzymywana za pomocą katalogu},%
		explain1={Tak, w NUMA do zachowania spójności danych można stosować wyłącznie katalogi (a np. protokołu MESI nie).},%
		isTrue2={Tak},%
		answer2={Pamięć operacyjna jest rozproszona fizycznie pomiędzy węzłami, ale wspólna logicznie},%
		explain2={Dokładnie tak.},%
		isTrue3={Nie},%
		answer3={Każdy procesor ma bezpośredni dostęp do pamięci operacyjnej każdego węzła},%
		explain3={Nie, dostęp jest pośredni. Pamięć w CC-NUMA jest fizycznie rozproszona, więc coś musi pośredniczyć w tej wymianie danych. Służy do tego mechanizm katalogów i węzłów. Procesor zgłasza zapotrzebowanie na linię pamięci do katalogu i ją później otrzymuje.},%
		isTrue4={Tak},%
		answer4={Dane są wymieniane między węzłami w postaci linii pamięci podręcznej},%
		explain4={Tak, ponieważ w architekturze NUMA wymiana rekordów pamięci następuje z użyciem całej linii. Np. jeśli procek chce pobrać jednego floata 4-bajtowego, a linia ma 16 bajtów, to musimy pobrać całą linię (ale i tak odbywa się to bardzo szybko).},%
	}
\end{minipage}
\begin{minipage}{\textwidth}
	\question{%
		question={W architekturze CC-NUMA czas dostępu do pamięci operacyjnej może zależeć od:}%
	}{%
		isTrue1={Tak},%
		answer1={Rodzaju dostępu (odczyt - zapis)},%
		explain1={Zapis jest znacznie wolniejszy, bo wymaga aktualizacji całej spójnej logicznie pamięci.},
		isTrue2={Nie},%
		answer2={Stanu linii (zapisanego w katalogu), do której następuje odwołanie},%
		explain2={Jak linia znajduje się już w katalogu, to następuje odczyt tylko.},%
		isTrue3={Tak},%
		answer3={Położenia komórki, do której odwołuje się rozkaz (lokalna pamięć węzła – pamięć innego węzła)},%
		explain3={Dostęp do pamięci nielokalnej (innego procka) jest znacznie dłuższy (ok. 10-krotnie)},%
		isTrue4={Nie},%
		answer4={Odległości węzłów, zaangażowanych w wykonanie rozkazu, w strukturze sieci łączącej},%
		explain4={Różnice są rzędu mikrosekund, nie jest to znaczący czas, tylko margines błędu najwyżej.},%
	}
\end{minipage}
\begin{minipage}{\textwidth}
	\question{%
		question={Katalog może być stosowany do:}%
	}{%
		isTrue1={NieWiem},%
		answer1={Utrzymania spójności pamięci podręcznych poziomu L1 i L2 w procesorach wielordzeniowych.},%
		explain1={Wiadomo, że latalog jest wykorzystywany do zachowania spójności pamięci w systemach wieloprocesorowych, UMA i NUMA. Ale czy w wielordzeniowym? A chuj wie.},%
		isTrue2={Tak},%
		answer2={Utrzymania spójności pamięci wszystkich węzłów w systemach CC-NUMA},%
		explain2={Systemy NUMA wykorzystują katalogi do zachowania spójności pamięci, i tylko katalogi.},%
		isTrue3={NieWiem},%
		answer3={Do utrzymania spójności pamięci węzłów systemów wieloprocesorowych z pamięcią rozproszoną (MPP)},%
		explain3={Jeśli chodzi o zachowanie spójności pamięci w obrębie węzła, to \textbf{tak}, bo węzłem może być system UMA lub NUMA. Ale, jeśli chodzi o zachowanie spójności pamięci między węzłami, to \textbf{nie}, bo do tego służy przesył komunikatów.},%
		isTrue4={Nie},%
		answer4={Sterowania realizacją wątków w architekturze CUDA},%
		explain4={Zdecydowanie bullshit, CUDA nie wykorzystuje katalogu. W CUDA wątkami steruje osobny procesor wątków.},%
	}
\end{minipage}
\begin{minipage}{\textwidth}
	\question{%
		question={Spójność pamięci podręcznych w procesorze wielordzeniowym może być m.in. zapewniona za pomocą:}%
	}{%
		isTrue1={Nie},%
		answer1={Przełącznicy krzyżowej},%
		explain1={Nie, to tylko jakieś rozwiązanie sieci połączeń.},%
		isTrue2={Nie},%
		answer2={Katalogu},%
		explain2={ie,to bardziej zaawansowany shit służący do komunikacji.},%
		isTrue3={Tak},%
		answer3={Protokołu MESI},%
		explain3={Tak, i tylko to do tego służy.},%
		isTrue4={Nie},%
		answer4={Wspólnej magistrali},%
		explain4={Nie, ona służy do komunikacji i synchronizacji (?) dostępu do pamięci.},%
	}
\end{minipage}
\begin{minipage}{\textwidth}
	\question{%
		question={Systemy wieloprocesorowe z pamięcią wspólną:}%
	}{%
		isTrue1={Nie},%
		answer1={Zapewniają jednorodny dostęp do pamięci},%
		explain1={Nie, bo NUMA nie zapewnia. Jednorodny dostęp występuje wtedy, gdy procesor ma dostęp wyłącznie do swojej pamięci podręcznej. Niejednorodny wtedy, gdy ma dostęp również do pamięci nielokalnej (pamięci podręcznej innych procesorów.)},%
		isTrue2={Tak},%
		answer2={Mogą wykorzystywać procesory CISC},%
		explain2={Nie ma takiego ograniczenia.},%
		isTrue3={Tak},%
		answer3={Są wykorzystywane w klastrach},%
		explain3={Tak, węzły w postaci serwerów SMP.},%
		isTrue4={Nie},%
		answer4={Wykorzystują przesył komunikatów między procesorami},%
		explain4={Nie, wykorzystują protokół MESI do \emph{snoopingu} oraz katalogi w celu zapewnienia spójności pamięci podręcznej.},%
		isTrue5={Tak},%
		answer5={Wykorzystują katalog do utrzymania spójności pamięci podręcznych}, %
		explain5={Patrz wyżej.}, %
	}
\end{minipage}
\begin{minipage}{\textwidth}
	\question{%
		question={W systemach wieloprocesorowych katalog służy do:}%
	}{%
		isTrue1={Nie},%
		answer1={Śledzenia adresów w protokole MESI},%
		explain1={Nie no kurwa, szanujmy się, katalogi powstały po to by wyprzeć protokół MESI.},%
		isTrue2={Nie},%
		answer2={Sterowania przesyłem komunikatów},%
		explain2={Nie, w UMA i NUMA nie ma przesyłu komunikatów, to jest w MPP i klastrach.},%
		isTrue3={Tak},%
		answer3={Utrzymania spójności pamięci w systemach o niejednorodnym dostępie do pamięci},%
		explain3={No tak, patrz poprzednie pytania. To jest problem z jakimi borykają się UMA i NUMA. Katalog można stosować do jego rozwiązania w obu architekturach.},%
		isTrue4={Tak},%
		answer4={Realizacji dostępu do nielokalnych pamięci w systemach NUMA},%
		explain4={Katalogi i węzły stanowią mechanizm do wymiany informacji między prockami i ich pamięciami.},%
	}
\end{minipage}
\begin{minipage}{\textwidth}
	\question{%
		question={Charakterystyczne cechy architektury MPP:}%
	}{%
		isTrue1={Nie},%
		answer1={Spójność pamięci podręcznej wszystkich węzłów},%
		explain1={Spójność wręcz nie powinna być zapewniana, każdy element są swoją własną, odrębną pamięć.},%
		isTrue2={Tak},%
		answer2={Fizycznie rozproszona PaO},%
		explain2={Tak, jest fizycznie i logicznie rozproszona.},%
		isTrue3={Nie},%
		answer3={Fizycznie rozproszona PaO, ale logicznie wspólna},%
		explain3={Nie, taka jest cecha systemów NUMA.},%
		isTrue4={Tak},%
		answer4={Przesył komunikatów między procesorami},%
		explain4={Tak, to metoda synchronizacji wykonywania zadań.},%
		isTrue5={Nie},%
		answer5={Niska skalowalność}, %
		explain5={Nie, jest przeogromna.}, %
		isTrue6={Nie}, %
		answer6={Jednorodny dostęp do pamięci wszystkich węzłów}, %
		explain6={Nie, pamięć jest rozproszona.}, %
	}
\end{minipage}
\begin{minipage}{\textwidth}
	\question{%
		question={Systemy wieloprocesorowe z pamięcią rozproszoną (MPP):}%
	}{%
		isTrue1={Tak},%
		answer1={Wyróżniają się bardzo dużą skalowalnością},%
		explain1={Posiadają ogromną skalowalność, tylko klastry mają lepszą.},%
		isTrue2={Nie},%
		answer2={Są budowane z węzłów, którymi są klastry},%
		explain2={Węzłami mogą być tylko systemy UMA i NUMA oraz zwykłe pojedyncze procesory.},%
		isTrue3={Nie},%
		answer3={Realizują synchronicznie jeden wspólny program},%
		explain3={Nie muszą być synchronicznie wykonywane (są synchronizacje, ale jeden węzeł może realizować szybciej pewne części programu)},%
		isTrue4={Nie},%
		answer4={Wymagają zapewnienia spójności pamięci podręcznych pomiędzy węzłami},%
		explain4={Wręcz nie powinny jej zapewniać, każdy węzeł pracuje osobno.},%
		isTrue5={Tak},%
		answer5={Wymianę danych i synchronizację procesów w węzłach realizują poprzez przesył komunikatów.},%
		explain5={Tak, pamięć jest logicznie rozproszona, a węzły są fizycznie oddzielne, do komunikacji wystarcza tylko przesył komunikatów.},%
		isTrue6={Nie},%
		answer6={W większości przypadków wykorzystują nietypowe, firmowe rozwiązania sieci łączących węzły systemu.},%
		explain6={Nie wiem czy w większości przypadków, ale na slajdach większość jest poświęcona typowym rozwiązaniom sieciowym (hipersześcian, krata, torus, przełącznica krzyżowa), a \emph{custom network} to tylko dodatek.},%
		isTrue7={Nie},%
		answer7={Wykorzystują katalog do utrzymania spójności pamięci węzłów systemu.},%
		explain7={Między węzłami w MPP wykorzystuje się przesył komunikatów. Katalogi sa wykorzystywane przez systemy UMA i NUMA, które mogą być węzłami w MPP. Jednak komunikacja między samymi węzłami ich nie wykorzystuje.},%
	}
\end{minipage}
\begin{minipage}{\textwidth}
	\question{%
		question={Systemy MPP są zbudowane z węzłów którymi mogą być:}%
	}{%
		isTrue1={Tak},%
		answer1={Systemy SMP},%
		explain1={Węzłami mogą być zarówno systemy UMA, jak i NUMA. Ponadto dopuszcza się zwykłe procesory z pamięcią operacyjną. Patrz: organizacja MPP \ref{subsec:organizacjaMPP}. Są to \textbf{jedyne} możliwe rodzaje węzłów.},%
		isTrue2={Nie},%
		answer2={Klastry},%
		explain2={Patrz wyżej.},%
		isTrue3={Nie},%
		answer3={Konstelacje},%
		explain3={Patrz wyżej.},%
		isTrue4={Tak},%
		answer4={Systemy NUMA},%
		explain4={Patrz wyżej.},%
		isTrue5={Tak},%
		answer5={Procesory}, %
		explain5={Patrz wyżej.}, %
	}
\end{minipage}
\begin{minipage}{\textwidth}
	\question{%
		question={Przesył komunikatów:}%
	}{%
		isTrue1={Tak},%
		answer1={Ma miejsce w systemach MPP},%
		explain1={Tak, w MPP oraz w klastrach.},%
		isTrue2={Nie},%
		answer2={W systemach MPP II-giej generacji angażuje wszystkie procesory na drodze przesyłu},%
		explain2={Nie, bo w II generacji są routery węzłów pośredniczących, które nie angażują procesorów. W I generacji wszystkie komputery były zaangażowane, ponieważ rolę routera pełnił sam procesor.},%
		isTrue3={Tak},%
		answer3={Ma miejsce w klastrach},%
		explain3={Tak, patrz wyżej.},%
	}
\end{minipage}
\begin{minipage}{\textwidth}
	\question{%
		question={Cechami wyróżniającymi klastry są:}%
	}{%
		isTrue1={Tak},%
		answer1={Niezależność programowa każdego węzła},%
		explain1={Tak, bo każdy węzeł na swój osobny system operacyjny.},%
		isTrue2={Nie},%
		answer2={Fizycznie rozproszona, ale logicznie wspólna pamięć operacyjna},%
		explain2={Jest fizycznie i logicznie rozproszona.},%
		isTrue3={Nie},%
		answer3={Nieduża skalowalność},%
		explain3={Kurwa, nie XD Klastry mają arcydupną skalowalność.},%
		isTrue4={Tak},%
		answer4={Na ogół duża niezawodność},%
		explain4={Tak, po to się je buduje i na ogół ją mają. Redundancja węzłów, mirroring dysków, kontrola funkcjonowania węzłów. Patrz: Niezawodność klastrów \ref{subsec:niezawodnoscKlastrow}},%
	}
\end{minipage}
\begin{minipage}{\textwidth}
	\question{%
		question={Klastry:}%
	}{%
		isTrue1={Nie},%
		answer1={Mają średnią skalowalność},%
		explain1={Mają największą skalowalność ze wszystkich poznanych systemów. Łatwiej go rozszerzać niż MPP, ponieważ jest jednym wielkim systemem komputerowym.},%
		isTrue2={Nie},%
		answer2={Wykorzystują model wspólnej pamięci},%
		explain2={Nie, jest rozproszona fizycznie i logicznie.},%
		isTrue3={Tak},%
		answer3={W węzłach mogą wykorzystywać systemy SMP},%
		explain3={Tak, serwery SMP są jednymi z dopuszczalnych węzłów. Drugimi są pełne komputery PC. Patrz: węzły w klastrach \ref{subsubsec:wezlyKlastry}},%
		isTrue4={Tak},%
		answer4={Do komunikacji między procesami wykorzystują przesył komunikatów},%
		explain4={Tak, bo jest efektownym rozwiązaniem, i tylko go wykorzystują.},%
		isTrue5={Nie},%
		answer5={Wykorzystują przełącznicę krzyżową jako sieć łączącą węzły}, %
		explain5={Nie, ona jest wykorzystywana tylko w systemach UMA, gdzie pamięć wspólna jest fizycznie jednorodna.}, %
		isTrue6={Tak}, %
		answer6={W każdym węźle posiadają pełną instalację systemu operacyjnego}, %
		explain6={Tak, węzłem musi być albo serwer SMP, albo PC, które muszą mieć swoje systemy operacyjne.}, %
	}
\end{minipage}
\begin{minipage}{\textwidth}
	\question{%
		question={Do czynników tworzących wysoką niezawodność klastrów należą:}%
	}{%
		isTrue1={Tak},%
		answer1={Mechanizm mirroringu dysków},%
		explain1={Tak, bo system operacyjny może.},%
		isTrue2={Tak},%
		answer2={Dostęp każdego węzła do wspólnych zasobów (pamięci zewnętrznych)},%
		explain2={Tak, w razie czego można podpiąć i korzystać z dodatkowej pamięci.},%
		isTrue3={Tak},%
		answer3={Redundancja węzłów},%
		explain3={No tak, jest.},%
		isTrue4={Nie},%
		answer4={Mechanizm "heartbeat"},%
		explain4={Co to jest?},%
		isTrue5={Nie},%
		answer5={Zastosowanie procesorów wielordzeniowych w węzłach}, %
		explain5={Nie, nie chodzi o liczbę rdzeni, ale o to, że każdy węzeł jest osobnych systemem, serwerem lub pecetem.}, %
	}
\end{minipage}
\begin{minipage}{\textwidth}
	\question{%
		question={Dla sieci systemowych (SAN) są charakterystyczne}%
	}{%
		isTrue1={Tak},%
		answer1={Przesył komunikatów w trybie zdalnego DMA},%
		explain1={Tak, bo przesyłamy dane między prockami, a DMA wykonuje to najszybciej.},%
		isTrue2={Tak},%
		answer2={Bardzo małe czasy opóźnień},%
		explain2={Tak, rzędu pojedynczych mikrosekund.},%
		isTrue3={Nie},%
		answer3={Topologia typu hipersześcian},%
		explain3={Bullshit, sieć jest taka jak topologia systemu, nie buduje się osobnej topologii.},%
		isTrue4={Nie},%
		answer4={Niska przepustowość},%
		explain4={Noelonie, do pamięci potrzebna jest duża przepustowość, bo przechodzi przez nią dużo danych.},%
	}
\end{minipage}
\begin{minipage}{\textwidth}
	\question{%
		question={Czy poniższa lista jest rosnąco uporządkowana według skalowalności:}%
	}{%
		isTrue1={Nie},%
		answer1={Systemy ściśle połączone, systemy ze wspólną pamięcią, systemy SMP},%
		explain1={Systemy SMP to cała kategoria systemów z pamięcią wspólną, z kolei systemy ściśle połączone i systemy ze wspólną pamięcią są \textbf{równoznaczne} - są przeciwieństwem do systemów luźno powiązanych (z pamięcią rozproszoną).},%
		isTrue2={Tak},%
		answer2={Systemy ze wspólną magistralą, systemy wielomagistralowe, systemy z przełącznicą krzyżową},%
		explain2={Są to systemy wieloprocesorowe (UMA) z pamięcią wspólną, patrz: Klasyfikacja  \ref{subsubsec:klasyfikacjaUMA}\\
			- \emph{Systemy ze wspólna magistralą} - najprostsze i najmniej skalowalne\\
			- \emph{Systemy wielomagistralowe} - szybsze i bardziej złożone, wciąż kiepsko skalowalne\\
			- \emph{Systemy z przełącznicą krzyżową} - duża szybkość i złożoność obliczeniowa, trudne w rozbudowie\\
			Ogółem jest to dolna półka tych systemów.},%
		isTrue3={Nie},%
		answer3={Systemy SMP, systemy z pamięcią wieloportową, systemy z przełącznicą krzyżową},%
		explain3={SMP to rodzaj architektury, z kolei w systemach UMA systemy z przełącznicą krzyżową są mniej skalowalne niż systemy z pamięcią wieloportową, patrz: Klasyfikacja  \ref{subsubsec:klasyfikacjaUMA}.},%
		isTrue4={Nie},%
		answer4={NUMA, MPP, SMP},%
		explain4={MPP jest znacznie bardziej skalowalny niż SMP (pamięć rozproszona $ > $ pamięć wspólna). NUMA to systemy SMP z niejednorodnym dostępem do pamięć - są bardziej skalowalne niż zwykłe SMP (dzięki szybkiej pamięci lokalnej \emph{cache}), ale mniej niż MPP.},%
		isTrue5={Tak},%
		answer5={Systemy z pamięcią wspólną, systemy o niejednorodnym dostępie do pamięci, z pamięcią rozproszoną}, %
		explain5={Dwa pierwsze to rodzaje systemów SMP. Najmniej skalowalne są systemy z pamięcią wspólną, domyślnie o jednorodnym dostępie do pamięci (UMA). Niejednorodny dostęp do pamięci wspólnej (NUMA) jest szybszy, ponieważ wykorzystuje pamięć lokalną procesora, węzły i katalogi. Mechanizm katalogów jest o wiele bardziej skalowalny niż mechanizm "\emph{snoopingu}", wykorzystywany w UMA. Następnie system z pamięcią rozproszoną to MPP - system masywnie równoległy. Jest najbardziej skalowalny ze wszystkich.}, %
		isTrue6={Nie}, %
		answer6={SMP, NUMA, klastry, UMA}, %
		explain6={SMP jest najmniej skalowalny z wymienionych. UMA ma jednorodny dostęp do pamięci i jest mniej skalowalna od NUMA. Klastry są najbardziej skalowalne (nie wiem czy mniej lub bardziej do MPP).}, %
		isTrue7={Tak}, %
		answer7={Systemy symetryczne, o niejednorodnym dostępie do pamięci, systemy z przesyłem komunikatów}, %
		explain7={Systemy symetryczne to SMP z jednorodnym dostępem do pamięci. Systemy SMP z niejednorodnym dostępem są bardziej skalowalne. Z kolei systemy z przesyłem komunikatów sugerują system MPP, z pamięcią rozproszoną - jest on najbardziej skalowalny.}, %
	}
\end{minipage}
\begin{minipage}{\textwidth}
	\question{%
		question={Sprzętowe przełączenie wątków może być wynikiem:}%
	}{%
		isTrue1={Tak},%
		answer1={Chybienia przy odwołaniu do pamięci podręcznej.},%
		explain1={Patrz: realizacja sprzętowej wielowątkowości \ref{subsec:realizacjaWatkow}},
		isTrue2={Nie},%
		answer2={Upływu zadanego czasu (np. taktu)},%
		explain2={Bullshit, szanujmy się. Można to zrobić programowo, ale nie sprzętowo.},%
		isTrue3={Nie},%
		answer3={Wystąpienia rozkazu rozgałęzienia},%
		explain3={Dopiero jak nastąpi błąd przewidywania. Przy drabince IFów byłby armageddon.},%
		isTrue4={Tak},%
		answer4={Błędnego przewidywania rozgałęzień},%
		explain4={Patrz: realizacja sprzętowej wielowątkowości \ref{subsec:realizacjaWatkow}},%
		isTrue5={Nie},%
		answer5={Przesunięcia okien rejestrów},%
		explain5={To tylko zmiana rejestrów, niezwiązana z wątkami.},%
	}
\end{minipage}
\end{enumerate}