% !TeX spellcheck = pl_PL
\documentclass[a4paper,twoside]{article}
\usepackage{polski}
\usepackage[utf8]{inputenc}
\usepackage{graphicx}
\usepackage{amsmath}

\usepackage[unicode, bookmarks=true]{hyperref} %do zakładek
\usepackage{tabto} % do tabulacji
\NumTabs{6} % globalne ustawienie wielkosci tabulacji
\usepackage{array}
\usepackage{multirow}
\usepackage{array}
\usepackage{dcolumn}
\usepackage{bigstrut}
\usepackage{color}
\usepackage[usenames,dvipsnames]{xcolor}
\usepackage{wrapfig}
\usepackage{listings,lstautogobble}
\usepackage[usenames,dvipsnames]{xcolor}
\usepackage{epigraph}

\newcommand{\HRule}{\rule{\linewidth}{0.5mm}}
\setlength{\textheight}{24cm}
\setlength{\textwidth}{15.92cm}
\setlength{\footskip}{10mm}
\setlength{\oddsidemargin}{0mm}
\setlength{\evensidemargin}{0mm}
\setlength{\topmargin}{0mm}
\setlength{\headsep}{5mm}

\newcolumntype{M}[1]{>{\centering\arraybackslash}m{#1}}
\newcolumntype{N}{@{}m{0pt}@{}}

\graphicspath{ {./images/} }

% === Reset inkrementacji sekcji przy nowym parcie === %
\usepackage{titlesec}

\makeatletter
\@addtoreset{section}{part}
\makeatother
\titleformat{\part}[display]
{\normalfont\LARGE\bfseries\centering}{}{0pt}{}


% --- Listing dialektu SPARC Assemblera
\lstdefinelanguage
[sparc]{Assembler}		%
[x86masm]{Assembler}	% based on the "x86masm" dialect
{
	morekeywords= %
	{
		ld, LD, st, ST, %
		mov, MOV, swap, SWAP, %
		nop, NOP, %
		AND, and, ANDcc, andcc, OR, or, ORcc, orcc, XOR, xor, XORcc, xorcc, %
		ADDX, addx, ADDXcc, addxcc, %
		SUB, sub, SUBcc, subcc, SUBX, subx, SUBXcc, subxcc, %
		UMUL, umul, SMUL, smul, UMULcc, umulcc, SMULcc, smulcc, %
		UDIV, udiv, SDIV, sdiv, UDIVcc, udivcc, SDIVcc, sdivcc, %
		BA, ba, BNE, bne, BG, bg, BNEG, bneg, bge, BGE, be, BE,%
		CALL, call, RET, ret, RETL, retl, %
		SAVE, save, RESTORE, restore %
	}	
}

% --- Opcje listingu kodu
\lstset{
	frame=single,
	autogobble=true,
	commentstyle=\ttfamily\itshape\color{ForestGreen},
	keywordstyle=\color{blue},
	stringstyle=\color[rgb]{0.64,0.08,0.08},
	numberstyle=\color[rgb]{0.205, 0.142, 0.73},
	frameround=ffff,
	rulecolor=\color{black},
	tabsize=4,
	breaklines=true, %
	% --- Polskie znaki w listingu kodu
	literate=%
	{ą}{{\c{a}}}1
	{ć}{{\'c}}1
	{ę}{{\c{e}}}1
	{ł}{{\l{}}}1
	{ń}{{\'n}}1
	{ó}{{\'o}}1
	{ś}{{\'s}}1
	{ź}{{\'z}}1
	{ż}{{\.z}}1
	{Ą}{{\c{A}}}1
	{Ć}{{\'C}}1
	{Ę}{{\c{E}}}1
	{Ł}{{\L{}}}1
	{Ń}{{\'N}}1
	{Ó}{{\'O}}1
	{Ś}{{\'S}}1
	{Ź}{{\'Z}}1
	{Ż}{{\.Z}}1
}

\begin{document}
	\bibliographystyle{plain}
	% === DEFINICJA ZIELONEGO ==================== %
\definecolor{Gurin}{rgb}{0, 0.35, 0}

% === MAKRODEFINICJA POPRAWNEJ I ZŁEJ ODPOWIEDZI ==================== %
\newcommand{\Tak}[1] {
	\color{Gurin}{#1}
}
\newcommand{\Nie}[1] {
	\color{Red}{#1}
}

% === Porównywarka odpowiedzi
\newcommand{\answer}[3] {
	\ifnum\pdfstrcmp{#1}{Tak}=0
	\Tak{\item \textbf{#2}}
	\else
	\Nie{\item #2}
	\fi
	\if\relax\detokenize{#3}\relax
	\else
	\\
	\fi
	\color{Black}{\emph{#3}}\\
}

% === MAKRODEFINICJA PYTANIA I ODPOWIEDZI =========================== %
\newcommand{\question}[2]{ %
	\setkeys{Question}{#1} %
	\begin{itemize}
		\setkeys{Answers}{#2}
	\end{itemize}
}

% definicje treści pytania i odpowiedzi
\makeatletter
\define@key{Question}{question}{\item \textbf{#1}}
\define@key{Answers}{isTrue1}{\def\QA@isTrueI{#1}}
\define@key{Answers}{answer1}{\def\QA@answerI{#1}}
\define@key{Answers}{explain1}{\answer{\QA@isTrueI}{\QA@answerI}{#1}}
\define@key{Answers}{isTrue2}{\def\QA@isTrueII{#1}}
\define@key{Answers}{answer2}{\def\QA@answerII{#1}}
\define@key{Answers}{explain2}{\answer{\QA@isTrueII}{\QA@answerII}{#1}}
\define@key{Answers}{isTrue3}{\def\QA@isTrueIII{#1}}
\define@key{Answers}{answer3}{\def\QA@answerIII{#1}}
\define@key{Answers}{explain3}{\answer{\QA@isTrueIII}{\QA@answerIII}{#1}}
\define@key{Answers}{isTrue4}{\def\QA@isTrueIV{#1}}
\define@key{Answers}{answer4}{\def\QA@answerIV{#1}}
\define@key{Answers}{explain4}{\answer{\QA@isTrueIV}{\QA@answerIV}{#1}}
\define@key{Answers}{isTrue5}{\def\QA@isTrueV{#1}}
\define@key{Answers}{answer5}{\def\QA@answerV{#1}}
\define@key{Answers}{explain5}{\answer{\QA@isTrueV}{\QA@answerV}{#1}}
\define@key{Answers}{isTrue6}{\def\QA@isTrueVI{#1}}
\define@key{Answers}{answer6}{\def\QA@answerVI{#1}}
\define@key{Answers}{explain6}{\answer{\QA@isTrueVI}{\QA@answerVI}{#1}}
\define@key{Answers}{isTrue7}{\def\QA@isTrueVII{#1}}
\define@key{Answers}{answer7}{\def\QA@answerVII{#1}}
\define@key{Answers}{explain7}{\answer{\QA@isTrueVII}{\QA@answerVII}{#1}}
\define@key{Answers}{isTrue8}{\def\QA@isTrueVIII{#1}}
\define@key{Answers}{answer8}{\def\QA@answerVIII{#1}}
\define@key{Answers}{explain8}{\answer{\QA@isTrueVIII}{\QA@answerVIII}{#1}}
\define@key{Answers}{isTrue9}{\def\QA@isTrueIX{#1}}
\define@key{Answers}{answer9}{\def\QA@answerIX{#1}}
\define@key{Answers}{explain9}{\answer{\QA@isTrueIX}{\QA@answerIX}{#1}}
\define@key{Answers}{isTrue10}{\def\QA@isTrueX{#1}}
\define@key{Answers}{answer10}{\def\QA@answerX{#1}}
\define@key{Answers}{explain10}{\answer{\QA@isTrueX}{\QA@answerX}{#1}}

%===============================================================================
%*** Zadania egzaminacyjne *****************************
%===============================================================================
	% !TeX spellcheck = pl_PL
\newpage

\section{Sparc}
	Uwagi:
	\begin{itemize}
		\item W języku asemblera SPARC komentarze są oznaczane przez znak wykrzyknika (!), a nie średnika (;). W listingach są średniki ze względu na wbudowany listingu asemblera w latexie.
	\end{itemize}

	\subsection{2008, I termin, Jerzy Respondek}
		\subsubsection{Treść}
			Napisz funkcję w asemblerze procesora SPARC obliczającą sumę liczb naturalnych od 1 do danej $ n $ jako argument funkcji. Założyć, że $ n >= 1 $.\\
			Przykład: f(5) = 1 + 2 + 3 + 4 + 5 = 15
		\subsubsection{Propozycja rozwiązania 1}
			\begin{lstlisting}[language={[sparc]Assembler}]
				.global funkcja
				.proc 4
				funkcja:
					save %sp, -96, %sp		; trzeba tutaj to robić ???
					mov %i0, %l0			; a
					mov 1, %l1				; liczba naturalna   
					mov 0, %l2				; wynik
				pętla:
					add %l1, %l2, %l2		; liczba + suma = suma
					add %l1, 1, %l1			; liczba++
					subbcc %l0, 1, %l0		; a--
					bl koneic
					nop
					ba pętla
					nop
				koniec:
					mov %l2, %i0			; wynik
					ret
					restore
			\end{lstlisting}
		\subsubsection{Propozycja rozwiązania 2}
			\begin{lstlisting}[language={[sparc]Assembler}]
			    .global sumator
			    .proc 4
			    sumator:
				    save %sp, -96, %sp		! przesunięcie okna
				    mov %i0, %l1			! a w l1
				    mov %l1, %l0			! suma = a
				   petla:
				    subcc %l1, 1, %l1		! dekrementacja licznika
				    bneg koniec
				    add %l0, %l1, %l0		! suma += licznik
				    ba petla
			    koniec:
				    mov %l0, %i0			! zwrócenie sumy
				    ret
				    restore					! przywrócenie stanu okna
			\end{lstlisting}
			
	\newpage
	\subsection{2010, I termin, Jerzy Respondek}
		\subsubsection{Treść}
			Napisz w asemblerze procesora SPARC funkcję obliczającą sumę kwadratów wszystkich liczb całkowitych z przedziału \emph{a} do \emph{b}. Założyć $ a < b $, np.\\
			f(2, 5) = 2 * 2 + 3 * 3 + 4 * 4 + 5 * 5\\
			Nagłówek funkcji ma mieć postać:
			\begin{lstlisting}[language=C]
				int f(int a, int b)
			\end{lstlisting}
		\subsubsection{Propozycja rozwiązania}
	\newpage
	\subsection{2012, I termin, Jerzy Respondek}
		\subsubsection{Treść}
			Napisz w asemblerze procesora SPARC funkcję realizującą dokładnie tę samą operację co jej odpowiednik w języku C:
			\begin{lstlisting}[language=C]
				int f(int *tab, int n)
				{
					int i, suma = 0;
					for(i = 0; i < n; i++)
					{
						suma -= (2 * i + 1) * tab[i];
						suma *= suma;
					}
					return suma;
				}
			\end{lstlisting}
		\subsubsection{Propozycja rozwiązania 1}
			\begin{lstlisting}[language={[sparc]Assembler}]
				.global func
				.proc 4
				
				funkcja:
					save %sp, -96, %sp
					mov %i0, %l0			; wskaźnik tablicy, tak podano argument
					ld [%i0], %l1			; wartość tablicy spod wskaźnika odczytujemy poprzez LD
					mov %i1, %l2			; rozmiar
					mov 1, %l3				; i
					mov 0, %l4				; temp
					mov 0, %l5				; suma
				pętla:
					subcc %l2, 1, %l2		; n--
					bl koniec				; if n < 0 koniec
					nop
					
					smul %l3, 2, %l4		; temp = 2*i
					add %l4, 1, %l4			; temp = temp +1 = 2*i+1
					smul %l1, %l4, %l4		; temp = temp * tab[i] = (2*i+1)*tab[i]
					subcc %l5, %l4, %l5		; suma = suma - temp = suma - (2*i+1)*tab[i]
					
					smul %l5, %l5, %l5		; suma = suma * suma
					add %l0, 4, %l0			; *tab++ przesuwamy sie o 4 na kolejny element bo tyle ma int
					ld [%l0], %l1			;pobieramy nowy element
					ba pętla
					nop
				
				koniec:
					mov %l5, %i0			; zwracamy wynik w i0 bo po restore zamienia się input na output
					ret						; ret bo było save
					restore
			\end{lstlisting}
		\newpage
		\subsubsection{Propozycja rozwiązania 2}
			\begin{lstlisting}[language={[sparc]Assembler}]
				.global fun
				.proc 4
				
				;   a(n) = a(n - 1) ^ k + n * k; a(0) = 1
				fun:
					save %sp, -96, %sp
					; %i0 == n
					; %i1 == k
				
					subcc %i0, 1, %o0	; %o0 == n - 1
					bneg return1
					nop
				
					; trzeba obliczyc a(n - 1)
					mov %i1, %o1
					call fun
					nop
				
					; %o0 == a(n - 1)
					mov %i1, %l1		; %l1 == k
					mov 1, %l2			; %l2 == 1 (tu bedzie wynik potegowania)
				power:
					umul %l2, %o0, %l2
					subcc %l1, 1, %l1	; dekrementuj licznik petli
					bg power			; skok, gdy licznik > 0
					nop
				
					; %l2 == a(n - 1) ^ k
					umul %i0, %i1, %i0
					; %i0 == n * k
					add %i0, %l2, %i0
					; %i0 == a(n - 1) ^ k + n * k == a(n)
					ba return
					nop
				
				return1:
					mov 1, %i0
				return:
					ret
					restore
			\end{lstlisting}
		
	\newpage
	\subsection{2013, I termin, Jerzy Respondek}
		\subsubsection{Treść}
			Napisz w asemblerze procesora SPARC funkcję zwracającą \emph{a(n)} wyliczoną z poniższego wzoru rekurencyjnego, a pobierającą dwa argumenty: \emph{n} oraz \emph{k}, obydwa typu \textit{unsigned int}.
			$$ a(n)=a(n-1)^k+n\cdot k,\;\;\;\;a(0)=1,\;\;\;n=1,2,3,... $$
		\subsubsection{Rozwiązanie}
			Podobno otrzymano za to 5, choć rozwiązanie NIE JEST w pełni poprawne.
			\begin{lstlisting}[language={[sparc]Assembler}]
				.global fun
				.proc 4
				
				fun:
					save %sp,-96,%sp
					
					mov %i0, %l0			; l0 - n
					mov %i1, %l1			; l1 - k
					mov 0, %l2				; power
					mov 1, %l3				; a(n) = 1
					
					subcc %i0, 1, %i0
					bl theEnd				; if n = 0 then jump to theEnd
					nop
					
					mov %l0, %l2			; power = n
					smul %l2, %l1, %l2		; power = power * k
					add %l2, %l1, %l2		; power = power + k
					
					call fun				; call recursion
					mov %i0, %l3			; get score of recursion
				
				expo:
					smul %l3, %l3, %l3
					subcc %l2, 1, %l2
					bl theEnd
					nop
					ba expo
					nop
				theEnd:
					mov %l3, %i0			; return score
					ret
					restore
				
				.end
			\end{lstlisting}
		
		
		
		
		
		
		
		
		
		
		
		
		
		
		
		
		
		
		
		
		
		
		
		
		
		
	% !TeX spellcheck = pl_PL
\newpage
\section{PVM}
	\subsection{Wstęp z laborek, szukanie min i max}
		\subsubsection{Treść}
			Napisać program znajdujący minimum i maksimum z macierzy.\\
			Hello.c - program główny, rodzic; Hello\_other.c - program podrzędny, potomek.
		\subsubsection{Rozwiązanie}
			Program przekazuje kolejne wiersze macierzy do programów potomnych, które znajdują lokalne minimum i maksimum. Program zbiera wszystkie minima i maksima do tablicy o rozmiarze wysokości macierzy. Pod koniec sam ręcznie wylicza min i max z tych dwóch tablic.\\
			Należy pamiętać, że programy potomne muszą fizycznie znajdować się na dyskach innych komputerów w sieci PVM.\\
			\textbf{Program działający, oceniony na 5.}
			\begin{lstlisting}[language={C}]
				/* - Autorzy:
				   -- Forczu Forczmański
				   -- Wuda Wudecki
				*/
				#include <stdio.h>
				#include <stdlib.h>
				#include <math.h>
				#include "pvm3.h"
				#define WYSOKOSC 5		// liczba wierszy
				#define SZEROKOSC 5		// liczba kolumn
				/// Program rodzica
				main()
				{
					// dane potrzebne do obliczeń
					int matrix[WYSOKOSC][SZEROKOSC];
					int min_result[WYSOKOSC], max_result[WYSOKOSC];
					int minimum, maksimum;
					// wypełnienie macierzy danymi
					int i, j;
					for ( i = 0; i < WYSOKOSC; ++i )
						for ( j = 0; j < SZEROKOSC; ++j)
							matrix[i][j] = rand() % 30;
					// wypisanie macierzy na konsoli
					for ( i = 0; i < WYSOKOSC; ++i )
					{
						for ( j = 0; j < SZEROKOSC; ++j)
							printf("%d ", matrix[i][j]);
						printf("\n\n");
					}
					// pobranie informacji
					int ilhost, ilarch;
					struct pvmhostinfo * info;
					pvm_config(&ilhost, &ilarch, &info);
					printf("Liczba hostow: %d\n", ilhost);
					
					int id1 = 0;
					int tid;
					
					
					
					// Dla każdego hosta - inicjujemy go
					for ( i = 0; i < ilhost; i++ )
					{
						pvm_spawn( "/home/pvm/pvm3/sekcja11/bin/LINUX/hello_other", 0, PvmTaskHost, info[i].hi_name, 1, &tid);
						if ( tid < 0 )
						{
							ilhost--;
							continue;
						}
						printf("tid: %d\n", tid);
						pvm_initsend(PvmDataDefault);
						// wysyłamy:
						// id wiersza
						pvm_pkint(&id1, 1, 1);
						// elementy wiersza
						pvm_pkint(&matrix[id1][0], SZEROKOSC, 1);
						pvm_send(tid, 100);
						id1++;
					}
					//// Wykonywanie programu aż do przedostatniej pętli
					int bufid, child_tid, child_id1, tmp;
					while ( id1 < WYSOKOSC )
					{
						bufid = pvm_recv(-1, 200);
						pvm_bufinfo(bufid, &tmp, &tmp, &child_tid);
						printf("recv: %d\n", child_tid);
						// pobranie id wiersza
						pvm_upkint(&child_id1, 1, 1);
						// pobranie nowych min / max
						pvm_upkint(&min_result[child_id1], 1, 1);
						pvm_upkint(&max_result[child_id1], 1, 1);
						// wysłanie nowych danych
						pvm_initsend(PvmDataDefault);
						// id kolejnego wiersza
						pvm_pkint(&id1, 1, 1);
						// nowy wiersz
						pvm_pkint(&matrix[id1][0], SZEROKOSC, 1);
						pvm_send(child_tid, 100);
						id1++;
					}
					//// Odebranie ostatnich danych
					for	(i = 0; i < id1 - ilhost + 1; i++ )
					{
						bufid = pvm_recv(-1, 200);
						pvm_bufinfo(bufid, &tmp, &tmp, &child_tid);
						printf("recv: %d\n", child_tid);
						// pobranie id wiersza
						pvm_upkint(&child_id1, 1, 1);
						// pobranie nowych min / max
						pvm_upkint(&min_result[child_id1], 1, 1);
						pvm_upkint(&max_result[child_id1], 1, 1);
					}
					
					
					
					
					// uzysaknie minimum z wiersza
					minimum = min_result[0];
					maksimum = max_result[0];
					for (j = 1; j < WYSOKOSC; j++)
					{
						if ( max_result[j] > maksimum )
							maksimum = max_result[j];
						if ( min_result[j] < minimum )
							minimum = min_result[j];
					}
					printf("Uzyskane wartosci:\nMIN: %d, MAX: %d\n", minimum, maksimum);
					pvm_exit();
					return 0;
				}
			\end{lstlisting}
			\textbf{Program potomka}
			\begin{lstlisting}[language={C}]
				#include <stdio.h>
				#include <math.h>
				#include "pvm3.h"
				#define WYSOKOSC 5		// liczba wierszy
				#define SZEROKOSC 5		// liczba kolumn
				/// Program potomka
				int main()
				{
					int masterid, id1, j, curr_row[SZEROKOSC], curr_min, curr_max;
					// pobierz id rodzica 
					masterid = pvm_parent();
					if (masterid == 0)
						exit(1);
					while(1)
					{
						pvm_recv(masterid, 100);
						// pobranie wartości:
						// id wiersza
						pvm_upkint(&id1, 1, 1);
						pvm_upkint(&curr_row[0], SZEROKOSC, 1);
						// uzysaknie minimum z wiersza
						curr_min = curr_max = curr_row[0];
						for (j = 1; j < SZEROKOSC; j++)
						{
							if ( curr_row[j] > curr_max )
								curr_max = curr_row[j];
							if ( curr_row[j] < curr_min )
								curr_min = curr_row[j];
						}
						// wysłanie nowych danych
						pvm_initsend(PvmDataDefault);
						pvm_pkint(&id1, 1, 1);
						pvm_pkint(&curr_min, 1, 1);
						pvm_pkint(&curr_max, 1, 1);
						pvm_send(masterid, 200);
					}
					pvm_exit();
					return 0;
				}
			\end{lstlisting}
	\subsection{Laborki, odejmowanie macierzy}
		\subsubsection{Treść}
			Odejmowanie macierzy.
		\subsubsection{Rozwiązanie}
			Ocena nieznana.
			\begin{lstlisting}[language=C]
				/* AK Lab 2 - PVM
					Anna Kusnierz
					Tomasz Szoltysek
					Temat: Odejmowanie dwoch macierzy
				*/
				#include <stdio.h>
				#include <math.h>
				#include "pvm3.h"
				#define MATRIX_SIZE 20
				int main() 
				{
					int i,j;
					int count = 0; 		//licznik wierszy macierzy
					int rescount;
					int tidmaster, ilhost, ilarch, bufid,t_id,bytes,msgtag;
					struct pvmhostinfo info;
					
					int a[MATRIX_SIZE][MATRIX_SIZE], b[MATRIX_SIZE][MATRIX_SIZE], r[MATRIX_SIZE][MATRIX_SIZE];
					FILE *txt = fopen("result.txt","w");
					
					for(i=0;i<MATRIX_SIZE;i++)
					{
						for(j=0;j<MATRIX_SIZE;j++)
						{
							a[i][j] = rand();
							b[i][j] = rand();
						}
					}
					fprintf(txt,"Macierz A:\n--------------------------------------------------\n\n");
					for(i=0;i<MATRIX_SIZE;i++)
					{
						for(j=0;j<MATRIX_SIZE;j++)
							fprintf(txt,"%d\t",a[i][j]);
						fprintf(txt,"\n");
					}
					fprintf(txt,"Macierz B:\n--------------------------------------------------\n\n");
					for(i=0; i<MATRIX_SIZE;i++)
					{
						for(j=0;j<MATRIX_SIZE;j++)
							fprintf(txt,"%d\t",b[i][j]);
						fprintf(txt,"\n");
					}
					
					
					
					tidmaster = pvm_mytid();
					pvm_config(&ilhost, &ilarch, &info);
					printf("%d",ilhost);
					for(i=0; i < (ilhost > MATRIX_SIZE ? MATRIX_SIZE : ilhost) ;i++)
					{
						pvm_spawn("/home/pvm3/pvm3/sekcja4/hello_other",0,PvmTaskHost,info[i].hi_name,1,&t_id);
						pvm_initsend(PvmDataDefault);
						pvm_pkint(&a[count][0],MATRIX_SIZE,1);
						pvm_pkint(&b[count][0],MATRIX_SIZE,1);
						pvm_pkint(&count,1,1);
						pvm_send(t_id,100);	
						++count;
					}
					while(count<MATRIX_SIZE)
					{
						bufid = pvm_recv(-1,200);
						pvm_bufinfo(bufid,&bytes,&msgtag,&t_id);
						pvm_upkint(&rescount,1,1);
						pvm_upkint(&r[rescount][0],MATRIX_SIZE,1);
						pvm_initsend(PvmDataDefault);
						pvm_pkint(&a[count][0],MATRIX_SIZE,1);
						pvm_pkint(&b[count][0],MATRIX_SIZE,1);
						pvm_pkint(&count,1,1);
						pvm_send(t_id,100);
						++count;
					}
					for(i = 0; i < (ilhost > MATRIX_SIZE ? MATRIX_SIZE : ilhost); i++)
					{
						bufid = pvm_recv(-1,200);
						pvm_bufinfo(bufid,&bytes,&msgtag,&t_id);
						pvm_upkint(&rescount,1,1);
						pvm_upkint(&r[rescount][0],MATRIX_SIZE,1);
						pvm_kill(t_id);	
					}
					fprintf(txt,"Macierz wynikowa:\n--------------------------------------------------\n\n");
					for(i=0; i<MATRIX_SIZE;i++)
					{
						for(j=0;j<MATRIX_SIZE;j++)
							fprintf(txt,"%d\t",r[i][j]);
						fprintf(txt,"\n");
					}
					fclose(txt);
					exit(0);
				}
			\end{lstlisting}
			\newpage
			\textbf{Program potomny}
			\begin{lstlisting}[language=C]
				#include <stdio.h>
				#include "pvm3.h"
				#define MATRIX_SIZE 20
				int main()
				{
					int masterid, count, i;
					double vecta[MATRIX_SIZE],vectb[MATRIX_SIZE],vectr[MATRIX_SIZE];
					masterid = pvm_parent();
					if(masterid == 0) exit(1);	//zabezpieczenie przed uruchomieniem z poziomu rodzica
					//OBSŁUGA OBLICZEŃ i WYSYŁANIA WYNIKÓW
					while(1)
					{
						pvm_recv(masterid,100);
						pvm_upkdouble(&vecta[0],MATRIX_SIZE,1);
						pvm_upkdouble(&vectb[0],MATRIX_SIZE,1);
						pvm_upkint(&count,1,1);
						for(i = 0; i < MATRIX_SIZE; i++)
							vectr[i] = vecta[i] - vectb[i];
						pvm_initsend(PvmDataDefault);
						pvm_pkint(&count,1,1);
						pvm_pkdouble(&vectr[0],MATRIX_SIZE,1);
						pvm_send(masterid,200);		
					}
					return(0);
				}
			\end{lstlisting}
			
	% !TeX spellcheck = pl_PL
\newpage

\section{Java Spaces}
	\subsection{Wstęp z laborek}
		\subsubsection{Treść}
			Napisać program zawierający jednego Nadzorcę oraz wielu Pracowników. Nadzorca przekazuje do JavaSpace 2 równe tablice zawierające obiekty typu Integer, a następnie otrzymuje wynikową tablicę zawierającą sumy odpowiadających sobie komórek. Operację dodawania mają realizować Pracownicy.
		\subsubsection{Rozwiązanie}
			Zadanie obliczania sumy tabel dzielimy na dwie części: \textit{Task} oraz \textit{Result}. \textit{Taski} są generowane przez \textit{Nadzorcę} i przekazywane \textit{Pracownikom}, ci zaś wykonują zadanie i tworzą obiekty klasy \textit{Result}, a następnie przekazuję je \textit{Nadzorcy}. \textit{Nadzorca} je odbiera, kompletuje i ew. coś z nimi robi.\\\\
			\textit{Nadzorca} przydziela tyle zadań, ile potrzebuje, z kolei \textit{Pracownicy} działają w nieskończoność. Aby zakończyć ich pracę, \textit{Nadzorca} musi wysłać zadania z tzw. zatrutą pigułką (ang. \emph{Poisoned Pill}), czyli obiekt zadania z nietypowym parametrem, który sygnalizuje zakończenie pracy. Może to być np. \textit{Boolean} o wartości \textit{false}, \textit{Integer} o wartości -1, itp.
			Składowymi klas implementujących interfejs Entry nie mogą być typu prostego (int, double itp.), muszą być opakowane (Integer, Double itp.). Najbezpieczniej dawać je wszędzie.\\
			\begin{lstlisting}[language=Java]
				/**
				* @author Son Mati
				* @waifu Itsuka Kotori
				*/
				public class Task implements Entry {
					public Integer cellID;	// ID komórki tabeli
					public Integer valueA;	// wartość z tabeli A
					public Integer valueB;	// wartość z tabeli B
					public Boolean isPill;	// czy zadanie jest zatrutą pigułką

					// Domyślny konstruktor, musi się znajdować
					public Task() {
						this.cellID = this.valueA = this.valueB = null;
						this.isPill = false;
					}
					
					public Task(Integer entryID, Integer valueA, Integer valueB, Boolean isPill) {
						this.cellID = entryID;
						this.valueA = valueA;
						this.valueB = valueB;
						this.isPill = isPill;
					}
				}
			\end{lstlisting}
			\newpage
			\begin{lstlisting}[language=Java]
				/**
				* @author Son Mati
				* @waifu Itsuka Kotori
				*/
				public class Result implements Entry {
					public Integer cellID, value;
					public Result() {
						this.cellID = this.value = null;
					}
					public Result(final Integer EntryID, final Integer Value) {
						this.cellID = EntryID;
						this.value = Value;
					}
				}
			\end{lstlisting}
			\begin{lstlisting}
				public class Client {
					protected Integer defaultLease = 100000;
					protected JavaSpace space;
					protected Lookup lookup;
					public Client() {
						lookup = new Lookup(JavaSpace.class);
					}
				}
			\end{lstlisting}
			\begin{lstlisting}[language=Java]
				/**
				* @author Son Mati
				* @waifu Itsuka Kotori
				*/
				public class Worker extends Client {
					public Worker() {
					}
					public void startWorking() {
						while(true) {
							try {
								this.space = (JavaSpace)lookup.getService();
								Task task = new Task();
								task = (Task) space.take(task, null, defaultLease);
								if (task.isPill == true)
								{
									space.write(task, null, defaultLease);
									System.out.println("Koniec pracy workera.");
									return;
								}
								Integer res = task.valueA + task.valueB;
								Result result = new Result(task.cellID, res);
								space.write(result, null, defaultLease);
							}
							catch (Exception ex) {}
						}
					}
					public static void main(String[] args) {
						Worker w = new Worker();	// utworzenie obiektu
						w.startWorking();			// realizacja zadan
					}
				}
			\end{lstlisting}
			\newpage
			\begin{lstlisting}[language=Java]
				/**
				* @author Son Mati
				* @waifu Itsuka Kotori
				*/
				public class Supervisor extends Client {
					static final Integer INT_NUMBER = 125;
					public Integer[] TableA = new Integer[INT_NUMBER];
					public Integer[] TableB = new Integer[INT_NUMBER];
					public Integer[] TableC = new Integer[INT_NUMBER];
					// konstruktor
					public Supervisor() {
					}
					// wygenerowanie zawartości tablic
					public void generateData() {
						Random rand = new Random();
						for (int i = 0; i < INT_NUMBER; ++i) {
							TableA[i] = rand.nextInt(INT_NUMBER);
							TableB[i] = rand.nextInt(INT_NUMBER);
							TableC[i] = 0;
						}
					}
					// rozpoczecie pracy
					public void startProducing() {
						try {
							this.space = (JavaSpace)lookup.getService();
							// utworzenie zadania
							for (Integer i = 0; i < INT_NUMBER; ++i) {
								Task task = new Task(i, this.TableA[i], this.TableB[i], false);
								space.write(task, null, defaultLease);
							}
							// pobranie wyniku zadania
							System.out.println("Tablica C:");
							for(Integer i = 0 ; i < INT_NUMBER; ++i) {
								Result result = new Result();
								result = (Result) space.take(result, null, defaultLease);
								TableC[result.cellID] = result.value;
							}
							// utworzenie zatrutej pigulki na sam koniec
							Task poisonPill = new Task(null, null, null, true);
							space.write(poisonPill, null, defaultLease);
						}
						catch (Exception ex) {
						}
					}
				
					public static void main(String[] args) {
						// utworzenie obiektu
						Supervisor sv = new Supervisor();
						// utworzenie zadan
						sv.generateData();
						sv.startProducing();
					}
				}
			\end{lstlisting}

	\subsection{Zadanie 1}
		\subsubsection{Treść}
			Napisać program odbierający z przestrzeni JavaSpace kolejno 100 obiektów klasy Zadanie posiadające w atrybucie typ (typu całkowitego) wartość 15 i dla każdego obiektu Zadanie wygenerować i umieścić w przestrzeni JavaSpace obiekt klasy Silnia posiadający jako atrybut... (dalej nie pamiętam dobrze) wartość będącą silnią wartości uzyskanej z liczba(typu całkowitego) z klasy Zadanie.
	
	\newpage
	\subsection{2010, I termin, Adam Duszeńko}
		\subsubsection{Treść}
			Napisać kod programu głównego zarządzającego równoległym wykonywaniem zadania w maszynie JavaSpace polegającym na wyznaczeniu zbioru klatek video zawierających ruch. Wykrywanie ruchu ma odbywać się w procesorach wykonawczych na zasadzie porównania różnicowego, czyli wymaga poddania analizie dwóch kolejnych klatek. W tym celu program główny posługując się \textit{byte[] GetImage()} (przyjąć, że jest zdefiniowana i zaimplementowana) ma pobierać kolejne klatki obrazu i umieszczać je w przestrzeni JavaSpace, wraz z jej kolejnym numerem (numerowania ma odbywać się na poziomie programu głównego). Program główny kończy wysyłania zadań gdy funkcja \textit{GetImage} zwróci wartość \textit{NULL}. Jako wynik swojego działania programy wykonawcze zwracają obiekt odpowiedzi zawierający numer pierwszego obrazu z analizowanej pary oraz wartość logiczną czy para była identyczna czy też zawierała wykryty ruch. Na zakończenie działania program główny po zebraniu wszystkich odpowiedzi powinien wypisać numery obrazów dla których wykryto ruch oraz zakończyć procesy wykonawcze rozsyłając "zatrutą pigułkę". Zaproponować strukturę obiektu zadania i odpowiedzi.
		\subsubsection{Propozycja rozwiązania 1}
			Nie jest do końca prawidłowa, ponieważ kod nie jest spójny i nie wiadomo czy analizuje pary klatek.
			\begin{lstlisting}[language=Java]
				public class Image implements Entry {
					//należy pamiętać o tym aby każde pole było publiczne!
					public byte[] frame;
					public Integer id;
					//wymagane konstruktory
					public Image() {}
					public Image(Integer id, byte[] frame) {
						this.id = new Integer(id);
						this.frame = frame;
					}
				}
			\end{lstlisting}
			\begin{lstlisting}[language=Java]
				//Dane przesyłane jako odpowiedź
				public class Result implements Entry {
					public Integer id;
					public Boolean move;
					public Result() {}
					public Result(Integer id, Boolean move) {
						this.id = new Integer(id);
						this.move = new Boolean(move);
					}
				}
			\end{lstlisting}
			\newpage
			\begin{lstlisting}[language=Java]
				public class Program {
					public int defaultLease = 100000;
					public int id = 1;
					
					public void producer() {
						byte [] img1, img2;
						img1 = getImage();
						try {
							Lookup lookup = new Lookup(JavaSpace.class);
							JavaSpace space = (JavaSpace) lookup.getService();
							img2 = getImage();
							while(true) {
								img = getImage();	// dostarczone w zadaniu
								if (img == null || img2 == null) break;	// null = koniec
								Package data = new Data(id, img1, id + 1, img2);
								space.write(data, null, defaultLease);	// paczka do space
								img1 = img2;
								id++;
							}
						// bo breaku przełączamy się w tryb odbierania
						for (int i = 1; i < id; i++) {
							Result result = (Result) space.takeIfExsists(new Result(), null, defaultLease);
							if (result.move)
								System.out.println("Ruch obrazków: " + result.id1 + " " + result.id2);
						}
						space.write(new Image(), null, defaultLease);
						} catch (Exception e) {}
					}
					
					public void consumer() {
						try {
							Lookup lookup = new Lookup(JavaSpace.class);
							JavaSpace space = (JavaSpace) lookup.getService();
							int i = 0;
							while(true) {
								Image img1 = new Image();	img1.id = i++;
								Image img2 = new Image(); 	img2.id = i;
								img1 = (Image) space.take(img1, null, defaultLese);
								img2 = (Image) space.read(img2, null, defaultLese);
								//czy wysłano "zatrutą pigułkę"
								if (img2.frame == null && img2.id == null) break;
								// czy wykonano ruch na obrazkach
								if (img1.frame.equals(img2.frame)) {
									// tego chyba nie trzeba nawet wysyłać w tym zadaniu
									result = new Result(img2.id, false);
								} else {
									result = new Result(img2.id, true);
								}
								// wysyłanie wyniku do space
								space.write(result, null, defaultLease);
							}
						} catch (Exception e) {}
					}
				}
			\end{lstlisting}
	\newpage		
	\subsection{2011, I termin, Adam Duszeńko}
		\subsubsection{Treść}
			Napisać program umieszczający w przestrzeni JavaSpace \textbf{10} obiektów zadań zawierających \textbf{dwa} pola typu całkowitego oraz \textbf{dwa} pola typu łańcuch znakowy (zawartość nieistotna, różna od NULL), podać deklarację klasy zadań. Następnie odebrać z przestrzeni kolejno \textbf{10} obiektów klasy \textit{Odpowiedź} o atrybutach \textit{id} typu \textit{Integer} oraz \textit{wynik} typu \textit{Integer} posiadające w atrybucie id wartość \textbf{15}, a następnie wszystkie z atrybutem \textit{id} = \textbf{110}. Przyjąć, że klasa \textit{Odpowiedź} jest już zdefiniowana zgodnie z powyższym opisem.
	
	\subsection{2012, I termin, Adam Duszeńko}
		\subsubsection{Treść}
			Napisać program umieszczający w przestrzeni JavaSpace \textbf{1000} obiektów zadań zawierających \textbf{trzy} pola typu całkowitego oraz \textbf{dwa} pola typu łańcuch znakowy (zawartość nieistotna, różna od NULL), podać deklarację klasy zadań. Następnie odebrać z przestrzeni kolejno \textbf{1000} obiektów klasy \textit{Odpowiedź} o atrybutach \textit{id} typu \textit{Integer} oraz \textit{wynik} typu \textit{Integer} posiadające w atrybucie id wartość \textbf{35}, a następnie wszystkie z atrybutem \textit{id} = \textbf{10}. Przyjąć, że klasa \textit{Odpowiedź} jest już zdefiniowana zgodnie z powyższym opisem.
	
	\newpage
	\subsection{2013, I termin}
		\subsubsection{Treść}
			Napisać program umieszczający w przestrzeni JavaSpace \textbf{200} obiektów zadań zawierających \textbf{dwa} pola typu całkowitego oraz \textbf{dwa} pola typu łańcuch znakowy (zawartość nieistotna, różna od NULL), podać deklarację klasy zadań. Następnie odebrać z przestrzeni kolejno \textbf{100} obiektów klasy \textit{Odpowiedź} o atrybutach \textit{id} typu \textit{Integer} oraz \textit{wynik} typu \textit{Integer} posiadające w atrybucie id wartość \textbf{35}, a następnie wszystkie z atrybutem \textit{id} = \textbf{10}. Przyjąć, że klasa \textit{Odpowiedź} jest już zdefiniowana zgodnie z powyższym opisem.
	
		\subsubsection{Rozwiązanie}
			\textbf{Klasa Zadanie}
			\begin{lstlisting}[language=Java]
			// deklaracja klasy, muszą być widoczne:
			// implementacja interfejsu Entry
			public class Zadanie implements Entry {
				// publiczne składowe, opakowujące typy zmiennych
				public Integer liczba;
				public String napis1;
				public String napis2;
				public Boolean poisonPill;
				// konstruktor domyślny, obowiązkowy
				public Zadanie() {
					Random rand = new Random();
					this.liczba = rand.nextInt();
					this.napis1 = Integer.toString(rand.nextInt());
					this.napis2 = Integer.toString(rand.nextInt());
					this.poisonPill = false;
				}
				public Zadanie(Integer liczba, String napis1, String napis2, Boolean poisonPill) {
					this.liczba = liczba;
					this.napis1 = napis1;
					this.napis2 = napis2;
					this.poisonPill = poisonPill;
				}
			}
			\end{lstlisting}
			\textbf{Nadrzędna klasa Klienta}
			\begin{lstlisting}[language=Java]
			/**
			 * @author Son Mati & Doxus
			 */
			public class Client {
				protected Integer defaultLease = 100000;
				protected JavaSpace space;
				protected Lookup lookup;
				public Client() {
					lookup = new Lookup(JavaSpace.class);
				}
			}
			\end{lstlisting}
			\newpage
			\textbf{Klasa Nadzorcy}
			\begin{lstlisting}[language=Java]
			/**
			* @author Son Mati & Doxus
			*/
			public class Boss extends Client {
				// domyślne wartości dla zadania
				static final int DEFAULT_TASK_NUMBER = 200;
				static final int DEFAULT_MAX_MISSES = 100;
				
				Integer taskNumber;
				Integer maxMisses;
				
				public Integer getTaskNumber() {
					return taskNumber;
				}
				public Integer getMaxMisses() {
					return maxMisses;
				}
				// obowiązkowy domyślny konstruktor
				public Boss() {
					taskNumber = DEFAULT_TASK_NUMBER;
					maxMisses = DEFAULT_MAX_MISSES;
				}
				public Boss(Integer taskNumber, Integer maxMisses) {
					this.taskNumber = taskNumber;
					this.maxMisses = maxMisses;
				}
				/**
				 * Wygenerowanie zadania z losowymi wartościami
				 * @param id identyfikator
				 * @param poisonPill pigułka, tak czy nie
				 */
				public Zadanie generateTask(int id, boolean poisonPill) {
					Random rand = new Random();
					return new Zadanie(id, Integer.toString(rand.nextInt(1000)),
					Integer.toString(rand.nextInt(1000)), poisonPill);
				}
				/**
				 * Utworzenie zadaniów
				 * @param count ilość zadaniów
				 */
				public void createTasksInJavaSpace(int count) {
					try {
						this.space = (JavaSpace)lookup.getService();
						for (int i = 0; i < count; ++i) {
							Zadanie zad = this.generateTask(i, false);
							space.write(zad, null, defaultLease);
							System.out.println("Wygenerowałem zad " + i + " o stringach " + zad.napis1 + " i " + zad.napis2);
						}
					}
					catch(RemoteException | TransactionException ex) {
						System.out.println("Dupa XD");
					}
				}
			\end{lstlisting}
			\newpage
			\begin{lstlisting}[language=Java]
			public class Boss extends Client {
				/**
				* Uzyskanie odpowiedzi
				* @param id odpowiedzi, ktora nas interesuje
				* @param costam interesujący nas wynik
				* @param count liczba odpowiedzi do odbioru
				*/
				public Integer receiveData(Integer id, Integer costam, Integer count) {
					Integer found = 0;
					try {
						this.space = (JavaSpace)lookup.getService();
						Odpowiedz wzor = new Odpowiedz(id, costam);
						for (int i = 0; i < count; i++) {
							// odczyt blokujący, zatrzymuje przepływ dopóki odp się nie pojawi
							Odpowiedz wynik = (Odpowiedz) space.takeIfExists(wzor, null, defaultLease);
							if (wynik != null) {
								System.out.println("Odpowiedz: id = " + wynik.getId() + ", wynik = " + wynik.getWynik());
								found++;
							}
						}
					} catch (UnusableEntryException | TransactionException | InterruptedException | RemoteException ex) {
					}
					return found;
				}
				/**
				* ZATRUJ DZIECIACZKI XD
				*/
				public void poisonKids() {
					// utworzenie zatrutej pigulki
					Zadanie poisonPill = new Zadanie(null, null, null, true);
					try {
						space.write(poisonPill, null, defaultLease);
					} catch (TransactionException | RemoteException ex) {
					}
				}
				
				public static void main(String[] args) {
					Integer misses = 0;
					Boss boss = new Boss();
					boss.createTasksInJavaSpace(boss.getTaskNumber());
					boss.receiveData(35, null, 100);
					// boss odbiera pozostałe odpowiedzi, o id 10, dopóki nie trafi na pewną liczbę chybień
					while(misses < boss.getMaxMisses()) {
						if (boss.receiveData(10, null, 1) == 0)
						misses++;
					}
					boss.poisonKids();
				}
			}
			\end{lstlisting}
			\newpage
			\textbf{Klasa Pracownika}
			\begin{lstlisting}[language=Java]
			/**
			* @author Son Mati & Doxus
			*/
			public class Sidekick extends Client {
				// obowiązkowy domyślny konstruktor
				public Sidekick() {
				}
				// praca
				public void zacznijMurzynic() {
					while(true) {
						try {
							Random rand = new Random();
							this.space = (JavaSpace)lookup.getService();
							Zadanie zad = new Zadanie(null, null, null, null);
							zad = (Zadanie) space.takeIfExists(zad, null, defaultLease);
							if (zad != null) {
								if (zad.poisonPill == true) {
									space.write(zad, null, defaultLease);
									return;
								}
								System.out.println("Odebrałem zadanie o id " + zad.liczba
								+ " i napisach " + zad.napis1 + " i " + zad.napis2);
							}
							Odpowiedz odp = new Odpowiedz(rand.nextInt(51), rand.nextInt(1000));
							space.write(odp, null, defaultLease);
						} catch (TransactionException | RemoteException | UnusableEntryException | InterruptedException ex) {
							Logger.getLogger(Sidekick.class.getName()).log(Level.SEVERE, null, ex);
						}
					}
				}
				// obowiązkowy Run
				public static void main(String[] args) {
					Sidekick murzyn = new Sidekick();
					murzyn.zacznijMurzynic();
				}
			}
			\end{lstlisting}
	\newpage
	\subsection{2014, I termin, Adam Duszeńko}
		\subsubsection{Treść}
			Napisać kod programu głównego wykonawczego do przetwarzania z wykorzystaniem maszyny JavaSpace przetwarzającego obiekty zadań zawierające dwie wartości całkowite, oraz numer obiektu i flagę logiczną początkowo zawierającą wartość \textit{FALSE}. W momencie pobrania obiektu zadania program wykonawczy ma podmienić w przestrzeni JavaSpace pobrany obiekt na ten sam, ale z flagą ustawioną na wartość \textit{TRUE}. Przetwarzanie obiektu realizowane jest w funkcji \textit{int check(int, int)} do której należy przekazać wartości z obiektu zadania. PO skończeniu przetwarzania zadania, przed zwróceniem wyniku, należy usunąć z przestrzeni JavaSpace obiekt przetwarzanego zadania. Wynik funkcji \textit{check} należy umieścić w obiekcie wynikowym którego strukturę proszę zaproponować. Obsłużyć koniec działania programu przez skonsumowanie "zatrutej pigułki".
		\subsubsection{Propozycja rozwiązania 1}
	\newpage
	\subsection{2015, 0 termin, Adam Duszeńko}
		\subsubsection{Treść}
			Napisać program umieszczający w przestrzeni JavaSpace \textbf{1000} obiektów zadań zawierających \textbf{dwa} pola typu całkowitego oraz \textbf{dwa} pola typu łańcuch znakowy (zawartość nieistotna, różna od NULL), podać deklarację klasy zadań. Następnie odebrać z przestrzeni \textbf{20} obiektów klasy \textit{Odpowiedź} o atrybutach \textit{id} typu \textit{Integer} oraz \textit{wynik} typu \textit{Integer} posiadające w atrybucie id wartość \textbf{50} (przyjąć, że klasa \textit{Odpowiedź} jest już zdefiniowana zgodnie z powyższym opisem).
		\subsubsection{Rozwiązanie}
			Działające i przetestowane w warunkach domowych na Jini.\\\\
			\textbf{Klasa Zadanie}
			\begin{lstlisting}[language=Java]
			/**
			* @author Son Mati & Doxus
			*/
			// deklaracja klasy, muszą być widoczne:
			// implementacja interfejsu Entry
			public class Zadanie implements Entry {
				// publiczne składowe, muszą być wielkich typów opakowujących
				public Integer liczba;
				public String napis1;
				public String napis2;
				public Boolean poisonPill;
				// konstruktor domyślny, wymagany
				public Zadanie() {
					Random rand = new Random();
					this.liczba = rand.nextInt();
					this.napis1 = Integer.toString(rand.nextInt());
					this.napis2 = Integer.toString(rand.nextInt());
					this.poisonPill = false;
				}
				// konstruktor z parametrami
				public Zadanie(Integer liczba, String napis1, String napis2, Boolean poisonPill) {
					this.liczba = liczba;
					this.napis1 = napis1;
					this.napis2 = napis2;
					this.poisonPill = poisonPill;
				}
			}
			\end{lstlisting}
			\newpage
			\textbf{Klasa nadzorcy}
			\begin{lstlisting}[language=Java]
					/**
					 * @author Son Mati & Doxus
					 */
					public class Boss extends Client {
						// liczba zadań do wykonania
						static final int TASK_NUMBER = 1000;
						// obowiązkowy domyślny konstruktor
						public Boss() {
						}
						/**
						 * Wygenerowanie zadania z losowymi wartościami
						 * @param count ilość zadaniów
						 */
						public Zadanie generateTask(int id, boolean poisonPill) {
							Random rand = new Random();
							return new Zadanie(id, Integer.toString(rand.nextInt(1000)), Integer.toString(rand.nextInt(1000)), poisonPill);
						}
						/**
						 * Utworzenie zadaniów
						 * @param count ilość zadaniów
						 */
						public void createTasksInJavaSpace(int count) {
							try {
								this.space = (JavaSpace)lookup.getService();
								for (int i = 0; i < count; ++i) {
									Zadanie zad = this.generateTask(i, false);
									space.write(zad, null, defaultLease);
									System.out.println("Wtgenerowałem zad " + i + " o stringach " + zad.napis1 + " i " + zad.napis2);
								}
							}
							catch(RemoteException | TransactionException ex) {
								System.out.println("Dupa XD");
							}
						}
			\end{lstlisting}
			\newpage
			\begin{lstlisting}[language=Java]
					public class Boss extends Client {
						/**
						* Uzyskanie odpowiedzi
						* @param id odpowiedzi, ktora nas interesuje
						* @param costam interesujący nas wynik
						* @param count liczba odpowiedzi do odbioru
						*/
						public void receiveData(Integer id, Integer costam, Integer count) {
							try {
								this.space = (JavaSpace)lookup.getService();
								Odpowiedz wzor = new Odpowiedz(id, costam);
								for (int i = 0; i < count; i++) {
									// odczyt blokujący, zatrzymuje przepływ dopóki odp się nie pojawi
									Odpowiedz wynik = (Odpowiedz) space.take(wzor, null, defaultLease);
									System.out.println("Odpowiedz: id = " + wynik.getId() + ", wynik = " + wynik.getWynik());
								}
							} catch (UnusableEntryException | TransactionException | InterruptedException | RemoteException ex) {
								Logger.getLogger(Boss.class.getName()).log(Level.SEVERE, null, ex);
							}
						}
						/**
						 * ZATRUJ DZIECIACZKI XD
						 */
						public void poisonKids() {
							// utworzenie zatrutej pigulki
							Zadanie poisonPill = new Zadanie(null, null, null, true);
							try {
								space.write(poisonPill, null, defaultLease);
							} catch (TransactionException | RemoteException ex) {
								Logger.getLogger(Boss.class.getName()).log(Level.SEVERE, null, ex);
							}
						}
						/**
						 * Obowiązkowy Run dla nadzorcy
						 */
						public static void main(String[] args) {
							Boss boss = new Boss();
							boss.createTasksInJavaSpace(TASK_NUMBER);
							boss.receiveData(50, null, 20);
							boss.poisonKids();
						}
					}
			\end{lstlisting}
			\newpage
			\textbf{Klasa Pracownika}
			\begin{lstlisting}[language=Java]
				/**
				 * @author Son Mati & Doxus
				 */
				public class Sidekick extends Client {
					// obowiązkowy domyślny konstruktor
					public Sidekick() {
					}
					// rozpoczęcie pracy
					public void zacznijMurzynic() {
						while(true) {
							try {
								Random rand = new Random();
								this.space = (JavaSpace)lookup.getService();
								Zadanie zad = new Zadanie(null, null, null, null);
								zad = (Zadanie) space.takeIfExists(zad, null, defaultLease);
								if (zad != null) {
									if (zad.poisonPill == true) {
										space.write(zad, null, defaultLease);
										return;
									}
									System.out.println("Odebrałem zadanie o id " + zad.liczba
									+ " i napisach " + zad.napis1 + " i " + zad.napis2);
								}
								Odpowiedz odp = new Odpowiedz(rand.nextInt(51), rand.nextInt(1000));
								space.write(odp, null, defaultLease);
							} catch (TransactionException | RemoteException | UnusableEntryException | InterruptedException ex) {
								Logger.getLogger(Sidekick.class.getName()).log(Level.SEVERE, null, ex);
							}
						}
					}
					// obowiązkowy punkt wejścia
					public static void main(String[] args) {
						Sidekick murzyn = new Sidekick();
						murzyn.zacznijMurzynic();
					}
				}
			\end{lstlisting}
	
	
	
	
	
	
	
	
	
	
	
	
	
	
	
	
	% !TeX spellcheck = pl_PL
\newpage
\section{CUDA}
	% !TeX spellcheck = pl_PL
\newpage
\section{MOSIX}
	\subsection{2015, 0 termin, Daniel Kostrzewa}
		\subsubsection{Treść}
			Napisać program, który utworzy \textit{n} procesów potomnych. Proces zarządzający ma wysyłać zestaw liczb do procesów potomnych (założyć, że liczba przesyłanych danych wynosi \textit{k}).
			Procesy potomne mają w pętli wykonywać następujące czynności: czekać na liczbę wysłaną przez proces zarządzający, na podstawie odebranej liczby obliczyć pole koła (odebrana liczba jest promieniem koła), odesłać wynik do procesu zarządzającego.
			Proces zarządzający po wysłaniu wszystkich liczb przechodzi w stan odbierania danych i sumuje pola kół. Końcowa wartość ma zostać wyświetlona na ekranie.
		\subsubsection{Rozwiązanie}
			\begin{lstlisting}[language=C]
			#include <stdio.h>
			#include <stdlib.h>
			#include <math.h>
			
			#define PI 3.14159
			
			float calculate_circle_area(int radius)
			{
			return PI * pow((float) radius, 2);
			}
			
			int main(int argc, char* argv[])
			{
			int n = atoi(argv[1]);
			int k = atoi(argv[2]);
			
			int poison = -1;		//promien -1 oznacza ze jest to trujaca pigulka zabijajaca proces potomny
			
			int * tab = (int*) malloc(sizeof(int) * k);	//allokacja tablicy k liczb (promieni)
			int i;
			for (i = 0; i < k; i++)
			{
			tab[i] = rand() % 20;
			}
			
			int process_number = n / k;
			int data_stream[2], response_stream[2];		//2 strumienie, jeden do wysyłania danych, drugi do odbioru odpowiedzi
			
			// wyniki
			int part;
			int sum = 0;
			
			// jebnięcie potoków
			// odpowiedzi z procesów potomnych
			pipe(response_stream);
			// in - promienie kół
			// out - otrzymane wyniki - pola kół
			pipe(data_stream);
			
			// utworzenie procesów potomnych
			for (i = 0; i < n; i++)
			{
			if (fork() == 0)
			{
			int radius;
			// wykonywanie obliczeń dopóki nie zostanie OTRUTY(!)
			while(true)
			{
			if (read(data_stream[0], &radius, sizeof(int)) != sizeof(int))
			continue;
			if (radius == -1)	// wyłącza się tylko jak otrzymamy pigułkę
			exit(0);
			
			float result = calculate_circle_area(radius);
			write(response_stream[1], &result, sizeof(float));
			}
			}
			}
			// wysłanie danych do procesów potomnych
			for (i = 0; i < k; i++)
			{
			write(data_stream[1], &tab[i], sizeof(int));
			}
			// Halo odbjoor danych
			for (i = 0; i < n; i++)
			{
			read(response_stream[0], &part, sizeof(int));
			sum = sum + part;
			}
			// ZABIJANIE DZIECI
			for (i = 0; i < k; i++)
			{
			write(data_stream[1], &poison, sizeof(int));
			}
			// wypisanie odpowiedzi
			printf("Suma pól kół: %d", sum);
			
			return 0;
			}
			
			\end{lstlisting}
		\newpage
\end{document}