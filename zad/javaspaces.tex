% !TeX spellcheck = pl_PL
\newpage
\section{Java Spaces}
	\subsection{Laborki}
		\subsubsection{Treść}
			Napisać program zawierający jednego Nadzorcę oraz wielu Pracowników. Nadzorca przekazuje do JavaSpace 2 równe tablice zawierające obiekty typu Integer, a następnie otrzymuje wynikową tablicę zawierającą sumy odpowiadających sobie komórek. Operację dodawania mają realizować Pracownicy.
		\subsubsection{Rozwiązanie}
			Zadanie obliczania sumy tabel dzielimy na dwie części: \textit{Task} oraz \textit{Result}. \textit{Taski} są generowane przez \textit{Nadzorcę} i przekazywane \textit{Pracownikom}, ci zaś wykonują zadanie i tworzą obiekty klasy \textit{Result}, a następnie przekazuję je \textit{Nadzorcy}. \textit{Nadzorca} je odbiera, kompletuje i ew. coś z nimi robi.\\\\
			\textit{Nadzorca} przydziela tyle zadań, ile potrzebuje, z kolei \textit{Pracownicy} działają w nieskończoność. Aby zakończyć ich pracę, \textit{Nadzorca} musi wysłać zadania z tzw. zatrutą pigułką (ang. \emph{Poisoned Pill}), czyli obiekt zadania z nietypowym parametrem, który sygnalizuje zakończenie pracy. Może to być np. \textit{Boolean} o wartości \textit{false}, \textit{Integer} o wartości -1, itp.
			Składowymi klas implementujących interfejs Entry nie mogą być typu prostego (int, double itp.), muszą być opakowane (Integer, Double itp.). Najbezpieczniej dawać je wszędzie.\\
			\begin{lstlisting}[language=Java]
				/**
				* @author Son Mati
				* @waifu Itsuka Kotori
				*/
				public class Task implements Entry {
					public Integer cellID;	// ID komórki tabeli
					public Integer valueA;	// wartość z tabeli A
					public Integer valueB;	// wartość z tabeli B
					public Boolean isPill;	// czy zadanie jest zatrutą pigułką

					// Domyślny konstruktor, musi się znajdować
					public Task() {
						this.cellID = this.valueA = this.valueB = null;
						this.isPill = false;
					}
					
					public Task(Integer entryID, Integer valueA, Integer valueB, Boolean isPill) {
						this.cellID = entryID;
						this.valueA = valueA;
						this.valueB = valueB;
						this.isPill = isPill;
					}
				}
			\end{lstlisting}
			\newpage
			\begin{lstlisting}[language=Java]
				/**
				* @author Son Mati
				* @waifu Itsuka Kotori
				*/
				public class Result implements Entry {
					public Integer cellID, value;
					public Result() {
						this.cellID = this.value = null;
					}
					public Result(final Integer EntryID, final Integer Value) {
						this.cellID = EntryID;
						this.value = Value;
					}
				}
			\end{lstlisting}
			\begin{lstlisting}[language=Java]
				public class Client {
					protected Integer defaultLease = 100000;
					protected JavaSpace space;
					protected Lookup lookup;
					public Client() {
						lookup = new Lookup(JavaSpace.class);
					}
				}
			\end{lstlisting}
			\begin{lstlisting}[language=Java]
				/**
				* @author Son Mati
				* @waifu Itsuka Kotori
				*/
				public class Worker extends Client {
					public Worker() {
					}
					public void startWorking() {
						while(true) {
							try {
								this.space = (JavaSpace)lookup.getService();
								Task task = new Task();
								task = (Task) space.take(task, null, defaultLease);
								if (task.isPill == true)
								{
									space.write(task, null, defaultLease);
									System.out.println("Koniec pracy workera.");
									return;
								}
								Integer res = task.valueA + task.valueB;
								Result result = new Result(task.cellID, res);
								space.write(result, null, defaultLease);
							}
							catch (Exception ex) {}
						}
					}
					public static void main(String[] args) {
						Worker w = new Worker();	// utworzenie obiektu
						w.startWorking();			// realizacja zadan
					}
				}
			\end{lstlisting}
			\newpage
			\begin{lstlisting}[language=Java]
				/**
				* @author Son Mati
				* @waifu Itsuka Kotori
				*/
				public class Supervisor extends Client {
					static final Integer INT_NUMBER = 125;
					public Integer[] TableA = new Integer[INT_NUMBER];
					public Integer[] TableB = new Integer[INT_NUMBER];
					public Integer[] TableC = new Integer[INT_NUMBER];
					// konstruktor
					public Supervisor() {
					}
					// wygenerowanie zawartości tablic
					public void generateData() {
						Random rand = new Random();
						for (int i = 0; i < INT_NUMBER; ++i) {
							TableA[i] = rand.nextInt(INT_NUMBER);
							TableB[i] = rand.nextInt(INT_NUMBER);
							TableC[i] = 0;
						}
					}
					// rozpoczecie pracy
					public void startProducing() {
						try {
							this.space = (JavaSpace)lookup.getService();
							// utworzenie zadania
							for (Integer i = 0; i < INT_NUMBER; ++i) {
								Task task = new Task(i, this.TableA[i], this.TableB[i], false);
								space.write(task, null, defaultLease);
							}
							// pobranie wyniku zadania
							System.out.println("Tablica C:");
							for(Integer i = 0 ; i < INT_NUMBER; ++i) {
								Result result = new Result();
								result = (Result) space.take(result, null, defaultLease);
								TableC[result.cellID] = result.value;
							}
							// utworzenie zatrutej pigulki na sam koniec
							Task poisonPill = new Task(null, null, null, true);
							space.write(poisonPill, null, defaultLease);
						}
						catch (Exception ex) {
						}
					}
				
					public static void main(String[] args) {
						// utworzenie obiektu
						Supervisor sv = new Supervisor();
						// utworzenie zadan
						sv.generateData();
						sv.startProducing();
					}
				}
			\end{lstlisting}
	\newpage
	\subsection{Treść}
			Program umieszczający w przestrzeni JavaSpace \textbf{200} obiektów zadań zawierających \textbf{dwa} pola typu całkowitego oraz \textbf{dwa} pola typu łańcuch znakowy (zawartość nieistotna, różna od NULL), podać deklarację klasy zadań. Następnie odebrać z przestrzeni kolejno \textbf{100} obiektów klasy \textit{Odpowiedź} o atrybutach \textit{id} typu \textit{Integer} oraz \textit{wynik} typu \textit{Integer} posiadające w atrybucie id wartość \textbf{35}, a następnie wszystkie z atrybutem \textit{id} = \textbf{10}. Przyjąć, że klasa \textit{Odpowiedź} jest już zdefiniowana zgodnie z powyższym opisem.
	
	\subsubsection{Rozwiązanie}
			\textbf{Klasa Zadanie}
			\begin{lstlisting}[language=Java]
			// deklaracja klasy, muszą być widoczne:
			// implementacja interfejsu Entry
			public class Zadanie implements Entry {
				// publiczne składowe, opakowujące typy zmiennych
				public Integer liczba;
				public String napis1;
				public String napis2;
				public Boolean poisonPill;
				// konstruktor domyślny, obowiązkowy
				public Zadanie() {
					Random rand = new Random();
					this.liczba = rand.nextInt();
					this.napis1 = Integer.toString(rand.nextInt());
					this.napis2 = Integer.toString(rand.nextInt());
					this.poisonPill = false;
				}
				public Zadanie(Integer liczba, String napis1, String napis2, Boolean poisonPill) {
					this.liczba = liczba;
					this.napis1 = napis1;
					this.napis2 = napis2;
					this.poisonPill = poisonPill;
				}
			}
			\end{lstlisting}
			\textbf{Nadrzędna klasa Klienta}
			\begin{lstlisting}[language=Java]
			/**
			 * @author Son Mati & Doxus
			 */
			public class Client {
				protected Integer defaultLease = 100000;
				protected JavaSpace space;
				protected Lookup lookup;
				public Client() {
					lookup = new Lookup(JavaSpace.class);
				}
			}
			\end{lstlisting}
			\newpage
			\textbf{Klasa Nadzorcy}
			\begin{lstlisting}[language=Java]
			/**
			* @author Son Mati & Doxus
			*/
			public class Boss extends Client {
				// domyślne wartości dla zadania
				static final int DEFAULT_TASK_NUMBER = 200;
				static final int DEFAULT_MAX_MISSES = 100;
				
				Integer taskNumber;
				Integer maxMisses;
				
				public Integer getTaskNumber() {
					return taskNumber;
				}
				public Integer getMaxMisses() {
					return maxMisses;
				}
				// obowiązkowy domyślny konstruktor
				public Boss() {
					taskNumber = DEFAULT_TASK_NUMBER;
					maxMisses = DEFAULT_MAX_MISSES;
				}
				public Boss(Integer taskNumber, Integer maxMisses) {
					this.taskNumber = taskNumber;
					this.maxMisses = maxMisses;
				}
				/**
				 * Wygenerowanie zadania z losowymi wartościami
				 * @param id identyfikator
				 * @param poisonPill pigułka, tak czy nie
				 */
				public Zadanie generateTask(int id, boolean poisonPill) {
					Random rand = new Random();
					return new Zadanie(id, Integer.toString(rand.nextInt(1000)),
					Integer.toString(rand.nextInt(1000)), poisonPill);
				}
				/**
				 * Utworzenie zadaniów
				 * @param count ilość zadaniów
				 */
				public void createTasksInJavaSpace(int count) {
					try {
						this.space = (JavaSpace)lookup.getService();
						for (int i = 0; i < count; ++i) {
							Zadanie zad = this.generateTask(i, false);
							space.write(zad, null, defaultLease);
							System.out.println("Wygenerowałem zad " + i + " o stringach " + zad.napis1 + " i " + zad.napis2);
						}
					}
					catch(RemoteException | TransactionException ex) {
						System.out.println("Dupa XD");
					}
				}
			\end{lstlisting}
			\newpage
			\begin{lstlisting}[language=Java]
			public class Boss extends Client {
				/**
				* Uzyskanie odpowiedzi
				* @param id odpowiedzi, ktora nas interesuje
				* @param costam interesujący nas wynik
				* @param count liczba odpowiedzi do odbioru
				*/
				public Integer receiveData(Integer id, Integer costam, Integer count) {
					Integer found = 0;
					try {
						this.space = (JavaSpace)lookup.getService();
						Odpowiedz wzor = new Odpowiedz(id, costam);
						for (int i = 0; i < count; i++) {
							// odczyt blokujący, zatrzymuje przepływ dopóki odp się nie pojawi
							Odpowiedz wynik = (Odpowiedz) space.takeIfExists(wzor, null, defaultLease);
							if (wynik != null) {
								System.out.println("Odpowiedz: id = " + wynik.getId() + ", wynik = " + wynik.getWynik());
								found++;
							}
						}
					} catch (UnusableEntryException | TransactionException | InterruptedException | RemoteException ex) {
					}
					return found;
				}
				/**
				* ZATRUJ DZIECIACZKI XD
				*/
				public void poisonKids() {
					// utworzenie zatrutej pigulki
					Zadanie poisonPill = new Zadanie(null, null, null, true);
					try {
						space.write(poisonPill, null, defaultLease);
					} catch (TransactionException | RemoteException ex) {
					}
				}
				
				public static void main(String[] args) {
					Integer misses = 0;
					Boss boss = new Boss();
					boss.createTasksInJavaSpace(boss.getTaskNumber());
					boss.receiveData(35, null, 100);
					// boss odbiera pozostałe odpowiedzi, o id 10, dopóki nie trafi na pewną liczbę chybień
					while(misses < boss.getMaxMisses()) {
						if (boss.receiveData(10, null, 1) == 0)
						misses++;
					}
					boss.poisonKids();
				}
			}
			\end{lstlisting}
			\newpage
			\textbf{Klasa Pracownika}
			\begin{lstlisting}[language=Java]
			/**
			* @author Son Mati & Doxus
			*/
			public class Sidekick extends Client {
				// obowiązkowy domyślny konstruktor
				public Sidekick() {
				}
				// praca
				public void zacznijMurzynic() {
					while(true) {
						try {
							Random rand = new Random();
							this.space = (JavaSpace)lookup.getService();
							Zadanie zad = new Zadanie(null, null, null, null);
							zad = (Zadanie) space.takeIfExists(zad, null, defaultLease);
							if (zad != null) {
								if (zad.poisonPill == true) {
									space.write(zad, null, defaultLease);
									return;
								}
								System.out.println("Odebrałem zadanie o id " + zad.liczba
								+ " i napisach " + zad.napis1 + " i " + zad.napis2);
							}
							Odpowiedz odp = new Odpowiedz(rand.nextInt(51), rand.nextInt(1000));
							space.write(odp, null, defaultLease);
						} catch (TransactionException | RemoteException | UnusableEntryException | InterruptedException ex) {
							Logger.getLogger(Sidekick.class.getName()).log(Level.SEVERE, null, ex);
						}
					}
				}
				// obowiązkowy Run
				public static void main(String[] args) {
					Sidekick murzyn = new Sidekick();
					murzyn.zacznijMurzynic();
				}
			}
			\end{lstlisting}
	\newpage
	\subsection{Treść}
		Kod programu głównego wykonawczego do przetwarzania z wykorzystaniem maszyny JavaSpace przetwarzającego obiekty zadań zawierające dwie wartości całkowite, oraz numer obiektu i flagę logiczną początkowo zawierającą wartość \textit{FALSE}. W momencie pobrania obiektu zadania program wykonawczy ma podmienić w przestrzeni JavaSpace pobrany obiekt na ten sam, ale z flagą ustawioną na wartość \textit{TRUE}. Przetwarzanie obiektu realizowane jest w funkcji \textit{int check(int, int)} do której należy przekazać wartości z obiektu zadania. Po skończeniu przetwarzania zadania, przed zwróceniem wyniku, należy usunąć z przestrzeni JavaSpace obiekt przetwarzanego zadania. Wynik funkcji \textit{check} należy umieścić w obiekcie wynikowym którego strukturę proszę zaproponować. Obsłużyć koniec działania programu przez skonsumowanie "zatrutej pigułki".
		\subsubsection{Rozwiązanie}
			\textbf{Klasa Odpowiedzi (wynik Zadania)}:
			\begin{lstlisting}[language=Java]
				/**
				 * @author Son Mati
				 * @waifu Itsuka Kotori
				 */
				public class Odpowiedz implements Entry {
					public Integer id;
					public Integer wynik;
					
					public Odpowiedz() {
					}
					
					public Odpowiedz(Integer id, Integer wynik) {
						this.id = id;
						this.wynik = wynik;
					}
				}
			\end{lstlisting}
			\newpage
			\textbf{Klasa Wykonawcy}:
			\begin{lstlisting}[language=Java]
				/**
				 * @author Son Mati
				 * @waifu Itsuka Kotori
				 */
				public class Sidekick {
					protected Integer defaultLease = 100000;
					protected JavaSpace space;
					protected Lookup lookup;
					// domyślny konstruktor obowiązakowy
					public Sidekick() {
						lookup = new Lookup(JavaSpace.class);
					}
					public void zacznijMurzynic() {
						while(true) {
							try {
								this.space = (JavaSpace)lookup.getService();
								Zadanie zad = new Zadanie(null, null, null, false, null);
								zad = (Zadanie) space.take(zad, null, defaultLease);
								if (zad.poisonPill == true) {
									space.write(zad, null, defaultLease);
									return;
								}
								zad.flag = true;
								space.write(zad, null, defaultLease);
								int result = this.check(zad.liczba1, zad.liczba2);
								Odpowiedz odp = new Odpowiedz(zad.id, result);
								space.take(zad, null, defaultLease);    // pełen wzorzec
								space.write(odp, null, defaultLease);
							} catch (TransactionException | RemoteException | UnusableEntryException | InterruptedException ex) {
								Logger.getLogger(Sidekick.class.getName()).log(Level.SEVERE, null, ex);
							}
						}
					}
					// dla przykładu
					public int check(int a, int b) {
						return a + b;
					}
					// obowiązkowy Run
					public static void main(String[] args) {
						Sidekick murzyn = new Sidekick();
						murzyn.zacznijMurzynic();
					}
				}
			\end{lstlisting}
	\newpage
	\subsection{Treść}
		Program umieszczający w przestrzeni JavaSpace \textbf{1000} obiektów zadań zawierających \textbf{dwa} pola typu całkowitego oraz \textbf{dwa} pola typu łańcuch znakowy (zawartość nieistotna, różna od NULL), podać deklarację klasy zadań. Następnie odebrać z przestrzeni \textbf{20} obiektów klasy \textit{Odpowiedź} o atrybutach \textit{id} typu \textit{Integer} oraz \textit{wynik} typu \textit{Integer} posiadające w atrybucie id wartość \textbf{50} (przyjąć, że klasa \textit{Odpowiedź} jest już zdefiniowana zgodnie z powyższym opisem).
		\subsubsection{Rozwiązanie}
			Działające i przetestowane w warunkach domowych na Jini.\\\\
			\textbf{Klasa Zadanie}
			\begin{lstlisting}[language=Java]
			/**
			* @author Son Mati & Doxus
			*/
			// deklaracja klasy, muszą być widoczne:
			// implementacja interfejsu Entry
			public class Zadanie implements Entry {
				// publiczne składowe, muszą być wielkich typów opakowujących
				public Integer liczba;
				public String napis1;
				public String napis2;
				public Boolean poisonPill;
				// konstruktor domyślny, wymagany
				public Zadanie() {
					Random rand = new Random();
					this.liczba = rand.nextInt();
					this.napis1 = Integer.toString(rand.nextInt());
					this.napis2 = Integer.toString(rand.nextInt());
					this.poisonPill = false;
				}
				// konstruktor z parametrami
				public Zadanie(Integer liczba, String napis1, String napis2, Boolean poisonPill) {
					this.liczba = liczba;
					this.napis1 = napis1;
					this.napis2 = napis2;
					this.poisonPill = poisonPill;
				}
			}
			\end{lstlisting}
			\newpage
			\textbf{Klasa nadzorcy}
			\begin{lstlisting}[language=Java]
					/**
					 * @author Son Mati & Doxus
					 */
					public class Boss extends Client {
						// liczba zadań do wykonania
						static final int TASK_NUMBER = 1000;
						// obowiązkowy domyślny konstruktor
						public Boss() {
						}
						/**
						 * Wygenerowanie zadania z losowymi wartościami
						 * @param count ilość zadaniów
						 */
						public Zadanie generateTask(int id, boolean poisonPill) {
							Random rand = new Random();
							return new Zadanie(id, Integer.toString(rand.nextInt(1000)), Integer.toString(rand.nextInt(1000)), poisonPill);
						}
						/**
						 * Utworzenie zadaniów
						 * @param count ilość zadaniów
						 */
						public void createTasksInJavaSpace(int count) {
							try {
								this.space = (JavaSpace)lookup.getService();
								for (int i = 0; i < count; ++i) {
									Zadanie zad = this.generateTask(i, false);
									space.write(zad, null, defaultLease);
									System.out.println("Wtgenerowałem zad " + i + " o stringach " + zad.napis1 + " i " + zad.napis2);
								}
							}
							catch(RemoteException | TransactionException ex) {
								System.out.println("Dupa XD");
							}
						}
			\end{lstlisting}
			\newpage
			\begin{lstlisting}[language=Java]
					public class Boss extends Client {
						/**
						* Uzyskanie odpowiedzi
						* @param id odpowiedzi, ktora nas interesuje
						* @param costam interesujący nas wynik
						* @param count liczba odpowiedzi do odbioru
						*/
						public void receiveData(Integer id, Integer costam, Integer count) {
							try {
								this.space = (JavaSpace)lookup.getService();
								Odpowiedz wzor = new Odpowiedz(id, costam);
								for (int i = 0; i < count; i++) {
									// odczyt blokujący, zatrzymuje przepływ dopóki odp się nie pojawi
									Odpowiedz wynik = (Odpowiedz) space.take(wzor, null, defaultLease);
									System.out.println("Odpowiedz: id = " + wynik.getId() + ", wynik = " + wynik.getWynik());
								}
							} catch (UnusableEntryException | TransactionException | InterruptedException | RemoteException ex) {
								Logger.getLogger(Boss.class.getName()).log(Level.SEVERE, null, ex);
							}
						}
						/**
						 * ZATRUJ DZIECIACZKI XD
						 */
						public void poisonKids() {
							// utworzenie zatrutej pigulki
							Zadanie poisonPill = new Zadanie(null, null, null, true);
							try {
								space.write(poisonPill, null, defaultLease);
							} catch (TransactionException | RemoteException ex) {
								Logger.getLogger(Boss.class.getName()).log(Level.SEVERE, null, ex);
							}
						}
						/**
						 * Obowiązkowy Run dla nadzorcy
						 */
						public static void main(String[] args) {
							Boss boss = new Boss();
							boss.createTasksInJavaSpace(TASK_NUMBER);
							boss.receiveData(50, null, 20);
							boss.poisonKids();
						}
					}
			\end{lstlisting}
			\newpage
			\textbf{Klasa Pracownika}
			\begin{lstlisting}[language=Java]
				/**
				 * @author Son Mati & Doxus
				 */
				public class Sidekick extends Client {
					// obowiązkowy domyślny konstruktor
					public Sidekick() {
					}
					// rozpoczęcie pracy
					public void zacznijMurzynic() {
						while(true) {
							try {
								Random rand = new Random();
								this.space = (JavaSpace)lookup.getService();
								Zadanie zad = new Zadanie(null, null, null, null);
								zad = (Zadanie) space.takeIfExists(zad, null, defaultLease);
								if (zad != null) {
									if (zad.poisonPill == true) {
										space.write(zad, null, defaultLease);
										return;
									}
									System.out.println("Odebrałem zadanie o id " + zad.liczba
									+ " i napisach " + zad.napis1 + " i " + zad.napis2);
								}
								Odpowiedz odp = new Odpowiedz(rand.nextInt(51), rand.nextInt(1000));
								space.write(odp, null, defaultLease);
							} catch (TransactionException | RemoteException | UnusableEntryException | InterruptedException ex) {
								Logger.getLogger(Sidekick.class.getName()).log(Level.SEVERE, null, ex);
							}
						}
					}
					// obowiązkowy punkt wejścia
					public static void main(String[] args) {
						Sidekick murzyn = new Sidekick();
						murzyn.zacznijMurzynic();
					}
				}
			\end{lstlisting}
	