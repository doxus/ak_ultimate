% !TeX spellcheck = pl_PL
\newpage

\section{Java Spaces}
	\subsection{Wstęp z laborek}
		\subsubsection{Treść}
			Napisać program zawierający jednego Nadzorcę oraz wielu Pracowników. Nadzorca przekazuje do JavaSpace 2 równe tablice zawierające obiekty typu Integer, a następnie otrzymuje wynikową tablicę zawierającą sumy odpowiadających sobie komórek. Operację dodawania mają realizować Pracownicy.
		\subsubsection{Rozwiązanie}
			Zadanie obliczania sumy tabel dzielimy na dwie części: \textit{Task} oraz \textit{Result}. \textit{Taski} są generowane przez \textit{Nadzorcę} i przekazywane \textit{Pracownikom}, ci zaś wykonują zadanie i tworzą obiekty klasy \textit{Result}, a następnie przekazuję je \textit{Nadzorcy}. \textit{Nadzorca} je odbiera, kompletuje i ew. coś z nimi robi.\\\\
			\textit{Nadzorca} przydziela tyle zadań, ile potrzebuje, z kolei \textit{Pracownicy} działają w nieskończoność. Aby zakończyć ich pracę, \textit{Nadzorca} musi wysłać zadania z tzw. zatrutą pigułką (ang. \emph{Poisoned Pill}), czyli obiekt zadania z nietypowym parametrem, który sygnalizuje zakończenie pracy. Może to być np. \textit{Boolean} o wartości \textit{false}, \textit{Integer} o wartości -1, itp.
			Składowymi klas implementujących interfejs Entry nie mogą być typu prostego (int, double itp.), muszą być opakowane (Integer, Double itp.). Najbezpieczniej dawać je wszędzie.\\
			\begin{lstlisting}[language=Java]
				/**
				* @author Son Mati
				* @waifu Itsuka Kotori
				*/
				public class Task implements Entry {
					public Integer cellID;	// ID komórki tabeli
					public Integer valueA;	// wartość z tabeli A
					public Integer valueB;	// wartość z tabeli B
					public Boolean isPill;	// czy zadanie jest zatrutą pigułką

					// Domyślny konstruktor, musi się znajdować
					public Task() {
						this.cellID = this.valueA = this.valueB = null;
						this.isPill = false;
					}
					
					public Task(Integer entryID, Integer valueA, Integer valueB, Boolean isPill) {
						this.cellID = entryID;
						this.valueA = valueA;
						this.valueB = valueB;
						this.isPill = isPill;
					}
				}
			\end{lstlisting}
			\newpage
			\begin{lstlisting}[language=Java]
				/**
				* @author Son Mati
				* @waifu Itsuka Kotori
				*/
				public class Result implements Entry {
					public Integer cellID, value;
					public Result() {
						this.cellID = this.value = null;
					}
					public Result(final Integer EntryID, final Integer Value) {
						this.cellID = EntryID;
						this.value = Value;
					}
				}
			\end{lstlisting}
			\begin{lstlisting}
				public class Client {
					protected Integer defaultLease = 100000;
					protected JavaSpace space;
					protected Lookup lookup;
					public Client() {
						lookup = new Lookup(JavaSpace.class);
					}
				}
			\end{lstlisting}
			\begin{lstlisting}[language=Java]
				/**
				* @author Son Mati
				* @waifu Itsuka Kotori
				*/
				public class Worker extends Client {
					public Worker() {
					}
					public void startWorking() {
						while(true) {
							try {
								this.space = (JavaSpace)lookup.getService();
								Task task = new Task();
								task = (Task) space.take(task, null, defaultLease);
								if (task.isPill == true)
								{
									space.write(task, null, defaultLease);
									System.out.println("Koniec pracy workera.");
									return;
								}
								Integer res = task.valueA + task.valueB;
								Result result = new Result(task.cellID, res);
								space.write(result, null, defaultLease);
							}
							catch (Exception ex) {}
						}
					}
					public static void main(String[] args) {
						Worker w = new Worker();	// utworzenie obiektu
						w.startWorking();			// realizacja zadan
					}
				}
			\end{lstlisting}
			\newpage
			\begin{lstlisting}[language=Java]
				/**
				* @author Son Mati
				* @waifu Itsuka Kotori
				*/
				public class Supervisor extends Client {
					static final Integer INT_NUMBER = 125;
					public Integer[] TableA = new Integer[INT_NUMBER];
					public Integer[] TableB = new Integer[INT_NUMBER];
					public Integer[] TableC = new Integer[INT_NUMBER];
					// konstruktor
					public Supervisor() {
					}
					// wygenerowanie zawartości tablic
					public void generateData() {
						Random rand = new Random();
						for (int i = 0; i < INT_NUMBER; ++i) {
							TableA[i] = rand.nextInt(INT_NUMBER);
							TableB[i] = rand.nextInt(INT_NUMBER);
							TableC[i] = 0;
						}
					}
					// rozpoczecie pracy
					public void startProducing() {
						try {
							this.space = (JavaSpace)lookup.getService();
							// utworzenie zadania
							for (Integer i = 0; i < INT_NUMBER; ++i) {
								Task task = new Task(i, this.TableA[i], this.TableB[i], false);
								space.write(task, null, defaultLease);
							}
							// pobranie wyniku zadania
							System.out.println("Tablica C:");
							for(Integer i = 0 ; i < INT_NUMBER; ++i) {
								Result result = new Result();
								result = (Result) space.take(result, null, defaultLease);
								TableC[result.cellID] = result.value;
							}
							// utworzenie zatrutej pigulki na sam koniec
							Task poisonPill = new Task(null, null, null, true);
							space.write(poisonPill, null, defaultLease);
						}
						catch (Exception ex) {
						}
					}
				
					public static void main(String[] args) {
						// utworzenie obiektu
						Supervisor sv = new Supervisor();
						// utworzenie zadan
						sv.generateData();
						sv.startProducing();
					}
				}
			\end{lstlisting}

	\subsection{Zadanie 1}
		\subsubsection{Treść}
			Napisać program odbierający z przestrzeni JavaSpace kolejno 100 obiektów klasy Zadanie posiadające w atrybucie typ (typu całkowitego) wartość 15 i dla każdego obiektu Zadanie wygenerować i umieścić w przestrzeni JavaSpace obiekt klasy Silnia posiadający jako atrybut... (dalej nie pamiętam dobrze) wartość będącą silnią wartości uzyskanej z liczba(typu całkowitego) z klasy Zadanie.
	
	\newpage
	\subsection{2010, I termin, Adam Duszeńko}
		\subsubsection{Treść}
			Napisać kod programu głównego zarządzającego równoległym wykonywaniem zadania w maszynie JavaSpace polegającym na wyznaczeniu zbioru klatek video zawierających ruch. Wykrywanie ruchu ma odbywać się w procesorach wykonawczych na zasadzie porównania różnicowego, czyli wymaga poddania analizie dwóch kolejnych klatek. W tym celu program główny posługując się \textit{byte[] GetImage()} (przyjąć, że jest zdefiniowana i zaimplementowana) ma pobierać kolejne klatki obrazu i umieszczać je w przestrzeni JavaSpace, wraz z jej kolejnym numerem (numerowania ma odbywać się na poziomie programu głównego). Program główny kończy wysyłania zadań gdy funkcja \textit{GetImage} zwróci wartość \textit{NULL}. Jako wynik swojego działania programy wykonawcze zwracają obiekt odpowiedzi zawierający numer pierwszego obrazu z analizowanej pary oraz wartość logiczną czy para była identyczna czy też zawierała wykryty ruch. Na zakończenie działania program główny po zebraniu wszystkich odpowiedzi powinien wypisać numery obrazów dla których wykryto ruch oraz zakończyć procesy wykonawcze rozsyłając "zatrutą pigułkę". Zaproponować strukturę obiektu zadania i odpowiedzi.
		\subsubsection{Propozycja rozwiązania 1}
			Nie jest do końca prawidłowa, ponieważ kod nie jest spójny i nie wiadomo czy analizuje pary klatek.
			\begin{lstlisting}[language=Java]
				public class Image implements Entry {
					//należy pamiętać o tym aby każde pole było publiczne!
					public byte[] frame;
					public Integer id;
					//wymagane konstruktory
					public Image() {}
					public Image(Integer id, byte[] frame) {
						this.id = new Integer(id);
						this.frame = frame;
					}
				}
			\end{lstlisting}
			\begin{lstlisting}[language=Java]
				//Dane przesyłane jako odpowiedź
				public class Result implements Entry {
					public Integer id;
					public Boolean move;
					public Result() {}
					public Result(Integer id, Boolean move) {
						this.id = new Integer(id);
						this.move = new Boolean(move);
					}
				}
			\end{lstlisting}
			\newpage
			\begin{lstlisting}[language=Java]
				public class Program {
					public int defaultLease = 100000;
					public int id = 1;
					
					public void producer() {
						byte [] img1, img2;
						img1 = getImage();
						try {
							Lookup lookup = new Lookup(JavaSpace.class);
							JavaSpace space = (JavaSpace) lookup.getService();
							img2 = getImage();
							while(true) {
								img = getImage();	// dostarczone w zadaniu
								if (img == null || img2 == null) break;	// null = koniec
								Package data = new Data(id, img1, id + 1, img2);
								space.write(data, null, defaultLease);	// paczka do space
								img1 = img2;
								id++;
							}
						// bo breaku przełączamy się w tryb odbierania
						for (int i = 1; i < id; i++) {
							Result result = (Result) space.takeIfExsists(new Result(), null, defaultLease);
							if (result.move)
								System.out.println("Ruch obrazków: " + result.id1 + " " + result.id2);
						}
						space.write(new Image(), null, defaultLease);
						} catch (Exception e) {}
					}
					
					public void consumer() {
						try {
							Lookup lookup = new Lookup(JavaSpace.class);
							JavaSpace space = (JavaSpace) lookup.getService();
							int i = 0;
							while(true) {
								Image img1 = new Image();	img1.id = i++;
								Image img2 = new Image(); 	img2.id = i;
								img1 = (Image) space.take(img1, null, defaultLese);
								img2 = (Image) space.read(img2, null, defaultLese);
								//czy wysłano "zatrutą pigułkę"
								if (img2.frame == null && img2.id == null) break;
								// czy wykonano ruch na obrazkach
								if (img1.frame.equals(img2.frame)) {
									// tego chyba nie trzeba nawet wysyłać w tym zadaniu
									result = new Result(img2.id, false);
								} else {
									result = new Result(img2.id, true);
								}
								// wysyłanie wyniku do space
								space.write(result, null, defaultLease);
							}
						} catch (Exception e) {}
					}
				}
			\end{lstlisting}
	\newpage		
	\subsection{2011, I termin, Adam Duszeńko}
		\subsubsection{Treść}
			Napisać program umieszczający w przestrzeni JavaSpace \textbf{10} obiektów zadań zawierających \textbf{dwa} pola typu całkowitego oraz \textbf{dwa} pola typu łańcuch znakowy (zawartość nieistotna, różna od NULL), podać deklarację klasy zadań. Następnie odebrać z przestrzeni kolejno \textbf{10} obiektów klasy \textit{Odpowiedź} o atrybutach \textit{id} typu \textit{Integer} oraz \textit{wynik} typu \textit{Integer} posiadające w atrybucie id wartość \textbf{15}, a następnie wszystkie z atrybutem \textit{id} = \textbf{110}. Przyjąć, że klasa \textit{Odpowiedź} jest już zdefiniowana zgodnie z powyższym opisem.
	
	\subsection{2012, I termin, Adam Duszeńko}
		\subsubsection{Treść}
			Napisać program umieszczający w przestrzeni JavaSpace \textbf{1000} obiektów zadań zawierających \textbf{trzy} pola typu całkowitego oraz \textbf{dwa} pola typu łańcuch znakowy (zawartość nieistotna, różna od NULL), podać deklarację klasy zadań. Następnie odebrać z przestrzeni kolejno \textbf{1000} obiektów klasy \textit{Odpowiedź} o atrybutach \textit{id} typu \textit{Integer} oraz \textit{wynik} typu \textit{Integer} posiadające w atrybucie id wartość \textbf{35}, a następnie wszystkie z atrybutem \textit{id} = \textbf{10}. Przyjąć, że klasa \textit{Odpowiedź} jest już zdefiniowana zgodnie z powyższym opisem.
	
	\newpage
	\subsection{2013, I termin}
		\subsubsection{Treść}
			Napisać program umieszczający w przestrzeni JavaSpace \textbf{200} obiektów zadań zawierających \textbf{dwa} pola typu całkowitego oraz \textbf{dwa} pola typu łańcuch znakowy (zawartość nieistotna, różna od NULL), podać deklarację klasy zadań. Następnie odebrać z przestrzeni kolejno \textbf{100} obiektów klasy \textit{Odpowiedź} o atrybutach \textit{id} typu \textit{Integer} oraz \textit{wynik} typu \textit{Integer} posiadające w atrybucie id wartość \textbf{35}, a następnie wszystkie z atrybutem \textit{id} = \textbf{10}. Przyjąć, że klasa \textit{Odpowiedź} jest już zdefiniowana zgodnie z powyższym opisem.
	
		\subsubsection{Rozwiązanie}
			\textbf{Klasa Zadanie}
			\begin{lstlisting}[language=Java]
			// deklaracja klasy, muszą być widoczne:
			// implementacja interfejsu Entry
			public class Zadanie implements Entry {
				// publiczne składowe, opakowujące typy zmiennych
				public Integer liczba;
				public String napis1;
				public String napis2;
				public Boolean poisonPill;
				// konstruktor domyślny, obowiązkowy
				public Zadanie() {
					Random rand = new Random();
					this.liczba = rand.nextInt();
					this.napis1 = Integer.toString(rand.nextInt());
					this.napis2 = Integer.toString(rand.nextInt());
					this.poisonPill = false;
				}
				public Zadanie(Integer liczba, String napis1, String napis2, Boolean poisonPill) {
					this.liczba = liczba;
					this.napis1 = napis1;
					this.napis2 = napis2;
					this.poisonPill = poisonPill;
				}
			}
			\end{lstlisting}
			\textbf{Nadrzędna klasa Klienta}
			\begin{lstlisting}[language=Java]
			/**
			 * @author Son Mati & Doxus
			 */
			public class Client {
				protected Integer defaultLease = 100000;
				protected JavaSpace space;
				protected Lookup lookup;
				public Client() {
					lookup = new Lookup(JavaSpace.class);
				}
			}
			\end{lstlisting}
			\newpage
			\textbf{Klasa Nadzorcy}
			\begin{lstlisting}[language=Java]
			/**
			* @author Son Mati & Doxus
			*/
			public class Boss extends Client {
				// domyślne wartości dla zadania
				static final int DEFAULT_TASK_NUMBER = 200;
				static final int DEFAULT_MAX_MISSES = 100;
				
				Integer taskNumber;
				Integer maxMisses;
				
				public Integer getTaskNumber() {
					return taskNumber;
				}
				public Integer getMaxMisses() {
					return maxMisses;
				}
				// obowiązkowy domyślny konstruktor
				public Boss() {
					taskNumber = DEFAULT_TASK_NUMBER;
					maxMisses = DEFAULT_MAX_MISSES;
				}
				public Boss(Integer taskNumber, Integer maxMisses) {
					this.taskNumber = taskNumber;
					this.maxMisses = maxMisses;
				}
				/**
				 * Wygenerowanie zadania z losowymi wartościami
				 * @param id identyfikator
				 * @param poisonPill pigułka, tak czy nie
				 */
				public Zadanie generateTask(int id, boolean poisonPill) {
					Random rand = new Random();
					return new Zadanie(id, Integer.toString(rand.nextInt(1000)),
					Integer.toString(rand.nextInt(1000)), poisonPill);
				}
				/**
				 * Utworzenie zadaniów
				 * @param count ilość zadaniów
				 */
				public void createTasksInJavaSpace(int count) {
					try {
						this.space = (JavaSpace)lookup.getService();
						for (int i = 0; i < count; ++i) {
							Zadanie zad = this.generateTask(i, false);
							space.write(zad, null, defaultLease);
							System.out.println("Wygenerowałem zad " + i + " o stringach " + zad.napis1 + " i " + zad.napis2);
						}
					}
					catch(RemoteException | TransactionException ex) {
						System.out.println("Dupa XD");
					}
				}
			\end{lstlisting}
			\newpage
			\begin{lstlisting}[language=Java]
			public class Boss extends Client {
				/**
				* Uzyskanie odpowiedzi
				* @param id odpowiedzi, ktora nas interesuje
				* @param costam interesujący nas wynik
				* @param count liczba odpowiedzi do odbioru
				*/
				public Integer receiveData(Integer id, Integer costam, Integer count) {
					Integer found = 0;
					try {
						this.space = (JavaSpace)lookup.getService();
						Odpowiedz wzor = new Odpowiedz(id, costam);
						for (int i = 0; i < count; i++) {
							// odczyt blokujący, zatrzymuje przepływ dopóki odp się nie pojawi
							Odpowiedz wynik = (Odpowiedz) space.takeIfExists(wzor, null, defaultLease);
							if (wynik != null) {
								System.out.println("Odpowiedz: id = " + wynik.getId() + ", wynik = " + wynik.getWynik());
								found++;
							}
						}
					} catch (UnusableEntryException | TransactionException | InterruptedException | RemoteException ex) {
					}
					return found;
				}
				/**
				* ZATRUJ DZIECIACZKI XD
				*/
				public void poisonKids() {
					// utworzenie zatrutej pigulki
					Zadanie poisonPill = new Zadanie(null, null, null, true);
					try {
						space.write(poisonPill, null, defaultLease);
					} catch (TransactionException | RemoteException ex) {
					}
				}
				
				public static void main(String[] args) {
					Integer misses = 0;
					Boss boss = new Boss();
					boss.createTasksInJavaSpace(boss.getTaskNumber());
					boss.receiveData(35, null, 100);
					// boss odbiera pozostałe odpowiedzi, o id 10, dopóki nie trafi na pewną liczbę chybień
					while(misses < boss.getMaxMisses()) {
						if (boss.receiveData(10, null, 1) == 0)
						misses++;
					}
					boss.poisonKids();
				}
			}
			\end{lstlisting}
			\newpage
			\textbf{Klasa Pracownika}
			\begin{lstlisting}[language=Java]
			/**
			* @author Son Mati & Doxus
			*/
			public class Sidekick extends Client {
				// obowiązkowy domyślny konstruktor
				public Sidekick() {
				}
				// praca
				public void zacznijMurzynic() {
					while(true) {
						try {
							Random rand = new Random();
							this.space = (JavaSpace)lookup.getService();
							Zadanie zad = new Zadanie(null, null, null, null);
							zad = (Zadanie) space.takeIfExists(zad, null, defaultLease);
							if (zad != null) {
								if (zad.poisonPill == true) {
									space.write(zad, null, defaultLease);
									return;
								}
								System.out.println("Odebrałem zadanie o id " + zad.liczba
								+ " i napisach " + zad.napis1 + " i " + zad.napis2);
							}
							Odpowiedz odp = new Odpowiedz(rand.nextInt(51), rand.nextInt(1000));
							space.write(odp, null, defaultLease);
						} catch (TransactionException | RemoteException | UnusableEntryException | InterruptedException ex) {
							Logger.getLogger(Sidekick.class.getName()).log(Level.SEVERE, null, ex);
						}
					}
				}
				// obowiązkowy Run
				public static void main(String[] args) {
					Sidekick murzyn = new Sidekick();
					murzyn.zacznijMurzynic();
				}
			}
			\end{lstlisting}
	\newpage
	\subsection{2014, I termin, Adam Duszeńko}
		\subsubsection{Treść}
			Napisać kod programu głównego wykonawczego do przetwarzania z wykorzystaniem maszyny JavaSpace przetwarzającego obiekty zadań zawierające dwie wartości całkowite, oraz numer obiektu i flagę logiczną początkowo zawierającą wartość \textit{FALSE}. W momencie pobrania obiektu zadania program wykonawczy ma podmienić w przestrzeni JavaSpace pobrany obiekt na ten sam, ale z flagą ustawioną na wartość \textit{TRUE}. Przetwarzanie obiektu realizowane jest w funkcji \textit{int check(int, int)} do której należy przekazać wartości z obiektu zadania. PO skończeniu przetwarzania zadania, przed zwróceniem wyniku, należy usunąć z przestrzeni JavaSpace obiekt przetwarzanego zadania. Wynik funkcji \textit{check} należy umieścić w obiekcie wynikowym którego strukturę proszę zaproponować. Obsłużyć koniec działania programu przez skonsumowanie "zatrutej pigułki".
		\subsubsection{Propozycja rozwiązania 1}
	\newpage
	\subsection{2015, 0 termin, Adam Duszeńko}
		\subsubsection{Treść}
			Napisać program umieszczający w przestrzeni JavaSpace \textbf{1000} obiektów zadań zawierających \textbf{dwa} pola typu całkowitego oraz \textbf{dwa} pola typu łańcuch znakowy (zawartość nieistotna, różna od NULL), podać deklarację klasy zadań. Następnie odebrać z przestrzeni \textbf{20} obiektów klasy \textit{Odpowiedź} o atrybutach \textit{id} typu \textit{Integer} oraz \textit{wynik} typu \textit{Integer} posiadające w atrybucie id wartość \textbf{50} (przyjąć, że klasa \textit{Odpowiedź} jest już zdefiniowana zgodnie z powyższym opisem).
		\subsubsection{Rozwiązanie}
			Działające i przetestowane w warunkach domowych na Jini.\\\\
			\textbf{Klasa Zadanie}
			\begin{lstlisting}[language=Java]
			/**
			* @author Son Mati & Doxus
			*/
			// deklaracja klasy, muszą być widoczne:
			// implementacja interfejsu Entry
			public class Zadanie implements Entry {
				// publiczne składowe, muszą być wielkich typów opakowujących
				public Integer liczba;
				public String napis1;
				public String napis2;
				public Boolean poisonPill;
				// konstruktor domyślny, wymagany
				public Zadanie() {
					Random rand = new Random();
					this.liczba = rand.nextInt();
					this.napis1 = Integer.toString(rand.nextInt());
					this.napis2 = Integer.toString(rand.nextInt());
					this.poisonPill = false;
				}
				// konstruktor z parametrami
				public Zadanie(Integer liczba, String napis1, String napis2, Boolean poisonPill) {
					this.liczba = liczba;
					this.napis1 = napis1;
					this.napis2 = napis2;
					this.poisonPill = poisonPill;
				}
			}
			\end{lstlisting}
			\newpage
			\textbf{Klasa nadzorcy}
			\begin{lstlisting}[language=Java]
					/**
					 * @author Son Mati & Doxus
					 */
					public class Boss extends Client {
						// liczba zadań do wykonania
						static final int TASK_NUMBER = 1000;
						// obowiązkowy domyślny konstruktor
						public Boss() {
						}
						/**
						 * Wygenerowanie zadania z losowymi wartościami
						 * @param count ilość zadaniów
						 */
						public Zadanie generateTask(int id, boolean poisonPill) {
							Random rand = new Random();
							return new Zadanie(id, Integer.toString(rand.nextInt(1000)), Integer.toString(rand.nextInt(1000)), poisonPill);
						}
						/**
						 * Utworzenie zadaniów
						 * @param count ilość zadaniów
						 */
						public void createTasksInJavaSpace(int count) {
							try {
								this.space = (JavaSpace)lookup.getService();
								for (int i = 0; i < count; ++i) {
									Zadanie zad = this.generateTask(i, false);
									space.write(zad, null, defaultLease);
									System.out.println("Wtgenerowałem zad " + i + " o stringach " + zad.napis1 + " i " + zad.napis2);
								}
							}
							catch(RemoteException | TransactionException ex) {
								System.out.println("Dupa XD");
							}
						}
			\end{lstlisting}
			\newpage
			\begin{lstlisting}[language=Java]
					public class Boss extends Client {
						/**
						* Uzyskanie odpowiedzi
						* @param id odpowiedzi, ktora nas interesuje
						* @param costam interesujący nas wynik
						* @param count liczba odpowiedzi do odbioru
						*/
						public void receiveData(Integer id, Integer costam, Integer count) {
							try {
								this.space = (JavaSpace)lookup.getService();
								Odpowiedz wzor = new Odpowiedz(id, costam);
								for (int i = 0; i < count; i++) {
									// odczyt blokujący, zatrzymuje przepływ dopóki odp się nie pojawi
									Odpowiedz wynik = (Odpowiedz) space.take(wzor, null, defaultLease);
									System.out.println("Odpowiedz: id = " + wynik.getId() + ", wynik = " + wynik.getWynik());
								}
							} catch (UnusableEntryException | TransactionException | InterruptedException | RemoteException ex) {
								Logger.getLogger(Boss.class.getName()).log(Level.SEVERE, null, ex);
							}
						}
						/**
						 * ZATRUJ DZIECIACZKI XD
						 */
						public void poisonKids() {
							// utworzenie zatrutej pigulki
							Zadanie poisonPill = new Zadanie(null, null, null, true);
							try {
								space.write(poisonPill, null, defaultLease);
							} catch (TransactionException | RemoteException ex) {
								Logger.getLogger(Boss.class.getName()).log(Level.SEVERE, null, ex);
							}
						}
						/**
						 * Obowiązkowy Run dla nadzorcy
						 */
						public static void main(String[] args) {
							Boss boss = new Boss();
							boss.createTasksInJavaSpace(TASK_NUMBER);
							boss.receiveData(50, null, 20);
							boss.poisonKids();
						}
					}
			\end{lstlisting}
			\newpage
			\textbf{Klasa Pracownika}
			\begin{lstlisting}[language=Java]
				/**
				 * @author Son Mati & Doxus
				 */
				public class Sidekick extends Client {
					// obowiązkowy domyślny konstruktor
					public Sidekick() {
					}
					// rozpoczęcie pracy
					public void zacznijMurzynic() {
						while(true) {
							try {
								Random rand = new Random();
								this.space = (JavaSpace)lookup.getService();
								Zadanie zad = new Zadanie(null, null, null, null);
								zad = (Zadanie) space.takeIfExists(zad, null, defaultLease);
								if (zad != null) {
									if (zad.poisonPill == true) {
										space.write(zad, null, defaultLease);
										return;
									}
									System.out.println("Odebrałem zadanie o id " + zad.liczba
									+ " i napisach " + zad.napis1 + " i " + zad.napis2);
								}
								Odpowiedz odp = new Odpowiedz(rand.nextInt(51), rand.nextInt(1000));
								space.write(odp, null, defaultLease);
							} catch (TransactionException | RemoteException | UnusableEntryException | InterruptedException ex) {
								Logger.getLogger(Sidekick.class.getName()).log(Level.SEVERE, null, ex);
							}
						}
					}
					// obowiązkowy punkt wejścia
					public static void main(String[] args) {
						Sidekick murzyn = new Sidekick();
						murzyn.zacznijMurzynic();
					}
				}
			\end{lstlisting}
	
	
	
	
	
	
	
	
	
	
	
	
	
	
	
	