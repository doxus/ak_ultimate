% !TeX spellcheck = pl_PL
\section{MOSIX}
		\textbf{n węzłów klastra liczy wartość \textit{F(x, y)}.}
		\begin{lstlisting}[language=C]
			double fun(double x, double y, int k)
		\end{lstlisting}
		\textbf{k - liczba elementów iloczynu; x, y - argumenty funkcji. k jest podzielne przez n bez reszty.}
		\begin{lstlisting}[language=C]
			#include <stdio.h>
			#include <stdlib.h>
			#include <math.h>
			
			double calculateProduct(double x, double y, double a) {
				return a * sqrt(x * y) / (2 * pow(x, 3) + 5 * y);
			}
			
			int main(int argc, char * argv[]) {
				int n = atoi(argv[1]);		// liczba węzłów klastra
				int k = atoi(argv[2]);		// liczba elementów iloczynu
				double x = atof(argv[3]);	// pierwszy argument
				double y = atof(argv[4]);	// drugi argument
				int final_product = 1;		// iloczyn wynikowy
				int process_number = k / n;
				int response_stream[2];		// strumień dla danych
				double part_result = 1;		// jebnięcie potoków
				pipe(response_stream);
				// utworzenie procesów potomnych
				int i, j;
				for (i = 0; i < process_number; i++)
				{
					if (fork() == 0)
					{
						for (j = i * process_number; j < (i + 1) * process_number; j++)
						{
							part_result = part_result * calculateProduct(x, y, j);
							write(response_stream[1], &part_result, sizeof(double));
						}
						exit(0);
					}
				}
				// odczytanie danych częściowych
				for (i = 0; i < process_number; i++)
				{
					read(response_stream[0], &part_result, sizeof(double));
					final_product = final_product * part_result;
				}
				printf("Twój szczęśliwy iloczyn to %g", final_product);
				return 0;
			}
		\end{lstlisting}
			\textbf{n węzłów klastra liczy całkę metodą trapezów z funkcji 	$$ y=15x^3-7x^2+sin(2x^4)-8 $$}
			\begin{lstlisting}[language=C]
				double calka(double xl, double xp, double krok)
			\end{lstlisting}
		\textbf{xl - początek przedziału całkowania, xp - koniec przedziału całkowania, krok - wartość kroku całkowania.}
		\begin{lstlisting}[language=C]
			#include <stdio.h>
			#include <stdlib.h>
			#define N 4
			
			double func(double x)
			{
				return 15 * pow(x, 3) - 7*x*x + sin(2 * pow(x, 4)) - 8;
			}
			// x1: wspolrzedna x poczatku przedzialu calkowania
			// xk: wspolrzedna x konca przedzialu calkowania
			// krok: wysokosc trapezu (x2 - x1)
			double liczCalke(double x1, double xk, double krok)
			{
				double x2 = x1 + krok;					// Wspolrzedna x drugiej podstawy trapezu
				double a, b;							// Wartosci funkcji w x1, x2
				double wynik = 0;
				
				while (x2 < xk)
				{
					a = func(x1);
					b = func(x2);
					wynik += (a + b) * krok / 2;
					
					x1 += krok;
					x2 += krok;
				}
				return wynik;
			}
			double calka(double xl, double xp, double krok)
			{
				int i, forkResult;
				// Wartosc calki oznaczonej w calym przedziale / przedziale dziecka
				double mainResult = 0, childResult = 0;
				// Szerokość przedziału, w jakim dziecko liczy całkę
				double delta = (xp - xl) / N;
				// Potok do zapisu i odczytu
				int potok[2];
				pipe(potok);
		\end{lstlisting}
		\begin{lstlisting}[language=C]
			{
				// Kindermachen
				for (i = 0; i < N; i++)
				{
					forkResult = fork();
					if (forkResult == 0)
					{
						childResult = liczCalke(xl, xl + delta, krok);
						write(potok[1], &childResult, sizeof(childResult));
						exit(0);
					}
					else if (forkResult < 0)
					{
						printf("Wystapil blad podczas tworzenia dziecka");
						return -1;
					}
					xl += delta;
				}
				// Odczytanie i zsumowanie wyników przez rodzica
				for (i = 0; i < N; i++)
				{
					read(potok[0], &childResult, sizeof(childResult));
					mainResult += childResult;
				}
				printf("\nCalka obliczona metoda trapezow jest rowna: %f\n", mainResult);
				return mainResult;
			}
		\end{lstlisting}
			\textbf{n procesów potomnych. Zarządca wysyła k danych do potomkówm które w pętli odbierają liczbe, liczą pole kołoa, wysyłają wynik.}
			\begin{lstlisting}[language=C]
			#include <stdio.h>
			#include <stdlib.h>
			#include <math.h>
			#define PI 3.14159
			
			float calculate_circle_area(int radius)
			{
				return PI * pow((float) radius, 2);
			}
			
			int main(int argc, char* argv[])
			{
				int n = atoi(argv[1]);
				int k = atoi(argv[2]);
				//promien -1 oznacza ze jest to trujaca pigulka zabijajaca proces potomny
				int poison = -1;		
				//allokacja tablicy k liczb (promieni)
				int * tab = (int*) malloc(sizeof(int) * k);
				int i;
				for (i = 0; i < k; i++)
				{
					tab[i] = rand() % 20;
				}
				int process_number = n / k;
				// 2 strumienie, jeden do wysyłania danych, drugi do odbioru odpowiedzi
				int data_stream[2], response_stream[2];
				// wyniki
				int part;
				int sum = 0;
				// jebnięcie potoków
				// odpowiedzi z procesów potomnych
				pipe(response_stream);
				// in - promienie kół
				// out - otrzymane wyniki - pola kół
				pipe(data_stream);
			\end{lstlisting}
			\begin{lstlisting}[language=C]
				// utworzenie procesów potomnych
				for (i = 0; i < n; i++)
				{
					if (fork() == 0)
					{
						int radius;
						// wykonywanie obliczeń dopóki nie zostanie OTRUTY(!)
						while(true)
						{
							if (read(data_stream[0], &radius, sizeof(int)) != sizeof(int))
								continue;
							if (radius == -1)	// wyłącza się tylko jak otrzymamy pigułkę
								exit(0);
							float result = calculate_circle_area(radius);
							write(response_stream[1], &result, sizeof(float));
						}
					}
				}
				// wysłanie danych do procesów potomnych
				for (i = 0; i < k; i++)
				{
					write(data_stream[1], &tab[i], sizeof(int));
				}
				// Halo odbjoor danych
				for (i = 0; i < n; i++)
				{
					read(response_stream[0], &part, sizeof(int));
					sum = sum + part;
				}
				// ZABIJANIE DZIECI
				for (i = 0; i < k; i++)
				{
					write(data_stream[1], &poison, sizeof(int));
				}
				// wypisanie odpowiedzi
				printf("Suma pól kół: %d", sum);
				return 0;
			}
			\end{lstlisting}
			\textbf{Poszukiwanie wartości wielomianu k stopnia, obliczenia na 2 klastrach za pomocą rekurencji}
			$$ P_k(x)=15P_{k-1}\left(\cfrac{3}{8}x\right)-P_{k-2}\left(\cfrac{x^2}{2}\right)\;\;\;\;\;\;P_0(x)=1\;\;\;\;\;\;P_1(x)=x$$
			\begin{lstlisting}[language=C]
				 #include <stdio.h>
				 #include <stdlib.h>
				 // Main
				 int main(int argc, char *argv[])
				 {	
					 double wynik = 0;				// Ostateczny wynik programu
					 double x;
					 int k;
					 if (argc == 3)
					 {		
						 k = atoi( argv[1] );
						 x = atof( argv[2] );
						 wynik = liczWielomian(k, x);
						 printf("Wielomian jest rowny: %f\n", wynik);
					 }
					 else
						 printf("Podano nieprawidlowe parametry!\n");
					 return 0;
				 }
				 // Metoda do liczenia wartosci wielomianu
				 double licz2(int k, double x)
				 {
					 double result, Pk1, Pk2;
					 if (k == 0)
						 return 1;
					 if (k == 1)
						 return x;
					 Pk1 = licz2(k - 1, 0.375 * x);
					 Pk2 = licz2(k - 2, x*x / 2.0);
					 result = 15 * Pk1 - Pk2;
					 return result;
				 }
			\end{lstlisting}
			\begin{lstlisting}[language=C]
				 // Obliczenie całości wielomianu
				 double liczWielomian(int k, double x)
				 {
					 double mainResult = 0, childResult1, childResult2;
					 int potok[2];
					 if (k == 0)
						 return 1;
					 if (k == 1)
						 return x;
					 pipe(potok);
					 // Pierwsze dziecko - wielomian k-1 stopnia
					 if ( fork() == 0 )
					 {
						 childResult1 = 15 * licz2(k - 1, 0.375 * x);
						 write( potok[1], &childResult1, sizeof(childResult1) );
						 exit(0);
					 }
					 // Drugie dziecko - wielomian k-2 stopnia
					 if ( fork() == 0 )
					 {
						 childResult2 = licz2(k - 2, x*x / 2.0);
						 write( potok[1], &childResult2, sizeof(childResult2) );
						 exit(0);
					 }
					 read( potok[0], &childResult1, sizeof(childResult1) );
					 read( potok[0], &childResult2, sizeof(childResult2) );
					 mainResult = childResult1 - childResult2;
					 return mainResult;
				 }
			\end{lstlisting}