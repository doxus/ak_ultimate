% !TeX spellcheck = pl_PL
\newpage
\section{PVM}
	\subsection{Wstęp z laborek, szukanie min i max}
		\subsubsection{Treść}
			Napisać program znajdujący minimum i maksimum z macierzy.\\
			Hello.c - program główny, rodzic; Hello\_other.c - program podrzędny, potomek.
		\subsubsection{Rozwiązanie}
			Program przekazuje kolejne wiersze macierzy do programów potomnych, które znajdują lokalne minimum i maksimum. Program zbiera wszystkie minima i maksima do tablicy o rozmiarze wysokości macierzy. Pod koniec sam ręcznie wylicza min i max z tych dwóch tablic.\\
			Należy pamiętać, że programy potomne muszą fizycznie znajdować się na dyskach innych komputerów w sieci PVM.\\
			\textbf{Program działający, oceniony na 5.}
			\begin{lstlisting}[language={C}]
				/* - Autorzy:
				   -- Forczu Forczmański
				   -- Wuda Wudecki
				*/
				#include <stdio.h>
				#include <stdlib.h>
				#include <math.h>
				#include "pvm3.h"
				#define WYSOKOSC 5		// liczba wierszy
				#define SZEROKOSC 5		// liczba kolumn
				/// Program rodzica
				main()
				{
					// dane potrzebne do obliczeń
					int matrix[WYSOKOSC][SZEROKOSC];
					int min_result[WYSOKOSC], max_result[WYSOKOSC];
					int minimum, maksimum;
					// wypełnienie macierzy danymi
					int i, j;
					for ( i = 0; i < WYSOKOSC; ++i )
						for ( j = 0; j < SZEROKOSC; ++j)
							matrix[i][j] = rand() % 30;
					// wypisanie macierzy na konsoli
					for ( i = 0; i < WYSOKOSC; ++i )
					{
						for ( j = 0; j < SZEROKOSC; ++j)
							printf("%d ", matrix[i][j]);
						printf("\n\n");
					}
					// pobranie informacji
					int ilhost, ilarch;
					struct pvmhostinfo * info;
					pvm_config(&ilhost, &ilarch, &info);
					printf("Liczba hostow: %d\n", ilhost);
					
					int id1 = 0;
					int tid;
					
					
					
					// Dla każdego hosta - inicjujemy go
					for ( i = 0; i < ilhost; i++ )
					{
						pvm_spawn( "/home/pvm/pvm3/sekcja11/bin/LINUX/hello_other", 0, PvmTaskHost, info[i].hi_name, 1, &tid);
						if ( tid < 0 )
						{
							ilhost--;
							continue;
						}
						printf("tid: %d\n", tid);
						pvm_initsend(PvmDataDefault);
						// wysyłamy:
						// id wiersza
						pvm_pkint(&id1, 1, 1);
						// elementy wiersza
						pvm_pkint(&matrix[id1][0], SZEROKOSC, 1);
						pvm_send(tid, 100);
						id1++;
					}
					//// Wykonywanie programu aż do przedostatniej pętli
					int bufid, child_tid, child_id1, tmp;
					while ( id1 < WYSOKOSC )
					{
						bufid = pvm_recv(-1, 200);
						pvm_bufinfo(bufid, &tmp, &tmp, &child_tid);
						printf("recv: %d\n", child_tid);
						// pobranie id wiersza
						pvm_upkint(&child_id1, 1, 1);
						// pobranie nowych min / max
						pvm_upkint(&min_result[child_id1], 1, 1);
						pvm_upkint(&max_result[child_id1], 1, 1);
						// wysłanie nowych danych
						pvm_initsend(PvmDataDefault);
						// id kolejnego wiersza
						pvm_pkint(&id1, 1, 1);
						// nowy wiersz
						pvm_pkint(&matrix[id1][0], SZEROKOSC, 1);
						pvm_send(child_tid, 100);
						id1++;
					}
					//// Odebranie ostatnich danych
					for	(i = 0; i < id1 - ilhost + 1; i++ )
					{
						bufid = pvm_recv(-1, 200);
						pvm_bufinfo(bufid, &tmp, &tmp, &child_tid);
						printf("recv: %d\n", child_tid);
						// pobranie id wiersza
						pvm_upkint(&child_id1, 1, 1);
						// pobranie nowych min / max
						pvm_upkint(&min_result[child_id1], 1, 1);
						pvm_upkint(&max_result[child_id1], 1, 1);
					}
					
					
					
					
					// uzysaknie minimum z wiersza
					minimum = min_result[0];
					maksimum = max_result[0];
					for (j = 1; j < WYSOKOSC; j++)
					{
						if ( max_result[j] > maksimum )
							maksimum = max_result[j];
						if ( min_result[j] < minimum )
							minimum = min_result[j];
					}
					printf("Uzyskane wartosci:\nMIN: %d, MAX: %d\n", minimum, maksimum);
					pvm_exit();
					return 0;
				}
			\end{lstlisting}
			\textbf{Program potomka}
			\begin{lstlisting}[language={C}]
				#include <stdio.h>
				#include <math.h>
				#include "pvm3.h"
				#define WYSOKOSC 5		// liczba wierszy
				#define SZEROKOSC 5		// liczba kolumn
				/// Program potomka
				int main()
				{
					int masterid, id1, j, curr_row[SZEROKOSC], curr_min, curr_max;
					// pobierz id rodzica 
					masterid = pvm_parent();
					if (masterid == 0)
						exit(1);
					while(1)
					{
						pvm_recv(masterid, 100);
						// pobranie wartości:
						// id wiersza
						pvm_upkint(&id1, 1, 1);
						pvm_upkint(&curr_row[0], SZEROKOSC, 1);
						// uzysaknie minimum z wiersza
						curr_min = curr_max = curr_row[0];
						for (j = 1; j < SZEROKOSC; j++)
						{
							if ( curr_row[j] > curr_max )
								curr_max = curr_row[j];
							if ( curr_row[j] < curr_min )
								curr_min = curr_row[j];
						}
						// wysłanie nowych danych
						pvm_initsend(PvmDataDefault);
						pvm_pkint(&id1, 1, 1);
						pvm_pkint(&curr_min, 1, 1);
						pvm_pkint(&curr_max, 1, 1);
						pvm_send(masterid, 200);
					}
					pvm_exit();
					return 0;
				}
			\end{lstlisting}
	\subsection{Laborki, odejmowanie macierzy}
		\subsubsection{Treść}
			Odejmowanie macierzy.
		\subsubsection{Rozwiązanie}
			Ocena nieznana.
			\begin{lstlisting}[language=C]
				/* AK Lab 2 - PVM
					Anna Kusnierz
					Tomasz Szoltysek
					Temat: Odejmowanie dwoch macierzy
				*/
				#include <stdio.h>
				#include <math.h>
				#include "pvm3.h"
				#define MATRIX_SIZE 20
				int main() 
				{
					int i,j;
					int count = 0; 		//licznik wierszy macierzy
					int rescount;
					int tidmaster, ilhost, ilarch, bufid,t_id,bytes,msgtag;
					struct pvmhostinfo info;
					
					int a[MATRIX_SIZE][MATRIX_SIZE], b[MATRIX_SIZE][MATRIX_SIZE], r[MATRIX_SIZE][MATRIX_SIZE];
					FILE *txt = fopen("result.txt","w");
					
					for(i=0;i<MATRIX_SIZE;i++)
					{
						for(j=0;j<MATRIX_SIZE;j++)
						{
							a[i][j] = rand();
							b[i][j] = rand();
						}
					}
					fprintf(txt,"Macierz A:\n--------------------------------------------------\n\n");
					for(i=0;i<MATRIX_SIZE;i++)
					{
						for(j=0;j<MATRIX_SIZE;j++)
							fprintf(txt,"%d\t",a[i][j]);
						fprintf(txt,"\n");
					}
					fprintf(txt,"Macierz B:\n--------------------------------------------------\n\n");
					for(i=0; i<MATRIX_SIZE;i++)
					{
						for(j=0;j<MATRIX_SIZE;j++)
							fprintf(txt,"%d\t",b[i][j]);
						fprintf(txt,"\n");
					}
					
					
					
					tidmaster = pvm_mytid();
					pvm_config(&ilhost, &ilarch, &info);
					printf("%d",ilhost);
					for(i=0; i < (ilhost > MATRIX_SIZE ? MATRIX_SIZE : ilhost) ;i++)
					{
						pvm_spawn("/home/pvm3/pvm3/sekcja4/hello_other",0,PvmTaskHost,info[i].hi_name,1,&t_id);
						pvm_initsend(PvmDataDefault);
						pvm_pkint(&a[count][0],MATRIX_SIZE,1);
						pvm_pkint(&b[count][0],MATRIX_SIZE,1);
						pvm_pkint(&count,1,1);
						pvm_send(t_id,100);	
						++count;
					}
					while(count<MATRIX_SIZE)
					{
						bufid = pvm_recv(-1,200);
						pvm_bufinfo(bufid,&bytes,&msgtag,&t_id);
						pvm_upkint(&rescount,1,1);
						pvm_upkint(&r[rescount][0],MATRIX_SIZE,1);
						pvm_initsend(PvmDataDefault);
						pvm_pkint(&a[count][0],MATRIX_SIZE,1);
						pvm_pkint(&b[count][0],MATRIX_SIZE,1);
						pvm_pkint(&count,1,1);
						pvm_send(t_id,100);
						++count;
					}
					for(i = 0; i < (ilhost > MATRIX_SIZE ? MATRIX_SIZE : ilhost); i++)
					{
						bufid = pvm_recv(-1,200);
						pvm_bufinfo(bufid,&bytes,&msgtag,&t_id);
						pvm_upkint(&rescount,1,1);
						pvm_upkint(&r[rescount][0],MATRIX_SIZE,1);
						pvm_kill(t_id);	
					}
					fprintf(txt,"Macierz wynikowa:\n--------------------------------------------------\n\n");
					for(i=0; i<MATRIX_SIZE;i++)
					{
						for(j=0;j<MATRIX_SIZE;j++)
							fprintf(txt,"%d\t",r[i][j]);
						fprintf(txt,"\n");
					}
					fclose(txt);
					exit(0);
				}
			\end{lstlisting}
			\newpage
			\textbf{Program potomny}
			\begin{lstlisting}[language=C]
				#include <stdio.h>
				#include "pvm3.h"
				#define MATRIX_SIZE 20
				int main()
				{
					int masterid, count, i;
					double vecta[MATRIX_SIZE],vectb[MATRIX_SIZE],vectr[MATRIX_SIZE];
					masterid = pvm_parent();
					if(masterid == 0) exit(1);	//zabezpieczenie przed uruchomieniem z poziomu rodzica
					//OBSŁUGA OBLICZEŃ i WYSYŁANIA WYNIKÓW
					while(1)
					{
						pvm_recv(masterid,100);
						pvm_upkdouble(&vecta[0],MATRIX_SIZE,1);
						pvm_upkdouble(&vectb[0],MATRIX_SIZE,1);
						pvm_upkint(&count,1,1);
						for(i = 0; i < MATRIX_SIZE; i++)
							vectr[i] = vecta[i] - vectb[i];
						pvm_initsend(PvmDataDefault);
						pvm_pkint(&count,1,1);
						pvm_pkdouble(&vectr[0],MATRIX_SIZE,1);
						pvm_send(masterid,200);		
					}
					return(0);
				}
			\end{lstlisting}
	\newpage
	\subsection{Laborki, Szyfr Cezara}
		\subsubsection{Treść}
			Szyfr Cezara
		\subsubsection{Rozwiązanie}
			Podobno działa, ocena nieznana.
			\begin{lstlisting}[language=C]
				/// program główny
				main()
				{
					if(pvm_parent() == PvmNoParent) 
						rodzic();
					else 
						potomek();		
					return 0;	
				}
			\end{lstlisting}
			\begin{lstlisting}[language=C]
				/// Potomek
				void potomek()
				{
					int masterid, dlugosc, pozycja;
					char znak;
					masterid = pvm_parent();	/*kto mnie stworzyl*/
					while(1)
					{
						pvm_recv(-1,100);
						pvm_upkbyte(&znak,1,1);
						pvm_upkint(&pozycja,1,1);
						if(znak != ' ')
						{
							if (((znak >= 'A') && (znak <= 'Z')) ||
								((znak >= 'a') && (znak <= 'z')))
							{
								if (znak == 'x')
									znak = 'a';
								if (znak == 'y')
									znak = 'b';
								if (znak == 'z')
									znak = 'c';
								if (znak == 'X')
									znak = 'A';
								if (znak == 'Y')
									znak = 'B';
								if (znak == 'Z')
									znak = 'C';
							}
							else
								znak += 3;
						}
						pvm_initsend(PvmDataDefault);
						pvm_pkbyte(&znak,1,1);
						pvm_pkint(&pozycja,1,1);
						pvm_send(masterid,200);
					}
				}
			\end{lstlisting}
			\begin{lstlisting}[language=C]
				/// Rodzic
				void rodzic()
				{
					char* text = "ala ma kota";
					char* text2;
					int length = strlen(text);
					int tidmaster, ilhost, ilarch, *tid, i, procesy = 0, budif, temp_id, zm1, zm2;
					struct pvmhostinfo *info;
					if ((tidmaster = pvm_mytid()) < 0) exit(1);
					pvm_config(&ilhost, &ilarch, &info);
					tid = (int*) calloc (ilhost, sizeof(int));
					text2 = (char*) calloc (length + 1, 1);
					text2[length] = '\0';
					for(i = 0; i < ilhost; ++i) {
						if(i >= length) break;
						pvm_spawn("/home/pvm/pvm3/sekcja15/bin/LINUX/hello", 0, PvmTaskHost, info[i].hi_name, 1, &tid[i]);
						pvm_initsend(PvmDataDefault);
						pvm_pkbyte((text + i), 1, 1);
						pvm_pkint(&i, 1, 1);
						pvm_send(tid[i], 100);
						++procesy;
					}
					i = ilhost;
					while (1) {
						budif = pvm_recv(-1, 200);
						char temp;
						int recIndex;
						pvm_upkbyte(&temp, 1, 1);
						pvm_upkint(&recIndex, 1, 1);
						text2[recIndex] = temp;
						if (i >= length) 	break;
						else {
							pvm_bufinfo(budif, &zm1, &zm2, &temp_id);
							printf("Od: %d \n", temp_id);
							pvm_initsend(PvmDataDefault);
							pvm_pkbyte((text + i), 1, 1);
							pvm_pkint(&i, 1, 1);
							pvm_send(temp_id, 100);
							++i;
						}
					}
					for (i = 0; i < (ilhost-1); ++i) {
						budif = pvm_recv(-1, 200);
						char temp;
						int recIndex;
						pvm_upkbyte(&temp, 1, 1);
						pvm_upkint(&recIndex, 1, 1);
						text2[recIndex] = temp;
						pvm_bufinfo(budif, &zm1, &zm2, &temp_id);
						printf("Od: %d \n", temp_id);
						pvm_kill(temp_id);		
					}
					printf("Tekst: %s\n", text2); pvm_exit();
				}
			\end{lstlisting}
	
	\subsection{Egzamin 2012, T1, Hafed Zghidi}
		\subsection{Treść}
			Naszym zadaniem jest przeszukiwanie fragmentu obrazu (\textit{wzorzec}) o rozmiarze $ n\times n $ w obrazie \textit{Obraz} o rozmiarze $ m\times n (m>>n) $. Do tego celu służy nam odpowiednia funkcja\\
			\begin{lstlisting}[language=C]
			int find_pattern(x, y, n)
			\end{lstlisting}
			gdzie x, y to współrzędne w obrazie, a $ n $ rozmiar wzorca. Funkcja zwraca 1, jeśli wzorzec znajduje się w obrazie lub 0 jeśli go nie ma. Wzorzec może się powtarzać w obrazie.\\
			Napisać w oparciu o środowisko PVM kod programu rodzica zliczającego ile razy wzorzec znajduje się w obrazie. Program powinien zapewnić równoległe rozwiązanie tego zadania z zachowaniem dynamicznego podziału zadań. Dodatkowe punkty można uzyskać za zapewnienie lepsze granulacji podziału zadań.
		\subsection{Propozycja rozwiązania}
			\begin{lstlisting}[language=C]
				#include <stdio.h>
				#include "pvm3.h"
				#include <string.h>
				
				void parent();
				void child();
				/// program główny
				int main(void) {
					if(pvm_parent() == PvmNoParent)
						parent();
					else
						child();
					return 0;
				}
				/// rodzic
				void parent()
				{
					//rozmiary obrazu
					int n = 32;
					int m = 1024;
					//aktualna pozycja
					int x = 0, y = 0;
					
					//pvm-owe zmienne
					int tIdMaster = 0, *tId;
					int ilHost = 0, ilArch = 0;
					struct pvmhostinfo *info;
					
					//suma zliczen
					int sum = 0;
					
					//dane do przeslania, odpowiednio x y n
					int dataToSend[3];
					
					if( (tIdMaster = pvm_mytid()) < 0 )
						exit(0);
					
					//pobranie konfiguracji
					pvm_config(&ilHost, &ilArch, &info);
					
					//stworzenie tablicy z id dzieci
					tId = (int*)calloc(ilHost, sizeof(int));
					
					//wyslanie poczatkowych danych
					for(int i = 0; i < ilHost; ++i)
					{
						pvm_spawn("/home/pvm/...", 0, PvmTaskHost, info[i].hi_name, 1, &tId[i]);
						
						//przygotowanie i wyslanie danych
						pvm_initsend(PvmDataDefault);
						
						dataToSend[0] = x;
						dataToSend[1] = y;
						dataToSend[2] = n;
						
						pvm_pkint(dataToSend, 3, 1);
						pvm_send(tId[i], 100);
						//przesuniecie w obrazie o pixel w prawo
						++x;
						
						//jesli wyjdzie poza prawa strone, zejscie w dol
						if(x >= (m - n) ) {
							x = 0;
							++y;
						}
						//sam koniec obrazu - zakoncz
						if(y >= (m - n) )
							break;
					}
					//jesli zostalo cos jeszcze do wyslania...
					if(y >= (m - n) )
						while(true)
						{
							int childId = 0;
							
							//odbieramy dowolna paczke wyslana przez dzieci
							int bi = pvm_recv(-1, 100);
							int rcvdData = 0;
							
							//pobieramy jego id
							pvm_bufinfo(bi, &dlbuf, &msgtag, &childId);
							
							//rozpakowujemy dane, dodajemy do sumu
							pvm_upkint(&rcvdData, 1, 1);
							
							suma += rcvdData;
							
							//inicjujemy dane
							dataToSend[0] = x;
							dataToSend[1] = y;
							dataToSend[2] = n;
							
							pvm_initsend(PvmDataDefault);
							pvm_pkint(dataToSend, 3, 1);
							
							
							//wysylamy do dziecka, z ktorego aktualnie odebrano rezultat
							pvm_send(childId, 100);
							
							//tak jak poprzednio, przesuniecie w obrazie
							++x;
							if(x >= (m - n) ) {
								x = 0;
								++y;
							}
							if(y >= (m - n) )
								break;
						}
					
					//w tym momencie wyslalismy juz wszystko
					//odbieramy wiec ostatnia paczke od kazdego z dzieci
					for(int i = 0; i < ilHost; ++i)
					{
						int tmp;
						pvm_rcv(masterId, 100);
						pvm_upkint(&tmp, 1, 1);
						
						suma += tmp;
						pvm_kill(tId[i]);
					}         
				}
				/// Potomek
				void child()
				{
					//id rodzica
					int masterId = pvm_parent();
					
					//tablica do ktorej zapiszemy odebrane dane
					int points[3];
					
					//odbieramy paczke od rodzica
					pvm_rcv(masterId, 100);
					pvm_upkint(points, 3, 1);
					
					//funkcja z zadania
					int result = find_pattern(points[0], points[1], points[2]);
					
					//odsylamy dane
					pvm_initsend(PvmDataDefault);
					pvm_pkint(result, 1, 1);
					pvm_send(masterId, 100);
				}
			\end{lstlisting}
	
	\newpage
	\subsection{Egzamin 2013, T1, Hafed Zghidi}
		\subsection{Treść}
			Naszymi danymi wejściowymi są tabela A zawierająca 10000 łańcuchów znakowych i tabela B zawierająca 10000 wartości typu \textit{integer} będących wartościami skrótu łańcuchów tabeli A. Napisać kod programu głównego (rodzic) realizującego w oparciu o środowisko PVM równoległe wykonanie porównywania zgodności łańcuchów z tabeli A ze skrótami z tabeli B. Program potomny ma otrzymywać łańcuch znakowy z tabeli A oraz odpowiadającą mu wartość typu \textit{integer} z tabeli B. Program główny ma zliczyć ile par jest poprawnych, a ile nie zgadza się. Program potomny ma zostać uruchomiony na wszystkich węzłach w maszynie PVM. Rozdział zadania na podzadania dla procesów potomnych ma uwzględniać aspekt równoważenia obciążenia poszczególnych węzłów maszyny PVM (napisać tylko program rodzica).
		\subsection{Propozycja rozwiązania}
			\begin{lstlisting}[language=C]
				#include <stdio.h>
				#include "pvm3.h"
				#define SIZE        10000
				#define STR_SIZE    256
				#define PATH        "pvm_child"
				int main()
				{
					int tidmaster, *tid, ilhost, ilarch, bufid, a, b, temp, wynik;
					struct pvmhostinfo *info;
					char stringi[SIZE][STR_SIZE];
					int wartosci[SIZE];
					const int ROZMIAR = 10;
					WYNIK = 0;
					tidmaster = pvm_mytid();
					pvm_config(&ilhost, &ilarch, &info);
					tid = (int*)malloc(ilhost,sizeof(int));
					for (a = 0; a < ilhost; a++)
					{
						pvm_spawn(PATH, 0, PvmTaskHost, info[a].hi_name, 1, &tid[a]);
						pvm_initsend(PvmDataDefault);
						pvm_pkint(&ROZMIAR, 1, 1);
						temp = STR_SIZE;
						pvm_pkint(&temp, 1, 1);
						for(b = 0; b < ROZMIAR; b++)
						{
							pvm_pkint(&wartosci[a * ROZMIAR + b], 1, 1);
							pvm_pkstr(stringi[a * ROZMIAR + b]);
						}
						pvm_send(tid[a],100);
					}
					a = ilhost;
					while( a < SIZE/ROZMIAR ) {
						bufid = pvm_recv(-1, 200);
						pvm_upkint(&temp, 1, 1);
						WYNIK += temp;
						pvm_bufinfo(bufid, &temp, &b, &tmp_id);
						pvm_initsend(PvmDataDefault);
						pvm_pkint(&ROZMIAR, 1, 1);
						temp = STR_SIZE;
						pvm_pkint(&temp, 1, 1);
						
						
						
						for(b = 0; b < ROZMIAR; b++)
						{
							pvm_pkint(&wartosci[a * ROZMIAR + b], 1, 1);
							pvm_pkstr(stringi[a * ROZMIAR + b]);
						}
						pvm_send(tmp_id, 100);
						a++;
					}
					for(a = 0; a < ilhost; a++)
					{
						bufid=pvm_recv(-1, 200);
						pvm_upkint(&temp, 1, 1);
						WYNIK += temp;
						pvm_bufinfo(bufid, &temp, &b, &tmp_id);
						pvm_kill(tmp_id);
					}
					printf("wynik %d",WYNIK);
					return 0;
				}
			\end{lstlisting}
	
			