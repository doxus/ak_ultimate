% !TeX spellcheck = pl_PL
\newpage

\section{Sparc}
	Uwagi:
	\begin{itemize}
		\item W języku asemblera SPARC komentarze są oznaczane przez znak wykrzyknika (!), a nie średnika (;). W listingach są średniki ze względu na wbudowany listingu asemblera w latexie.
	\end{itemize}

	\subsection{2008, I termin, Jerzy Respondek}
		\subsubsection{Treść}
			Napisz funkcję w asemblerze procesora SPARC obliczającą sumę liczb naturalnych od 1 do danej $ n $ jako argument funkcji. Założyć, że $ n >= 1 $.\\
			Przykład: f(5) = 1 + 2 + 3 + 4 + 5 = 15
		\subsubsection{Propozycja rozwiązania 1}
			\begin{lstlisting}[language={[sparc]Assembler}]
				.global funkcja
				.proc 4
				funkcja:
					save %sp, -96, %sp		; trzeba tutaj to robić ???
					mov %i0, %l0			; a
					mov 1, %l1				; liczba naturalna   
					mov 0, %l2				; wynik
				pętla:
					add %l1, %l2, %l2		; liczba + suma = suma
					add %l1, 1, %l1			; liczba++
					subbcc %l0, 1, %l0		; a--
					bl koneic
					nop
					ba pętla
					nop
				koniec:
					mov %l2, %i0			; wynik
					ret
					restore
			\end{lstlisting}
		\subsubsection{Propozycja rozwiązania 2}
			\begin{lstlisting}[language={[sparc]Assembler}]
			    .global sumator
			    .proc 4
			    sumator:
				    save %sp, -96, %sp		! przesunięcie okna
				    mov %i0, %l1			! a w l1
				    mov %l1, %l0			! suma = a
				   petla:
				    subcc %l1, 1, %l1		! dekrementacja licznika
				    bneg koniec
				    add %l0, %l1, %l0		! suma += licznik
				    ba petla
			    koniec:
				    mov %l0, %i0			! zwrócenie sumy
				    ret
				    restore					! przywrócenie stanu okna
			\end{lstlisting}
			
	\newpage
	\subsection{2010, I termin, Jerzy Respondek}
		\subsubsection{Treść}
			Napisz w asemblerze procesora SPARC funkcję obliczającą sumę kwadratów wszystkich liczb całkowitych z przedziału \emph{a} do \emph{b}. Założyć $ a < b $, np.\\
			f(2, 5) = 2 * 2 + 3 * 3 + 4 * 4 + 5 * 5\\
			Nagłówek funkcji ma mieć postać:
			\begin{lstlisting}[language=C]
				int f(int a, int b)
			\end{lstlisting}
		\subsubsection{Propozycja rozwiązania}
	\newpage
	\subsection{2012, I termin, Jerzy Respondek}
		\subsubsection{Treść}
			Napisz w asemblerze procesora SPARC funkcję realizującą dokładnie tę samą operację co jej odpowiednik w języku C:
			\begin{lstlisting}[language=C]
				int f(int *tab, int n)
				{
					int i, suma = 0;
					for(i = 0; i < n; i++)
					{
						suma -= (2 * i + 1) * tab[i];
						suma *= suma;
					}
					return suma;
				}
			\end{lstlisting}
		\subsubsection{Propozycja rozwiązania 1}
			\begin{lstlisting}[language={[sparc]Assembler}]
				.global func
				.proc 4
				
				funkcja:
					save %sp, -96, %sp
					mov %i0, %l0			; wskaźnik tablicy, tak podano argument
					ld [%i0], %l1			; wartość tablicy spod wskaźnika odczytujemy poprzez LD
					mov %i1, %l2			; rozmiar
					mov 1, %l3				; i
					mov 0, %l4				; temp
					mov 0, %l5				; suma
				pętla:
					subcc %l2, 1, %l2		; n--
					bl koniec				; if n < 0 koniec
					nop
					
					smul %l3, 2, %l4		; temp = 2*i
					add %l4, 1, %l4			; temp = temp +1 = 2*i+1
					smul %l1, %l4, %l4		; temp = temp * tab[i] = (2*i+1)*tab[i]
					subcc %l5, %l4, %l5		; suma = suma - temp = suma - (2*i+1)*tab[i]
					
					smul %l5, %l5, %l5		; suma = suma * suma
					add %l0, 4, %l0			; *tab++ przesuwamy sie o 4 na kolejny element bo tyle ma int
					ld [%l0], %l1			;pobieramy nowy element
					ba pętla
					nop
				
				koniec:
					mov %l5, %i0			; zwracamy wynik w i0 bo po restore zamienia się input na output
					ret						; ret bo było save
					restore
			\end{lstlisting}
		\newpage
		\subsubsection{Propozycja rozwiązania 2}
			\begin{lstlisting}[language={[sparc]Assembler}]
				.global fun
				.proc 4
				
				;   a(n) = a(n - 1) ^ k + n * k; a(0) = 1
				fun:
					save %sp, -96, %sp
					; %i0 == n
					; %i1 == k
				
					subcc %i0, 1, %o0	; %o0 == n - 1
					bneg return1
					nop
				
					; trzeba obliczyc a(n - 1)
					mov %i1, %o1
					call fun
					nop
				
					; %o0 == a(n - 1)
					mov %i1, %l1		; %l1 == k
					mov 1, %l2			; %l2 == 1 (tu bedzie wynik potegowania)
				power:
					umul %l2, %o0, %l2
					subcc %l1, 1, %l1	; dekrementuj licznik petli
					bg power			; skok, gdy licznik > 0
					nop
				
					; %l2 == a(n - 1) ^ k
					umul %i0, %i1, %i0
					; %i0 == n * k
					add %i0, %l2, %i0
					; %i0 == a(n - 1) ^ k + n * k == a(n)
					ba return
					nop
				
				return1:
					mov 1, %i0
				return:
					ret
					restore
			\end{lstlisting}
		
	\newpage
	\subsection{2013, I termin, Jerzy Respondek}
		\subsubsection{Treść}
			Napisz w asemblerze procesora SPARC funkcję zwracającą \emph{a(n)} wyliczoną z poniższego wzoru rekurencyjnego, a pobierającą dwa argumenty: \emph{n} oraz \emph{k}, obydwa typu \textit{unsigned int}.
			$$ a(n)=a(n-1)^k+n\cdot k,\;\;\;\;a(0)=1,\;\;\;n=1,2,3,... $$
		\subsubsection{Rozwiązanie}
			Podobno otrzymano za to 5, choć rozwiązanie NIE JEST w pełni poprawne.
			\begin{lstlisting}[language={[sparc]Assembler}]
				.global fun
				.proc 4
				
				fun:
					save %sp,-96,%sp
					
					mov %i0, %l0			; l0 - n
					mov %i1, %l1			; l1 - k
					mov 0, %l2				; power
					mov 1, %l3				; a(n) = 1
					
					subcc %i0, 1, %i0
					bl theEnd				; if n = 0 then jump to theEnd
					nop
					
					mov %l0, %l2			; power = n
					smul %l2, %l1, %l2		; power = power * k
					add %l2, %l1, %l2		; power = power + k
					
					call fun				; call recursion
					mov %i0, %l3			; get score of recursion
				
				expo:
					smul %l3, %l3, %l3
					subcc %l2, 1, %l2
					bl theEnd
					nop
					ba expo
					nop
				theEnd:
					mov %l3, %i0			; return score
					ret
					restore
				
				.end
			\end{lstlisting}
		
		
		
		
		
		
		
		
		
		
		
		
		
		
		
		
		
		
		
		
		
		
		
		
		
		