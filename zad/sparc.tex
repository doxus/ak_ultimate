% !TeX spellcheck = pl_PL
\newpage

\section{Sparc}
	Uwagi:
	\begin{itemize}
		\item W języku asemblera SPARC komentarze są oznaczane przez znak wykrzyknika (!), a nie średnika (;). W listingach są średniki ze względu na wbudowany listingu asemblera w latexie.
	\end{itemize}
	
	\subsection{Laborka: min, max oraz max - min}
		Ocena nieznana.
		\subsubsection{Funkcja w języku C}
			\begin{lstlisting}[language=C]
				#include <stdio.h>
				extern int minmax(int *tab, int n, int *max, int *min);
				
				int main()
				{
					int i, N, *tab;
					int max, min, span;
					scanf("%i", &N);	
					if (N < 0) {
						printf("N < 0!\n");
						return -1;
					}
					tab = malloc(N*sizeof(*tab));
					for(i = 0; i < N; ++i)
						scanf("%i", tab + i);
					span = minmax(tab, N, &max, &min);
					printf("min = %i, max = %i, span = %i\n",
					min, max, span);
					free(tab);
					return 0;
				}
			\end{lstlisting}
		\newpage
		\subsubsection{Odpowiednik w SPARCu}
			\begin{lstlisting}[language={[sparc]Assembler}]
				.global minmax
				.proc 4
				; rejestry:
				; %i0 - adres do tablicy
				; %i1 - ilosc liczb (N)
				; %i2 - adres do max
				; %i3 - adres do min
				;
				; %l6 - pomocnicza do przechowania przesuniecia w bajtach
				; %l7 - pomocnicza do porownywania z min/max
				; %l1 - max
				; %l2 - min
				minmax:
					save %sp, -96, %sp ; przesuniecie okienka
					; zaladuj wartosci dla max i min, gdy n <= 0
					mov 0, %l1
					mov 0, %l2
					; sprawdz czy n > 0
					subcc %i1, 1, %i1
					blt end
					nop
					; zaladuj startowe max (%l1) i min(%l2) z pierwszej liczby
					ld [%i0], %l1
					mov %l1, %l2
				petla:
					; sprawdz koniec petli
					blt end
					; wylicz adres i zaladuj kolejna liczbe
					smul %i1, 4, %l6
					ld [%i0+%l6], %l0	; %l0 - obecna liczba
				
					; update max
					subcc %l0, %l1, %l7	; %l1 - max
					blt next
					nop
					mov %l0, %l1
				next:
					; update min
					subcc %l0, %l2, %l7	; %l2 - min
					bgt next2
					nop
					mov %l0, %l2
				
				next2:
					ba petla
					subcc %i1, 1, %i1
				
				end:
					; zapisz wynik
					st %l1, [%i2]
					st %l2, [%i3]
					sub %l1, %l2, %i0
					ret
					restore ; odtworzenie okienka
			\end{lstlisting}
			
	\subsection{Laborka, szukanie min i max}
		Ocena nieznana.
		\begin{lstlisting}[language={[sparc]Assembler}]
			.global _start
			;' defajny:'
			define (ilosc, 4)
			define (delta, 3)
			;' rejestry:'
			define (adres, l3)			;' %l3 - przesuniecie tablicy'
			define (min, l1)			;' %l1 - min'
			define (max, l2)			; ' %l2 - max'
			define (indeks, l4)			;' %l4 - indeks'
			define (pobrane, l5)		;' %l5 - pobrana wartosc'
			define (tmp, l6)			;' %l6 - pomocniczy'
			define (rozpietosc, o0)		;' %o1 - zwrocona rozpietosc'
			;' kod:'
			_start:
				save %sp, -96, %sp
				mov 0, %adres
				mov 0, %indeks
				;'zapis wartosci poczatkowych min i max'
				ld [%adres], %min
				ld [%adres], %max
				;'dodajemy delte do pierwszej komorki'
				add %min, delta, %tmp
				st %tmp, [%adres]
			petla:   
				add %indeks, 1, %indeks		;'indeks++'
				add %adres, 4, %adres		;'przesuniecie na kolejny element'
				cmp %indeks, ilosc
				be end
				nop
				;'zwiekszamy o delte'
				ld [%adres], %pobrane
				add %pobrane, delta, %pobrane
				st %pobrane, [%adres]
			;'sprawdzamy czy pobranma wartosc jest mniejsza niz min'
				subcc %pobrane, %min, %tmp
				bg dalej
				nop
				mov %pobrane, %min
				ba petla_end
				nop
			dalej:   
				;'sprawdzamy czy pobrana wartosc jest wieksza niz max'
				subcc %pobrane, %max, %tmp
				bl petla_end
				nop
				mov %pobrane, %max
			petla_end:
				ba petla
				nop
				end:
				sub %max, %min, %rozpietosc    ;'zapisujemy rozpietosc'
				ret
		\end{lstlisting}
	
	\newpage
	\subsection{2008, I termin, Jerzy Respondek}
		\subsubsection{Treść}
			Napisz funkcję w asemblerze procesora SPARC obliczającą sumę liczb naturalnych od 1 do danej $ n $ jako argument funkcji. Założyć, że $ n >= 1 $.\\
			Przykład: f(5) = 1 + 2 + 3 + 4 + 5 = 15
		\subsubsection{Propozycja rozwiązania 1}
			\begin{lstlisting}[language={[sparc]Assembler}]
				.global funkcja
				.proc 4
				funkcja:
					save %sp, -96, %sp		; trzeba tutaj to robić ???
					mov %i0, %l0			; a
					mov 1, %l1				; liczba naturalna   
					mov 0, %l2				; wynik
				pętla:
					add %l1, %l2, %l2		; liczba + suma = suma
					add %l1, 1, %l1			; liczba++
					subbcc %l0, 1, %l0		; a--
					bl koneic
					nop
					ba pętla
					nop
				koniec:
					mov %l2, %i0			; wynik
					ret
					restore
			\end{lstlisting}
		\subsubsection{Propozycja rozwiązania 2}
			\begin{lstlisting}[language={[sparc]Assembler}]
			    .global sumator
			    .proc 4
			    sumator:
				    save %sp, -96, %sp		! przesunięcie okna
				    mov %i0, %l1			! a w l1
				    mov %l1, %l0			! suma = a
				   petla:
				    subcc %l1, 1, %l1		! dekrementacja licznika
				    bneg koniec
				    add %l0, %l1, %l0		! suma += licznik
				    ba petla
			    koniec:
				    mov %l0, %i0			! zwrócenie sumy
				    ret
				    restore					! przywrócenie stanu okna
			\end{lstlisting}
			
	\newpage
	\subsection{2010, I termin, Jerzy Respondek}
		\subsubsection{Treść}
			Napisz w asemblerze procesora SPARC funkcję obliczającą sumę kwadratów wszystkich liczb całkowitych z przedziału \emph{a} do \emph{b}. Założyć $ a < b $, np.\\
			f(2, 5) = 2 * 2 + 3 * 3 + 4 * 4 + 5 * 5\\
			Nagłówek funkcji ma mieć postać:
			\begin{lstlisting}[language=C]
				int f(int a, int b)
			\end{lstlisting}
		\subsubsection{Propozycja rozwiązania}
	\newpage
	\subsection{2012, I termin, Jerzy Respondek}
		\subsubsection{Treść}
			Napisz w asemblerze procesora SPARC funkcję realizującą dokładnie tę samą operację co jej odpowiednik w języku C:
			\begin{lstlisting}[language=C]
				int f(int *tab, int n)
				{
					int i, suma = 0;
					for(i = 0; i < n; i++)
					{
						suma -= (2 * i + 1) * tab[i];
						suma *= suma;
					}
					return suma;
				}
			\end{lstlisting}
		\subsubsection{Propozycja rozwiązania 1}
			\begin{lstlisting}[language={[sparc]Assembler}]
				.global func
				.proc 4
				
				funkcja:
					save %sp, -96, %sp
					mov %i0, %l0			; wskaźnik tablicy, tak podano argument
					ld [%i0], %l1			; wartość tablicy spod wskaźnika odczytujemy poprzez LD
					mov %i1, %l2			; rozmiar
					mov 1, %l3				; i
					mov 0, %l4				; temp
					mov 0, %l5				; suma
				pętla:
					subcc %l2, 1, %l2		; n--
					bl koniec				; if n < 0 koniec
					nop
					
					smul %l3, 2, %l4		; temp = 2*i
					add %l4, 1, %l4			; temp = temp +1 = 2*i+1
					smul %l1, %l4, %l4		; temp = temp * tab[i] = (2*i+1)*tab[i]
					subcc %l5, %l4, %l5		; suma = suma - temp = suma - (2*i+1)*tab[i]
					
					smul %l5, %l5, %l5		; suma = suma * suma
					add %l0, 4, %l0			; *tab++ przesuwamy sie o 4 na kolejny element bo tyle ma int
					ld [%l0], %l1			;pobieramy nowy element
					ba pętla
					nop
				
				koniec:
					mov %l5, %i0			; zwracamy wynik w i0 bo po restore zamienia się input na output
					ret						; ret bo było save
					restore
			\end{lstlisting}
		\newpage
		\subsubsection{Propozycja rozwiązania 2}
			\begin{lstlisting}[language={[sparc]Assembler}]
				.global fun
				.proc 4
				
				;   a(n) = a(n - 1) ^ k + n * k; a(0) = 1
				fun:
					save %sp, -96, %sp
					; %i0 == n
					; %i1 == k
				
					subcc %i0, 1, %o0	; %o0 == n - 1
					bneg return1
					nop
				
					; trzeba obliczyc a(n - 1)
					mov %i1, %o1
					call fun
					nop
				
					; %o0 == a(n - 1)
					mov %i1, %l1		; %l1 == k
					mov 1, %l2			; %l2 == 1 (tu bedzie wynik potegowania)
				power:
					umul %l2, %o0, %l2
					subcc %l1, 1, %l1	; dekrementuj licznik petli
					bg power			; skok, gdy licznik > 0
					nop
				
					; %l2 == a(n - 1) ^ k
					umul %i0, %i1, %i0
					; %i0 == n * k
					add %i0, %l2, %i0
					; %i0 == a(n - 1) ^ k + n * k == a(n)
					ba return
					nop
				
				return1:
					mov 1, %i0
				return:
					ret
					restore
			\end{lstlisting}
		
	\newpage
	\subsection{2013, I termin, Jerzy Respondek}
		\subsubsection{Treść}
			Napisz w asemblerze procesora SPARC funkcję zwracającą \emph{a(n)} wyliczoną z poniższego wzoru rekurencyjnego, a pobierającą dwa argumenty: \emph{n} oraz \emph{k}, obydwa typu \textit{unsigned int}.
			$$ a(n)=a(n-1)^k+n\cdot k,\;\;\;\;a(0)=1,\;\;\;n=1,2,3,... $$
		\subsubsection{Rozwiązanie nr 1 by Doxus}
			\begin{lstlisting}[language={[sparc]Assembler}]
				.global _start
				
				_start:
					MOV		0x05,	%g1					;! g1 - K
					MOV		0x0A,	%o7					;! rej o7 i i7 -> N (lokalne)
					MOV		%o7,	%g7					;! N absolutne
				
				_petla:
					SAVE	%sp,	-96,	%sp			;! otworzenie okna
				
					SUBcc	%i7,	0x00,	%g0			;! sprawdzenie, czy to dno rekurencji
					BE		_nzero
					NOP
					
					SUB		%i7,	0x01,	%o7			;! wykonanie rekurencji
					BA		_petla
					NOP
					
				_nzero:
					MOV		0x00,	%i5
					MOV		0x00,	%g2					;! g2 temp n
				_petlapowrot:
					RESTORE								;! zamknięcie koła
					
					MOV		%i5,	%l0					;! obliczenia
					MOV		0x01,	%l1					;! temp k
				_petlamnoz:
					UMUL	%i5,	%l0,	%l0			;! obliczenia zgodnie ze wzorem
					ADD		%l1,	0x01,	%l1
					SUBcc	%l1,	%g1,	%g0
					BNE		_petlamnoz
					
					UMUL	%g2,	%g1,	%l2
					ADD		%l1,	%l2,	%o5
				
					ADD		%g2,	0x01,	%g2
					
					SUBcc	%g2,	%g7,	%g0			;! czy koniec odkręcania koła
					BLE		_petlapowrot
					NOP
				
					MOV		%i5,	%g1					;! g1 - wynik koncowy
					NOP									;! kuniec
			\end{lstlisting}
		\subsubsection{Rozwiązanie nr 2}
			Podobno otrzymano za to 5, choć rozwiązanie NIE JEST w pełni poprawne.
			\begin{lstlisting}[language={[sparc]Assembler}]
				.global fun
				.proc 4
				
				fun:
					save %sp,-96,%sp
				
					mov %i0, %l0			; l0 - n
					mov %i1, %l1			; l1 - k
					mov 0, %l2				; power
					mov 1, %l3				; a(n) = 1
				
					subcc %i0, 1, %i0
					bl theEnd				; if n = 0 then jump to theEnd
					nop
					
					mov %l0, %l2			; power = n
					smul %l2, %l1, %l2		; power = power * k
					add %l2, %l1, %l2		; power = power + k
				
					call fun				; call recursion
					mov %i0, %l3			; get score of recursion
				
				expo:
					smul %l3, %l3, %l3
					subcc %l2, 1, %l2
					bl theEnd
					nop
					ba expo
					nop
				theEnd:
					mov %l3, %i0			; return score
					ret
					restore
				
				.end
			\end{lstlisting}
		
	\newpage
	\subsection{2014, 0 termin, Jerzy Respondek}
		\subsubsection{Treść}
			Napisz w asemblerze procesora SPARC funkcję realizującą dokładnie tę samą operację, co jej odpowiednik w języku C:
			\begin{lstlisting}[language=C]
			int f(int *tab, int n) 
			{
				int i, suma = 0;
				for(i = 0; i < n; i++)
				{
					suma += i*tab[i];
				}
				return suma;
			}
			\end{lstlisting}
		\subsubsection{Rozwiązanie}
			\begin{lstlisting}[language={[sparc]Assembler}]
			.global _start
			
			_start:
			! i1 wskaznik na pocz tablicy
			! i2 n
			
			! i0 - wyjsciowa suma (RESTORE spowoduje ze bedzie to w rej.
			! wyjsciowych funkcji nadrzędnej
			
			! l0, l1 - wskaznik na el. tablicy, n
			! l2 - iterator
			! l3 - suma
			
			mov %i1, %l0
			mov %i2, %l1
			mov 0x00, %l2
			mov 0x00, %l3
			_loop:
			! if sprawdzają czy i < n
			subcc %l2, %l1, %g0
			bge _koniec
			nop
			
			ld [%l0], %l7
			umul %l7, %l2, %l7
			add %l7, %l3, %l3
			
			add %l2, 0x01, %l2
			
			ba _loop
			add %l0, 0x04, %l0		!można zamienić miejscami i dać nop, ale tak optymalniej
			
			_koniec:
			mov %l3, %i0
			\end{lstlisting}