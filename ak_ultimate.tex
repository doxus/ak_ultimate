% !TeX spellcheck = pl_PL
\documentclass[a4paper,twoside]{article}
\usepackage{polski}
\usepackage[utf8]{inputenc}
\usepackage{graphicx}
\usepackage{amsmath}

\usepackage[unicode, bookmarks=true]{hyperref} %do zakładek

\setlength{\textheight}{24cm}
\setlength{\textwidth}{15.92cm}
\setlength{\footskip}{10mm}
\setlength{\oddsidemargin}{0mm}
\setlength{\evensidemargin}{0mm}
\setlength{\topmargin}{0mm}
\setlength{\headsep}{5mm}

\begin{document}
\bibliographystyle{plain}

\begin{titlepage}
\title{\huge Architektura komputerów - ULTIMATE}
\author{\large SonMati \\ Doxus}
\maketitle
\end{titlepage}

%===============================================================================
% *** PYTANIA I ODPOWIEDZI *****************************************************
%===============================================================================
\part*{Pytania i odpowiedzi}
\section{Moc obliczeniowa komputerów wektorowych :}
	\begin{itemize}
    \item Zależy od liczby stopni potoku
    \item \textbf{Jest odwrotnie proporcjonalna do długości taktu zegarowego}
    \item Jest wprost proporcjonalna do długości taktu zegarowego
    \item Zależy odwrotnie proporcjonalnie od liczby jednostek potokowych połączonych łańcuchowo
    \item \textbf{Zmierza asymptotycznie do wartości maksymalnej wraz ze wzrostem długości wektora}
    \item Nie zależy od długości wektora
    \item Zależy liniowo od długości wektora
    \end{itemize}

\section{Czy poniższa lista jest rosnąco uporządkowana według skalowalności:}
	\begin{itemize}
    \item Systemy ściśle połączone, systemy ze wspólną pamięcią, systemy SMP
    \item \textbf{Systemy ze wspólna magistralą, systemy wielomagistralowe, systemy z przełącznicą krzyżową}
    \item Systemy SMP, systemy z pamięcią wieloportową, systemy z przełącznicą krzyżową
    \item NUMA, MPP, SMP
    \item \textbf{Systemy z pamięcią wspólną, systemy o niejednorodnym dostępie do pamięci, z pamięcią rozproszoną}
    \item Systemy SMP, NUMA, klastry, UMA
    \item \textbf{Systemy symetryczne, o niejednorodnym dostępie do pamięci, systemy z przesyłem komunikatów}
    \end{itemize}
    
\section{Komputery macierzowe}
	\begin{itemize}
    \item \textbf{Mają w liście rozkazów m.in. rozkazy operujące na wektorach danych}
    \item Mają macierzowe potokowe układy arytmetyczne
    \item Mają w typowych rozwiązaniach zestaw pełnych procesów połączonych siecią połączeń
    \item \textbf{Wykonują synchroniczną operację wektorową w sieci elementów przetwarzających}
    \end{itemize}
    
\section{Przetwarzanie potokowe}
	\begin{itemize}
    \item Nie jest realizowane dla operacji zmiennoprzecinkowych
    \item Nie jest realizowane w procesorach CISC
    \item \textbf{Daje przyspieszenie nie większe od liczby segmentów (stopni) jednostki potokowej}
    \item W przypadku wystąpienia zależności między danymi wywołuje błąd i przerwanie wewnętrzne
    \item Jest realizowane tylko dla operacji zmiennoprzecinkowych
    \end{itemize}

\section{W procesorach superskalarnych}
	\begin{itemize}
    \item \textbf{Liczba rozkazów, które procesor może wykonać w 1 takcie zależy od liczby jednostek potokowych w procesorze}
    \item Liczba rozkazów, które procesor może wykonać w jednym takcie, zależy od liczby stopni potoku
    \item Liczba rozkazów pobieranych z pamięci, w każdym takcie musi przekraczać liczbę jednostek potokowych
    \item \textbf{Liczba rozkazów, które procesor może wykonać w taktach zależy od liczby jednostek potokowych w procesorze}
    \end{itemize}

\section{Systemy SMP}
	\begin{itemize}
    \item Wykorzystują protokół MESI do sterowania dostępem do wspólnej magistrali
    \item Posiadają skalowalne procesory
    \item Posiadają pamięć fizycznie rozproszoną, ale logicznie wspólną
    \end{itemize}

\section{Komputery wektorowe}
	\begin{itemize}
    \item Posiadają jednostki potokowe o budowie wektorowej
    \item \textbf{Posiadają w liście rozkazów m.in. rozkazy operujące na wektorach danych}
    \item \textbf{Wykorzystują od kilku do kilkunastu potokowych jednostek arytmetycznych}
    \item Posiadają listę rozkazów operujących wyłącznie na wektorach
    \end{itemize}

\section{Procesory wektorowe}
	\begin{itemize}
    \item \textbf{Mogą być stosowane w systemach wieloprocesorowych}
    \item \textbf{Mają listę rozkazów operującą jedynie na wektorach}
    \item \textbf{Mają moc kilka razy większą od procesorów skalarnych}
    \end{itemize}

\section{Systemy MPP są zbudowane z węzłów którymi mogą być}
	\begin{itemize}
    \item \textbf{Systemy SMP}
    \item Klastry
    \item Konstelacje
    \item \textbf{Systemy NUMA}
    \item \textbf{Procesory}
    \end{itemize}

\section{W architekturze NUMA}
	\begin{itemize}
    \item \textbf{Dane są wymieniane między węzłami w postaci linii pamięci podręcznej (PaP)}
    \item Spójność PaP węzłów jest utrzymywana za pomocą protokołu MESI
    \item Czas dostępu do pamięci lokalnej w węźle jest podobny do czasu dostępu do pamięci nielokalnej
    \item \textbf{Czas zapisu danych do pamięci nielokalnej może być znacznie dłuższy od czasu odczytu z tej pamięci}
    \item \textbf{Każdy procesor ma dostęp do pamięci operacyjnej każdego węzła}
    \item Procesy komunikują się poprzez przesył komunikatów
    \item \textbf{Pamięć operacyjna jest rozproszona fizycznie pomiędzy węzłami, ale wspólna logicznie}
    \end{itemize}

\section{Mechanizmy potokowe stosowane są w celu}
	\begin{itemize}
    \item Uszeregowania ciągu wykonywanych rozkazów
    \item \textbf{Uzyskania równoległej realizacji rozkazów}
    \item \textbf{Przyspieszenia realizacji rozkazów}
    \end{itemize}

\section{Protokół MESI}
	\begin{itemize}
    \item Jest wykorzystywany do sterowania dostępem do magistrali w systemie SMP
    \item \textbf{Zapewnia spójność pamięci cache w systemie SMP}
    \item Służy do wymiany komunikatów w systemie MPP
    \item Chroni przed hazardem w proc superskalarnych
    \end{itemize}
    
\section{Mechanizm skoków opóźnionych}
	\begin{itemize}
    \item \textbf{Polega na opóźnianiu wykonywania skoku do czasu wykonania rozkazu następnego za skokiem}
    \item Wymaga wstrzymania potoku na jeden takt
    \item Powoduje błąd na końcu pętli
    \item \textbf{Wymaga umieszczenia rozkazu NOP za rozkazem skoku lub reorganizacje programu}
    \end{itemize}
    
\section{Charakterystyczne cechy architektury MPP}
	\begin{itemize}
    \item Spójność pamięci podręcznej wszystkich węzłów
    \item \textbf{Fizycznie rozproszona PaO}
    \item Fizycznie rozproszona PaO, ale logicznie wspólna
    \item \textbf{Przesył komunikatów między procesorami}
    \item Niska skalowalność
    \item Jednorodny dostęp do pamięci wszystkich węzłów
    \end{itemize}

\section{Jak można ominąć hazard danych}
	\begin{itemize}
    \item Poprzez rozgałęzienia
    \item Poprzez uproszczenie adresowania - adresowanie bezpośrednie
    \item \textbf{Przez zamianę rozkazów}
    \end{itemize}

\section{Cechy architektury CISC}
	\begin{itemize}
    \item Czy może być wykonana w VLIW
    \item \textbf{Czy występuje model wymiany danych typu pamięć - pamięć}
    \item Jest mała liczba rozkazów
    \end{itemize}

\section{Cechy architektury RISC}
	\begin{itemize}
    \item Czy występuje model wymiany danych typu rej-rej
    \item Jest mała liczba trybów adresowania
    \item Jest wykonywanych kilka rozkazów w jednym takcie
    \item Jest wykonywanych kilka rozkazów w jednym takcie (w danej chwili czasu)
    \item textbf{Jest wykonywanych kilka instrukcji procesora w jednym rozkazie asemblerowym}
    \item Układ sterowania w postaci logiki szytej
    \end{itemize}

\section{Przepustowość (moc obliczeniowa) dużych komputerów jest podawana w}
	\begin{itemize}
    \item \textbf{GFLOPS}
    \item Liczbie instrukcji wykonywanych na sekundę
    \item \textbf{Liczbie operacji zmiennoprzecinkowych na sekundę}
    \item Mb/sek
    \end{itemize}

\section{Podstawą klasyfikacji Flynna jest}
	\begin{itemize}
    \item Liczba jednostek przetwarzających i sterujących w systemach komputerowych
    \item Protokół dostępu do pamięci operacyjnej
    \item \textbf{Liczba strumieni rozkazów i danych w systemach komputerowych}
    \item Liczba modułów pamięci operacyjnej w systemach komputerowych
    \end{itemize}

\section{Liczba modułów pamięci operacyjnej w systemach komputerowych}
	\begin{itemize}
    \item \textbf{Macierzy elementów przetwarzających}
    \item Zestawu procesorów superskalarnych
    \item \textbf{Technologii MMX}
    \item Sieci połączeń typu krata
    \item \textbf{Potokowych jednostek arytmetycznych}
    \end{itemize}

\section{Architektura superskalarna}
	\begin{itemize}
    \item Dotyczy systemów SMP
    \item Wymaga zastosowania protokołu MESI
    \item \textbf{Umożliwia równoległe wykonywanie kilku rozkazów w jednym procesorze}
    \item Wywodzi się z architektury VLIW
    \end{itemize}

\section{Klastry}
	\begin{itemize}
    \item Mają średnią skalowalność
    \item Wykorzystują model wspólnej pamięci
    \item \textbf{W węzłach mogą wykorzystywać systemy SMP}
    \item \textbf{Do komunikacji między procesami wykorzystują przesył komunikatów}
    \item Wykorzystują przełącznicę krzyżową jako sieć łączącą węzły
    \item \textbf{W każdym węźle posiadają pełną instalację systemu operacyjnego}
    \end{itemize}

\section{Pojęcie równoległości na poziomie rozkazów:}
	\begin{itemize}
    \item Dotyczy architektury MIMD
    \item \textbf{Odnosi się m.in. do przetwarzania potokowego}
    \item Dotyczy architektury MPP
    \item \textbf{Dotyczy m.in. architektury superskalarnej}
    \end{itemize}

\section{Systemy wieloprocesorowe z pamięcią wspólną}
	\begin{itemize}
    \item Zapewniają jednorodny dostęp do pamięci
    \item \textbf{Mogą wykorzystywać procesory CISC}
    \item \textbf{Są wykorzystywane w klastrach}
    \item Wykorzystują przesył komunikatów między procesorami
    \item \textbf{Wykorzystują katalog do utrzymania spójności pamięci podręcznych}
    \end{itemize}

\section{Hazard danych}
	\begin{itemize}
    \item \textbf{Czasami może być usunięty przez zmianę kolejności wykonania rozkazów}
    \item Nie występuje w architekturze superskalarnej
    \item Jest eliminowany przez zastosowanie specjalnego bitu w kodzie programu
    \item Może wymagać wyczyszczenia potoku i rozpoczęcia nowej (…)
    \end{itemize}

\section{Przetwarzanie wielowątkowe}
	\begin{itemize}
    \item \textbf{Zapewnia lepsze wykorzystanie potoków}
    \item \textbf{Minimalizuje straty wynikające z chybionych odwołań do pamięci podręcznej}
    \item \textbf{Wymaga zwielokrotnienia zasobów procesora (rejestry, licznik rozkazów…)}
    \item Nie może być stosowane w przypadku hazardu danych
    \end{itemize}

\section{Okna rejestrów}
	\begin{itemize}
    \item Chronią przez hazardem danych
    \item \textbf{Minimalizują liczbę odwołań do pamięci operacyjnej przy operacjach wywołania procedur}
    \item Są charakterystyczne dla architektury CISC
    \item Są zamykane po błędnym przewidywaniu wykonania skoków warunkowych
    \item \textbf{Są przesuwane przy operacjach wywołania procedur}
    \end{itemize}

\section{Tablica historii rozgałęzień}
	\begin{itemize}
    \item \textbf{Zawiera m.in. adresy rozkazów (?) rozgałęzień}
    \item \textbf{Pozwala zminimalizować liczbę błędnych przewidywań rozgałęzień w zagnieżdżonej… (pętli?)}
    \item Nie może być stosowana w procesorach CISC
    \item Jest obsługiwana przez jądro systemu operacyjnego
    \end{itemize}

\section{Rozkazy wektorowe}
	\begin{itemize}
    \item Nie mogą być wykonywane bez użycia potokowych jednostek arytmetycznych
    \item \textbf{W komputerach wektorowych ich czas wykonania jest wprost proporcjonalny do długości wektora}
    \item \textbf{Są charakterystyczne dla architektury SIMD}
    \item Są rozkazami dwuargumentowymi i w wyniku zawsze dają wektor
    \end{itemize}

\section{Model SIMD}
	\begin{itemize}
    \item Był wykorzystywany tylko w procesorach macierzowych
    \item \textbf{Jest wykorzystywany w multimedialnych rozszerzeniach współczesnych procesorów}
    \item \textbf{Jest wykorzystywany w heterogenicznej architekturze PowerXCell}
    \item \textbf{Zapewnia wykonanie tej samej operacji na wektorach argumentów}
    \end{itemize}

\section{Przesył komunikatów}
	\begin{itemize}
    \item \textbf{Ma miejsce w systemach MPP}
    \item W systemach MPP II-giej generacji angażuje wszystkie procesory na drodze przesyłu
    \item \textbf{Ma miejsce w klastrach}
    \end{itemize}

\section{Cechami wyróżniającymi klastry są}
	\begin{itemize}
    \item \textbf{niezależność programowa każdego węzła}
    \item Fizycznie rozproszona, ale logicznie wspólna pamięć operacyjna
    \item Nieduża skalowalność
    \item \textbf{Na ogół duża niezawodność}
    \end{itemize}

\section{Systemy wieloprocesorowe z pamięcią rozproszoną}
	\begin{itemize}
    \item Wyróżniają się bardzo dużą skalowalnością
    \item Są budowane z węzłów, którymi są klastry
    \item \textbf{Realizują synchronicznie jeden wspólny program}
    \item Wymagają zapewnienia spójności pamięci podręcznych pomiędzy węzłami
    \end{itemize}

\section{Problemy z potokowym wykonywaniem rozkazów skoków (rozgałęzień) mogą być wyeliminowane lub ograniczone przy pomocy}
	\begin{itemize}
    \item Zapewnienia spójności pamięci podręcznej
    \item \textbf{Tablicy historii rozgałęzień}
    \item Techniki wyprzedzającego pobrania argumentu
    \item \textbf{Wystawienia do programu rozkazów typu „nic nie rób”}
    \item Protokołu MESI
    \item \textbf{Wykorzystania techniki skoków opóźniających}
    \item Technologii MMX
    \end{itemize}

\section{W architekturze ccNUMA}
	\begin{itemize}
    \item \textbf{Każdy procesor ma dostęp do pamięci operacyjnej każdego węzła}
    \item Spójność pamięci pomiędzy węzłami jest utrzymywana za pomocą protokołu MESI
    \item \textbf{Dane są wymieniane między węzłami w postaci linii pamięci podręcznej}
    \item \textbf{Pamięć operacyjna jest fizycznie rozproszona pomiędzy węzłami, ale wspólna logicznie}
    \end{itemize}

\section{Dla sieci systemowych (SAN) są charakterystyczne}
	\begin{itemize}
    \item \textbf{Przesył komunikatów w trybie zdalnego DMA}
    \item \textbf{Bardzo małe czasy opóźnień}
    \item Topologia typu hipersześcian
    \item Niska przepustowość
    \end{itemize}

\section{W systemach wieloprocesorowych katalog służy do}
	\begin{itemize}
    \item Śledzenia adresów w protokole MESI
    \item Sterowania przesyłem komunikatów
    \item \textbf{Utrzymania spójności pamięci w systemach o niejednorodnym dostępie do pamięci}
    \item \textbf{Realizacji dostępu do nielokalnych pamięci w systemach NUMA}
    \end{itemize}

\section{W procesorach superskalarnych}
	\begin{itemize}
    \item \textbf{Jest możliwe równoległe wykonywanie kilku rozkazów w jednym procesorze(rdzeniu) }
    \item \textbf{Rozszerzenia architektury wykorzystujące model SIMD umożliwiają wykonanie rozkazów wektorowych}
    \item Nie występuje prawdziwa zależność danych
    \item \textbf{Mogą wystąpić nowe formy hazardu danych: zależności wyjściowe między rozkazami oraz antyzależności}
    \end{itemize}

\section{Efektywne wykorzystanie równoległości na poziomie danych umożliwiają}
	\begin{itemize}
    \item \textbf{Komputery wektorowe}
    \item \textbf{Komputery macierzowe}
    \item \textbf{Klastry}
    \item \textbf{Procesory graficzne}
    \item \textbf{Rozszerzenia SIMD procesorów superskalarnych}
    \end{itemize}

\section{Wielowątkowość współbieżna w procesorze wielopotokowym zapewnia}
	\begin{itemize}
    \item \textbf{Możliwość wprowadzenia rozkazów różnych wątków do wielu potoków}
    \item \textbf{Realizację każdego z wątków do momentu wstrzymania któregoś rozkazu z danego wątku}
    \item Przełączanie wątków co takt
    \item Automatyczne przemianowanie rejestrów
    \end{itemize}

\section{Architektura CUDA}
	\begin{itemize}
    \item \textbf{Umożliwia bardzo wydajne wykonywanie operacji graficznych}
    \item \textbf{Stanowi uniwersalną architekturę obliczeniowa połączoną z równoległym modelem programistycznym}
    \item \textbf{Realizuje model obliczeniowy SIMT}
    \item Jest podstawą budowy samodzielnych, bardzo wydajnych komputerów
    \end{itemize}

\section{Spójność pamięci podręcznych w procesorze wielordzeniowym może być m.in. zapewniona za pomocą}
	\begin{itemize}
    \item Przełącznicy krzyżowej
    \item Katalogu
    \item \textbf{Protokołu MESI}
    \item Wspólnej magistrali
    \end{itemize}

\section{Metoda przemianowania rejestrów jest stosowana w celu eliminacji}
	\begin{itemize}
    \item Błędnego przewidywania rozgałęzień
    \item Chybionego odwołania do pamięci podręcznej
    \item Prawdziwej zależności danych
    \item \textbf{Zależności wyjściowej między rozkazami}
    \item \textbf{Antyzależności między rozkazami}
    \end{itemize}

\section{W systemach wieloprocesorowych o architekturze CC-NUMA}
	\begin{itemize}
    \item \textbf{Spójność pamięci wszystkich węzłów jest utrzymywana za pomocą katalogu}
    \item \textbf{Pamięć operacyjna jest rozproszona fizycznie pomiędzy węzłami, ale wspólna logicznie}
    \item Każdy procesor ma bezpośredni dostęp do pamięci operacyjnej każdego węzła
    \item Dane są wymieniane między węzłami w postaci linii pamięci podręcznej
    \end{itemize}

\section{W tablicy historii rozgałęzień z 1 bitem historii można zastosować następujący algorytm przewidywania (najbardziej złożony)}
	\begin{itemize}
    \item Skok opóźniony
    \item Przewidywanie, że rozgałęzienie(skok warunkowy) zawsze nastąpi
    \item Przewidywanie, że rozgałęzienie nigdy nie nastąpi
    \item \textbf{Przewidywanie, że kolejne wykonanie rozkazu rozgałęzienia będzie przebiegało tak samo jak poprzednie}
    \item Wstrzymanie napełniania potoku
    \end{itemize}

\section{Do czynników tworzących wysoką niezawodność klastrów należą}
	\begin{itemize}
    \item \textbf{Mechanizm mirroringu dysków}
    \item \textbf{Dostęp każdego węzła do wspólnych zasobów(pamięci zewnętrznych)}
    \item \textbf{Redundancja węzłów}
    \item Mechanizm "heartbeat"
    \item Zastosowanie procesorów wielordzeniowych w węzłach
    \end{itemize}


\newpage
%===============================================================================
%*******************************************************************************
%===============================================================================
\part*{Opracowanie wykładów}
\section*{Historia rozwoju komputerów}
	\begin{enumerate}
    \item Liczydło
    \item Pascalina - maszyna licząca Pascala (dodawanie i odejmowanie)
    \item Maszyna mnożąca Leibniza (dodawanie, odejmowanie, mnożenie, dzielenie, pierwiastek kwadratowy
    \item Maszyna różnicowa - Charles Babbage, obliczanie wartości matematycznych do tablic
    \item Maszyna analityczna - Charles Babvage, programowalna za pomocą kart perforowanych
  	\item Elektryczna maszyna sortująca i tabelaryzująca Holleritha 1890
    \item Kalkulator elektromechaniczny Mark I, tablicowanie funkcji, całkowanie numeryczne, rozwiązywanie równań różniczkowych, rozwiązywanie układów równań liniowych, analiza harmoniczna, obliczenia statystyczne
    \item Maszyny liczące Z1: pamięć mechaniczna, zmiennoprzecinkowa reprezentacja liczb, binarna jednostka zmiennoprzecinkowa
    \item Z3: Pierwsza maszyna w pełni automatyczna, kompletna w sensie Turinga, pamięć przekaźnikowa
    \item Colossus i Colossus 2
    \item ENIAC
    \item EDVAC - J. von Neumann (wtedy utworzył swoją architekturę) \\
    	\begin{figure}[h]
		\centering
		\includegraphics[scale=0.1]{architektura_von_Neumanna.png}
		\end{figure}
    \item UNIVAC I (pierwszy udany komputer komercyjny)
    \item IBM 701, potem 709
    \item po 1955 zaczyna się zastosowanie tranzystorów w komputerach (komputery II generacji)
    \item po 1965 komputery III generacji z układami scalonymi
    \item od 1971 komputery IV generacji - z układami scalonymi wielkiej skali inegracji VLSI
    \end{enumerate}
    
\section*{Charakterystyka architektury CISC}
	\subsection*{Przyczyny rozwoju architektury CISC}
    	\begin{itemize}
        \item drogie, małe i wolne pamięci komputerów
        \item duża popularność mikroprogramowalnych układów sterujących (prostych w rozbudowie)
        \item dążenie do uproszczenia kompilatorów
        \end{itemize}
    
    \subsection*{Cechy architektury CISC}
    	\begin{itemize}
        \item duża liczba rozkazów (z czego te najbardziej zaawansowane i tak nie były używane)
        \item model obliczeń pamięć - pamięć
        \item duża ilość trybów adresowania (związane z modelem obliczeń)
        \item zróżnicowana długość słowa rozkazu
        \item niewiele rejestrów (bo były droższe niż pamięć i przy przełączaniu kontekstu obawiano się )
        \end{itemize}
   
   \textbf{\large CIEKAWOSTKA:} Przeanalizowano jakieś tam programy i w procesorze VAX 20\% najbardziej złożonych rozkazów odpowiadało za 60\% kodu, stanowiąc przy tym ok 0.2\% wywołań.\\ W procesorze MC68020 71\% rozazów nie zostało nawet użytych w badanych programach
   
	\section*{Architektura RISC}
   		\subsection*{TODO}
        
    \section*{Procesory superskalarne}
    	\subsection*{Cechy architektury superskalarnej}
        	\begin{itemize}
            \item Możliwośc wykonania kilku rozkazów w jednym takcie, co powoduje konieczność:
            \item Kilka jednostek potokowych
            \item Konieczność załadowania kilku rozkazów z pamięci operacyjnej w jednym takcie procesora
            \end{itemize}
            
		
        \subsection*{Zależności między rozkazami i sposoby ich eliminacji}
        	\begin{itemize}
            \item \textbf{Prawdziwa zależność danych:} Występuje w momencie kiedy jeden rozkaz wymaga argumentu obliczanego przez poprzedni rozkaz. Eliminowane za pomocą "wyprzedzającego pobierania argumentu" - dana nie jest zapisywana do rejestru, tylko pobierana bezpośrednio z 
            \end{itemize}
	
%===============================================================================	
%*******************************************************************************
%===============================================================================
\end{document}